\section{环}

\subsection{环的基本概念}
\begin{definition}[环 Ring]
    设 $R$ 是一个非空集合,$+$ 和 $\cdot$ 是定义在 $R$ 上的两种代数运算,如果对任意 $a,b,c \in R$,都有
    \begin{enumerate}
        \item $(R,+)$ 是一个交换群;
            \begin{enumerate}
                \item 封闭性:$a+b \in R$;
                \item 结合律:$(a+b)+c=a+(b+c)$;
                \item 交换律:$a+b=b+a$;
                \item 存在加法单位元:存在 $0 \in R$,使得对任意 $a \in R$,都有 $a+0=0+a=a$;
                \item 存在加法逆元:对任意 $a \in R$,存在 $-a \in R$,使得 $a+(-a)=(-a)+a=0$。
            \end{enumerate}
        \item $(R,\cdot)$ 是一个半群;
            \begin{enumerate}
                \item 封闭性:$a \cdot b \in R$;
                \item 结合律:$(a \cdot b) \cdot c=a \cdot (b \cdot c)$。
            \end{enumerate}
        \item 乘法对加法的分配律:
            \begin{enumerate}
                \item 左分配律:$a \cdot (b+c)=a \cdot b + a \cdot c$;
                \item 右分配律:$(a+b) \cdot c=a \cdot c + b \cdot c$。
            \end{enumerate}
    \end{enumerate}
    则称 $(R,+,\cdot)$ 是一个\textbf{环},$+$ 和 $\cdot$ 分别称为环的\textbf{加法}和\textbf{乘法}。
    如果 $R=\{0\}$,则称 $(R,+,\cdot)$ 是一个\textbf{平凡环 Trivial Ring};否则称为\textbf{非平凡环 Non-trivial Ring}。
    \label{def:ring}
\end{definition}

\begin{definition}[交换环 Commutative Ring]
    设 $ (R,+,\cdot) $ 为环。如果乘法 $\cdot$ 满足交换律,则称 $ (R,+,\cdot) $ 为交换环。
    \label{def:commutative_ring}
\end{definition}

\begin{definition}[无零因子环 Ring without Zero Divisors]
    设 $ (R,+,\cdot) $ 为环。如果对任意 $ a,b\in R $,$ a\neq 0 $,$ b\neq 0 $,均有 $ a\cdot b \neq 0 $,则称环 $ (R,+,\cdot) $ 为无零因子环。
    \label{def:ring_without_zero_divisors}
\end{definition}

\begin{definition}[含幺环 Ring with identity]
    设 $ (R,+,\cdot) $ 为环。如果存在乘法单位元 $1$ 并且 $1\neq 0$,则称环 $ (R,+,\cdot) $ 为含幺环。也即 $(R,\cdot)$ 是幺半群,$ (R,+,\cdot) $ 为含幺环。
    \label{def:ring_with_identity}
\end{definition}

\begin{definition}[整环 Integral Domain]
    设 $ (R,+,\cdot) $ 为含幺环。如果 $ (R,+,\cdot) $ 是无零因子环,则称 $ (R,+,\cdot) $ 为整环。
    \label{def:integral_domain}
\end{definition}

\begin{definition}[除环 Division Ring]
    设 $ (R,+,\cdot) $ 为含幺环。如果对任意 $ a\in R $,$ a\neq 0 $,均存在 $ a^{-1}\in R $,使得 $ a\cdot a^{-1} = a^{-1}\cdot a = e $,则称 $ (R,+,\cdot) $ 为除环。
    也即  $(R\setminus\{0\},\cdot)$ 是群,$ (R,+,\cdot) $ 为含幺环。
    \label{def:division_ring}
\end{definition}

\begin{note}
    环在交换群的基础上,配备了第二代数运算 —— “乘法”,并且乘法对加法满足分配律。
    对环乘法附加更多的结构,可以得到交换环、含幺环、整环、除环等。
    比如,$(\mathbb{Z},+,\cdot)$ 是一个整环;
    $(\mathbb{Q},+,\cdot),\ (\mathbb{R},+,\cdot),\ (\mathbb{C},+,\cdot)$ 都是除环。
\end{note}
\vspace{1em}

\begin{definition}[子环 Subring]
    设 $ (R,+,\cdot) $ 为环,$ S\subseteq R $。如果 $ (S,+,\cdot) $ 也是环,则称 $ S $ 为 $ R $ 的子环。
    ${0}$ 和 $ R $ 都是 $ R $ 的子环,称为\textbf{平凡子环 Trivial subring},其他子环称为\textbf{非平凡子环 Non-trivial subring}。
    \label{def:subring}
\end{definition}

\begin{theorem}[子环判定定理]
    设 $ (R,+,\cdot) $ 为环,$ S\subseteq R $。则 $ S $ 是 $ R $ 的子环的充分必要条件是
    \begin{enumerate}
        \item $ 0\in S $;
        \item $ \forall a,b\in S $,$ a-b\in S $;
        \item $ \forall a,b\in S $,$ a\cdot b \in S $。
    \end{enumerate}
    \label{thm:subring_test}
\end{theorem}

\begin{note}
    子环判定定理相比子群判定定理,多了一个关于乘法封闭性的条件。
\end{note}
\vspace{1em}

\begin{definition}[环同态与环同构 Ring Homomorphism and Isomorphism]
    设 $R$ 和 $R^{\prime}$ 为环。函数 $ f:R\to R^{\prime} $ 是一个环同态,当且仅当,
    \begin{enumerate}
        \item $\forall a,b\in R$,$f(a+b)=f(a)+f(b)$;
        \item $\forall a,b\in R$,$f(a\cdot b)=f(a)\cdot f(b)$;
    \end{enumerate}
    如果函数 $ f $ 是单射,则称 $ f $ 为单同态;
    如果函数 $ f $ 是满射,则称 $ f $ 为满同态;
    如果映射 $ f $ 是双射,则称 $ f $ 为环同构,记为 $ R\cong R^{\prime} $。
    \label{def:ring_homomorphism_and_isomorphism}
\end{definition}

\begin{definition}[环同态的核 Kernel of Ring Homomorphism]
    设 $R$ 和 $R'$ 为环,$ f:R\to R^{\prime} $ 是一个环同态。定义集合:
    \[
        \ker f := \{a\in R : f(a)=0\in R^{\prime}\}
    \]
    称为环同态 $ f $ 的核。
    \label{def:kernel_of_ring_homomorphism}
\end{definition}

\begin{proposition}[环同态的性质]
    设 $R$ 和 $R'$ 为环,$ f:R\to R^{\prime} $ 是一个环同态。则
    \begin{enumerate}
        \item $ f(0)=0 $;
        \item $\forall a\in R$,$ f(-a)=-f(a) $;
        \item $\ker f \leq R$
    \end{enumerate}
\end{proposition}

\begin{note}
    环同态与群同态类似,都是保持代数结构的映射。环同态的核表示与另一个环 $0$ 元对应的环元素的集合,它是原环的一个子环。
\end{note}
\vspace{1em}

\subsection{多项式环}
\begin{definition}[多项式 Polynomial]
    设 $(R,+,\cdot)$ 为环,定义多项式:
    \[
        a_n x^n + a_{n-1} x^{n-1} + \cdots + a_1 x + a_0
    \]
    其中,$x$ 称为变量,$a_i \in R$ 称为系数,$n \in \mathbb{N}$ 称为次数。
    如果 $a_n \neq 0$ 称为变量 $x$ 的 $n$ 次多项式。
    如果 $a_0=a_1=\cdots=a_n=0$,称为 0 多项式。
    全体系数在环 $R$ 上的所有多项式的集合,记为 $R[x]$。
    \label{def:polynomial}
\end{definition}

\begin{definition}[等于关系]
    设 $(R,+,\cdot)$ 为环,$f,g\in R[x]$ 相等,记为 $f=g$,当且仅当 $f$ 和 $g$ 的对应系数相等。
\end{definition}

\begin{definition}[多项式环 Polynomial Ring]
    设 $(R,+,\cdot)$ 为环,$R[x]$ 为全体系数在 $R$ 上的所有多项式的集合。定义 $R[x]$ 上的加法和乘法如下:
    \begin{enumerate}
        \item 多项式加法:设 $f(x)=a_n x^n + a_{n-1} x^{n-1} + \cdots + a_1 x + a_0$,$g(x)=b_m x^m + b_{m-1} x^{m-1} + \cdots + b_1 x + b_0$,其中 $a_i,b_j \in R$,则
            \[
                f(x)+g(x) = \sum_{k=0}^{\max(n,m)} (a_k+b_k) x^k
            \]
        \item 多项式乘法:设 $f(x)=a_n x^n + a_{n-1} x^{n-1} + \cdots + a_1 x + a_0$,$g(x)=b_m x^m + b_{m-1} x^{m-1} + \cdots + b_1 x + b_0$,其中 $a_i,b_j \in R$,则
            \[
                f(x) \cdot g(x) = \sum_{k=0}^{n+m} c_k x^k
            \]
    \end{enumerate}
    其中,
    \begin{enumerate}
        \item $a_i = 0$ 当 $i>n$ 或 $i<0$;
        \item $b_j = 0$ 当 $j>m$ 或 $j<0$;
        \item $c_k = \sum_{i=0}^{k} a_i b_{k-i}$。
    \end{enumerate}
    则 $(R[x],+,\cdot)$ 满足环公理,称为 $R$ 上的多项式环。
    \label{def:polynomial_ring}
\end{definition}

\vspace{1em}
\subsection{理想与商环}

环 $(R,+,\cdot)$ 的 $(R,+)$ 是一个交换群,其所有子群都是正规子群。
而 $(R,\cdot)$ 只是一个半群,无法定义乘法的正规子群,我们可以类似地定义一种特殊的子环——理想子环,相关的结论与正规子群中的结论有相似性。

\vspace{1em}

\begin{definition}[左理想子环 Left Ideal]
    设 $(R,+,\cdot)$ 为环,$I \leq R$。$I$ 是 $R$ 的一个左理想子环,当且仅当,$\forall r\in R$,$\forall i\in I$,$r\cdot i \in I$。
\end{definition}

\begin{definition}[右理想子环 Right Ideal]
    设 $(R,+,\cdot)$ 为环,$I \leq R$。$I$ 是 $R$ 的一个右理想子环,当且仅当,$\forall r\in R$,$\forall i\in I$,$i\cdot r \in I$。
\end{definition}

\begin{definition}[理想子环 Ideal]
    设 $(R,+,\cdot)$ 为环,$I \leq R$。$I$ 是 $R$ 的一个理想子环,当且仅当,$I$ 同时是 $R$ 的一个左理想子环和右理想子环。简称为理想。
    \label{def:ideal}
\end{definition}

\begin{definition}[理想的等价定义]
    设 $(R,+,\cdot)$ 为环,$I \subset S$。$I$ 是 $R$ 的理想,当且仅当,
    \begin{enumerate}
        \item $0\in I$;
        \item $\forall a,b\in I$,$a-b\in I$;
        \item $\forall r\in R$,$\forall i\in I$,$r\cdot i \in I$ 且 $i\cdot r \in I$。
    \end{enumerate}
\end{definition}

\begin{proposition}
    设 $(R,+,\cdot)$ 为环,存在环同态 $f$,使得 $I=\ker f$,则 $I$ 是 $R$ 的一个理想。
\end{proposition}
\begin{proof}
    设 $a,b\in I$,则 $f(a)=0$,$f(b)=0$,所以
    \[
        f(a-b)=f(a)-f(b)=0-0=0
    \]
    因此 $a-b\in I$。又设 $r\in R$,$i\in I$,则 $f(i)=0$,所以
    \[
        f(r\cdot i)=f(r)\cdot f(i)=f(r)\cdot 0=0
    \]
    因此 $r\cdot i \in I$。同理可证 $i\cdot r \in I$。综上所述,$I$ 是 $R$ 的一个理想。
\end{proof}

\begin{definition}[主理想 Principal Ideal]
    设 $(R,+,\cdot)$ 为环,$a\in R$。定义集合:
    \[
        (a) := \{r\cdot a : r\in R\}
    \]
    称为由 $a$ 生成的主理想。也即一个理想中所有的元素都可以依据子环判定定理生成,那么它是主理想。
    \label{def:principal_ideal}
\end{definition}

\begin{example}
    设 $(\mathbb{Z},+,\cdot)$ 为整数环,$2\mathbb{Z}=\{\ldots,-4,-2,0,2,4,\ldots\}$ 是 $\mathbb{Z}$ 的一个理想。因为
    \begin{enumerate}
        \item $0\in 2\mathbb{Z}$;
        \item $\forall a,b\in 2\mathbb{Z}$,$a-b\in 2\mathbb{Z}$;
        \item $\forall r\in \mathbb{Z}$,$\forall i\in 2\mathbb{Z}$,$r\cdot i \in 2\mathbb{Z}$ 且 $i\cdot r \in 2\mathbb{Z}$。
    \end{enumerate}
    $(\mathbb{Z},+,\cdot)$ 的所有理想都是主理想,并且 $n\mathbb{Z}=(n)$。
\end{example}

\begin{note}
    理想是环的一个特殊子环,类似于群的正规子群。理想的定义中,除了满足子环的条件外,还要求对任意环元素与理想元素的乘积仍在理想中。
    如果 $R$ 是交换环,则 $I$ 是 $R$ 的左理想子环、右理想子环和理想是等价的。
\end{note}

\vspace{1em}

\begin{definition}[商环 Quotient Ring]
    设 $(R,+,\cdot)$ 为环,$I$ 是 $R$ 的理想。定义集合:
    \[
        R/I := \{a+I : a\in R\}
    \]
    其中,$a+I := \{a+i : i\in I\}$。在 $R/I$ 上定义加法和乘法如下:
    \begin{enumerate}
        \item 加法:$\forall a+I,b+I\in R/I$,$(a+I)+(b+I)=(a+b)+I$;
        \item 乘法:$\forall a+I,b+I\in R/I$,$(a+I)\cdot(b+I)=(a\cdot b)+I$。
    \end{enumerate}
    则 $(R/I,+,\cdot)$ 满足环公理,称为 $R$ 关于理想 $I$ 的商环。
    \label{def:quotient_ring}
\end{definition}

\begin{proposition}[商环的性质]
    设 $(R,+,\cdot)$ 为环,$I$ 是 $R$ 的理想,那么:
    \begin{enumerate}
        \item 如果 $R$ 是交换环,则 $R/I$ 也是交换环;
        \item 如果 $R$ 是含幺环,并且 $1\in R$ 是乘法单位元,则 $R/I$ 也是含幺环,且 $(1+I)$ 是 $R/I$ 的乘法单位元;
        \item 如果 $R$ 是除环,并且 $I\neq \{0\}$,则 $R/I$ 也是除环,那么 $a^{-1}+I$ 是 $R/I$ 中的乘法逆元。
    \end{enumerate}
\end{proposition}

\begin{example}
    设 $(\mathbb{Z},+,\cdot)$ 为整数环,$2\mathbb{Z} $ 是 $\mathbb{Z}$ 的一个理想。其中,
    \begin{align*}
        \vdots  \\
        -2 + 2\mathbb{Z} &= \{\ldots,-4,-2,0,2,4,\ldots\} = 0 + 2\mathbb{Z} \\
        -1 + 2\mathbb{Z} &= \{\ldots,-3,-1,1,3,5,\ldots\} = 1 + 2\mathbb{Z} \\
        0 + 2\mathbb{Z} &= \{\ldots,-4,-2,0,2,4,\ldots\} \\
        1 + 2\mathbb{Z} &= \{\ldots,-3,-1,1,3,5,\ldots\} \\
        2 + 2\mathbb{Z} &= \{\ldots,-2,0,2,4,6,\ldots\} = 0 + 2\mathbb{Z} \\
        3 + 2\mathbb{Z} &= \{\ldots,-1,1,3,5,7,\ldots\} = 1 + 2\mathbb{Z} \\
        \vdots 
    \end{align*}
    那么
    \[
        \mathbb{Z}/2\mathbb{Z} = \{0+2\mathbb{Z}, 1+2\mathbb{Z}\}
    \]
\end{example}

\begin{note}
    商环的定义与商群类似,都是将原代数结构划分为若干个等价类,然后在等价类上定义新的代数运算。
    商环的结构与原环和理想密切相关。
\end{note}
\vspace{1em}

\subsection{环同态基本定理}

\begin{theorem}[第一同构定理 First Isomorphism Theorem for Rings]
    设 $f:R\to R'$ 是一个环同态。那么 $\ker f$ 是 $R$ 的一个理想,并且 $R/\ker f \cong \mathrm{Im} f$。
    \label{thm:first_isomorphism_theorem_for_rings}
\end{theorem}
\begin{proof}
    
\end{proof}

\begin{note}
    环同态基本定理与群同态基本定理类似,都是通过同态映射将原代数结构与其像联系起来。
    环同态基本定理说明环的任一同态像,同构于原环关于同态核的商环。
\end{note}

\vspace{1em}

\begin{theorem}[第二同构定理 Second Isomorphism Theorem for Rings]
    设 $(R,+,\cdot)$ 为环,$I,J$ 是 $R$ 的理想。那么
    \[
        I/(I\cap J)  \cong  (I+J)/J
    \]
    \label{thm:second_isomorphism_theorem_for_rings}
\end{theorem}
\begin{proof}

\end{proof}

\begin{theorem}[第三同构定理 Third Isomorphism Theorem for Rings]
    设 $(R,+,\cdot)$ 为环,$I,J$ 是 $R$ 的理想,并且 $I\subseteq J$。那么
    \[
        (R/I)/(J/I) \cong R/J
    \]
    \label{thm:third_isomorphism_theorem_for_rings}
\end{theorem}
\begin{proof}

\end{proof}

\newpage