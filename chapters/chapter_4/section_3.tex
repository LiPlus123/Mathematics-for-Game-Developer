\section{域}

\subsection{域的定义}
\begin{definition}[域 Field]
    设 $F$ 是一个非空集合,$+$ 和 $\cdot$ 是定义在 $F$ 上的两种代数运算。
    $ (F,+,\cdot) $ 称为域,当且仅当,$(F,+,\cdot)$ 是一个交换环并且是一个除环。
    \label{def:field}
\end{definition}

\begin{definition}[子域 Subfield]
    设 $F$ 是一个域,$K \subseteq F$。$K$ 称为 $F$ 的一个子域,当且仅当,$K$ 在 $F$ 上关于加法和乘法是封闭的,并且 $(K,+,\cdot)$ 也是一个域。
    \label{def:subfield}
\end{definition}

\begin{theorem}[子域判定定理]
    设 $(F,+,\cdot)$ 是一个域,$K \subseteq F$。则 $K$ 是 $F$ 的子域的充分必要条件是:
    \begin{enumerate}
        \item $0,1\in K$;
        \item $\forall a,b\in K$,$a-b\in K$;
        \item $\forall a,b\in K\setminus\{0\}$,$a\cdot b^{-1} \in K$;
    \end{enumerate}
\end{theorem}

\begin{note}
    域是在环的基础上,添加了乘法交换律和乘法逆元存在性两个条件,并且域中没有零因子。
    常见的,$(\mathbb{Q},+,\cdot)$,$(\mathbb{R},+,\cdot)$ 和 $(\mathbb{C},+,\cdot)$ 都是域。
    $\mathbb{Q}$ 是 $\mathbb{R}$ 的子域,$\mathbb{R}$ 也是 $\mathbb{C}$ 的子域。
\end{note}

% \begin{definition}[有限域 Finite Field]
%     如果一个域 $F$ 含有有限个元素,则称 $F$ 为有限域。
%     \label{def:finite_field}
% \end{definition}

% \begin{definition}[有限域的特征 Characteristic of a Finite Field]
%     设 $F$ 是一个有限域,$1_F$ 是 $F$ 的乘法单位元。
%     如果存在一个最小的正整数 $p$,使得
%     \[
%         \underbrace{1_F + 1_F + \cdots + 1_F}_{p\text{ 个}} = 0_F,
%     \]
%     则称 $p$ 为 $F$ 的特征,记为 $\mathrm{char}(F)=p$。
%     如果不存在这样的正整数,则称 $F$ 的特征为零,记为 $\mathrm{char}(F)=0$。
%     \label{def:characteristic_of_finite_field}
% \end{definition}

% \begin{example}
%     设 $F=\mathbb{Z}/5\mathbb{Z}=\{0,1,2,3,4\}$,则 $F$ 是一个有限域,并且 $\mathrm{char}(F)=5$,因为
%     \[
%         1_F + 1_F + 1_F + 1_F + 1_F = 5 \equiv 0 \mod 5.
%     \]
%     设 $F=\mathbb{Q}$,则 $F$ 是一个无限域,并且 $\mathrm{char}(F)=0$,因为不存在正整数 $p$ 使得
%     \[
%         \underbrace{1_F + 1_F + \cdots + 1_F}_{p\text{ 个}} = 0_F.
%     \]
% \end{example}

\vspace{1em}

\subsection{域扩张}

\begin{definition}[域扩张 Field Extension]
    设 $F$ 是一个域,如果存在一个域 $K$,使得 $F$ 是 $K$ 的一个子域,则称 $K$ 是 $F$ 的一个域扩张,记为 $K/F$。
    \label{def:field_extension}
\end{definition}
\begin{note}
    域扩张是域论中的一个重要概念,如何从一个小域构造出一个更大的域,或者研究两个域之间的关系,都是通过域扩张来实现的。
\end{note}

\begin{definition}[单扩域 Simple Extension]
    设 $F$ 是一个域,$K/F$ 是 $F$ 的一个域扩张。
    如果存在 $a\in K$,使得 $K=F(a)$,则称 $K/F$ 是一个单扩域。
    这里 $F(a)$ 表示包含 $F$ 和 $a$ 的最小子域。
    \label{def:simple_extension}
\end{definition}
\begin{note}
    单扩域在 $K/F$ 所有子域中包含基域 $F$ 和另一个元素 $a\in K$ 的最小子域。
    例如,$\mathbb{C}=\mathbb{R}(i)= \{a+bi:a,b\in \mathbb{R}\}$ 是 $\mathbb{R}$ 的一个单扩域,因为 $\mathbb{C}$ 是包含 $\mathbb{R}$ 和 $i$ 的最小子域。
    另一个例子是 $\mathbb{Q}(\sqrt{2}) = \{a+b\sqrt{2}:a,b\in \mathbb{Q}\}$ 是 $\mathbb{Q}$ 的一个单扩域,因为它是包含 $\mathbb{Q}$ 和 $\sqrt{2}$ 的最小子域。
\end{note}

\begin{definition}[分裂域 Splitting Field]
    设 $F$ 是一个域,$f(x)\in F[x]$ 是一个正次数多项式。
    如果存在一个域扩张 $K/F$,使得 $f(x)$ 可以写成一系列在 $K[x]$ 中的一次多项式的乘积,也即
    \[
        f(x) = a(x - r_1)(x - r_2) \cdots (x - r_n),\ a,r_1,r_2,\ldots,r_n \in K,
    \]
    并且 $K$ 是包含所有 $f(x)$ 所有根的最小域,则称 $K$ 是 $f(x)$ 在 $F$ 上的一个分裂域。
    \label{def:splitting_field}
\end{definition}
\begin{note}
    在某个域上某个多项式方程可能没有解,因此希望通过构造一个更大的域,使得该多项式方程在这个更大的域中有解。
    分裂域就是这样一个更大的域,它是相对特定多项式而言的,它包含了多项式的所有根,并且是包含这些根的最小域。
\end{note}
\vspace{1em}

\begin{definition}[有限扩张与无限扩张 Finite and Infinite Extensions]
    设 $K/F$ 是一个域扩张,$K$ 作为 $F$ 上的一个向量空间。
    如果 $K$ 作为 $F$ 上的向量空间是有限维的,则称 $K/F$ 是一个有限扩张,否则称 $K/F$ 是一个无限扩张。
    $K$ 的维数称为扩张的次数,记为 $[K:F]$。
    \label{def:finite_infinite_extension}
\end{definition}

\begin{note}
    $\mathbb{C}$ 是 $\mathbb{R}$ 的一个有限扩展,并且 $\{1,i\}$ 是 $\mathbb{C}$ 作为 $\mathbb{R}$ 上的一个向量空间的一个基,因此 $[\mathbb{C}:\mathbb{R}]=2$。
    $\mathbb{R}$ 是 $\mathbb{Q}$ 的一个无限扩展,因为 $\mathbb{R}$ 作为 $\mathbb{Q}$ 上的一个向量空间是无限维的。
\end{note}
\vspace{1em}

\begin{definition}[代数数与超越数 Algebraic and Transcendental Numbers]
    设 $F$ 是一个域,$F[x]$ 是 $F$ 上的多项式环,$K$ 是 $F$ 的一个域扩张。
    如果存在一个非零多项式 $f(x)\in F[x]$,使得 $f(a)=0,\ a\in K$,则称 $a$ 是 $F$ 上的一个代数数,否则称 $a$ 是 $F$ 上的一个超越数。
    \label{def:algebraic_transcendental}
\end{definition}

\begin{definition}[代数扩张和超越扩张 Algebraic and Transcendental Extensions]
    设 $F$ 是一个域,$K/F$ 是 $F$ 的一个域扩张。
    如果 $K$ 中的每个元素都是 $F$ 上的代数数,则称 $K/F$ 是一个代数扩张,否则称 $K/F$ 是一个超越扩张。
    \label{def:algebraic_transcendental_extension}
\end{definition}

\begin{note}
    代数数是指那些可以作为某个非零多项式的根的数,而超越数则是那些不能作为任何非零多项式的根的数。
    例如,$\sqrt{2}$ 是 $\mathbb{Q}$ 上的一个代数数,因为它是多项式 $x^2 - 2$ 的一个根。
    另一方面,$\pi$ 是 $\mathbb{Q}$ 上的一个超越数,因为不存在任何非零多项式 $f(x)\in \mathbb{Q}[x]$ 使得 $f(\pi)=0$。
    代数扩张是指扩张域中的每个元素都是基域上的代数数,而超越扩张则包含至少一个超越数。
    根据代数基本定理 \ref{thm:FundamentalTheoremOfAlgebra},$\mathbb{C}$ 中的每个元素都是 $\mathbb{R}$ 上的代数数,
    也即任意一个实数域上的多项式方程在复数域中都有解,因此 $\mathbb{C}$ 是 $\mathbb{R}$ 的一个代数扩张。
    $\mathbb{R}$ 是 $\mathbb{Q}$ 的一个超越扩张,存在 $\pi\in \mathbb{R}$ 是 $\mathbb{Q}$ 上的一个超越数。
\end{note}
\vspace{1em}

\begin{definition}[可约多项式与不可约多项式 Reducible and Irreducible Polynomials]
    设 $F$ 是一个域,$F[x]$ 是 $F$ 上的多项式环。
    如果存在两个次数均小于 $f(x)$ 的非零多项式 $g(x),h(x)\in F[x]$,使得 $f(x)=g(x)h(x)$,则称 $f(x)$ 在 $F$ 上是可约的,否则称 $f(x)$ 在 $F$ 上是不可约的。
    \label{def:reducible_irreducible_polynomial}
\end{definition}
\begin{note}
    则多项式 $f(x)=x^2-2$ 在 $\mathbb{Q}$ 上是不可约的,因为它不能分解为两个次数均小于 2 的有理系数多项式的乘积。
    另一方面,多项式 $g(x)=x^2-1$ 在 $\mathbb{Q}$ 上是可约的,因为它可以分解为 $(x-1)(x+1)$。
\end{note}

\begin{definition}[可分多项式与不可分多项式 Separable and Inseparable Polynomials]
    设 $F$ 是一个域,$F[x]$ 是 $F$ 上的多项式环。
    如果正次数多项式 $f(x)\in F[x]$ 的每个根在其分裂域中都是单根,则称 $f(x)$ 在 $F$ 上是可分的;
    反之,存在至少一个重根,则称 $f(x)$ 在 $F$ 上是不可分的。
    \label{def:separable_inseparable_polynomial}
\end{definition}
\begin{note}
    多项式 $f(x)=x^2-2$ 在 $\mathbb{Q}$ 上是可分的,因为它在其分裂域 $\mathbb{Q}(\sqrt{2})$ 中有两个不同的根 $\sqrt{2}$ 和 $-\sqrt{2}$。
    另一方面,多项式 $g(x)=x^2-2x+1=(x-1)^2$ 在 $\mathbb{Q}$ 上是不可分的,因为它在其分裂域 $\mathbb{Q}$ 中只有一个根 $1$,且该根是重根。
\end{note}
\vspace{1em}

\begin{definition}[可分扩张 Separable Extension]
    设 $F$ 是一个域,$K/F$ 是 $F$ 的一个代数扩张。$K/F$ 称为可分扩张,当且仅当,$K$ 中的每个元素都是 $F[x]$ 中某个不可约但可分多项式的根。
\end{definition}
\begin{note}
    $\mathbb{Q}(\sqrt{2})/\mathbb{Q}$ 是一个可分扩张,其中 $\pm\sqrt{2}$ 是 $\mathbb{Q}$ 上不可约但可分多项式 $x^2-2$ 的两个单根。
\end{note}
\vspace{1em}

\begin{definition}[正规扩张 Normal Extension]
    设 $F$ 是一个域,$K/F$ 是 $F$ 的一个代数扩张。$K/F$ 称为正规扩张,当且仅当,所有 $F[x]$ 中不可约多项式在 $K$ 中能完全可以分解为一次多项式的乘积。
    也即,$K$ 是包含所有 $F[x]$ 中不可约多项式所有根。
    \label{def:normal_extension}
\end{definition}

\begin{note}
    $\mathbb{C}/\mathbb{R}$ 是一个正规扩张,因为 $\mathbb{C}$ 是包含所有 $\mathbb{R}[x]$ 中不可约多项式所有根的代数扩张域。
    例如,$\mathbb{R}[x]$ 中不可约多项式 $x^2+1$ 在 $\mathbb{C}$ 中可以分解为 $(x-i)(x+i)$。
    另一方面,$\mathbb{Q}(\sqrt[3]{2})/\mathbb{Q}$ 不是一个正规扩张,因为 $\mathbb{Q}(\sqrt[3]{2})$ 不包含不可约多项式 $x^3-2$ 的所有根。
    $x^3-2$ 在 $\mathbb{C}$ 中有三个根,分别是 $\sqrt[3]{2}$、$\sqrt[3]{2}\omega$ 和 $\sqrt[3]{2}\omega^2$,其中 $\omega = e^{2\pi i/3}$ 是一个非实的复数。
\end{note}
\vspace{1em}

\subsection{伽罗瓦理论}

\begin{definition}[伽罗瓦扩张 Galois Extension]
    设 $F$ 是一个域,$K/F$ 是 $F$ 的一个代数扩张。
    如果 $K/F$ 既是正规扩张又是可分扩张,则称 $K/F$ 是一个伽罗瓦扩张。
    \label{def:galois_extension}
\end{definition}

\begin{definition}[伽罗瓦群 Galois Group]
    设 $F$ 是一个域,$K/F$ 是 $F$ 的一个域扩张。
    定义 $K$ 上的一个 $F$-自同构:$\sigma:K\to K$,使得 $\sigma(a)=a,\ \forall a\in F$。
    所有 $K$ 上的 $F$-自同构,记为 $\mathrm{Gal}(K/F)$。
    $\mathrm{Gal}(K/F)$ 上的二元运算时函数的复合,那么 $(\mathrm{Gal}(K/F),\circ)$ 满足群公理,称为 $K/F$ 的伽罗瓦群。
    \label{def:galois_group}
\end{definition}

\begin{note}
    伽罗瓦群是研究域扩张对称性的重要工具。它捕捉了域扩张中元素之间的对称关系,并且与多项式方程的可解性密切相关。
    伽罗瓦群的结构可以揭示多项式方程根的排列方式,从而帮助理解多项式方程是否可以通过根式来解。
\end{note}

\begin{definition}[伽罗瓦对应 Galois Correspondence]
    设 $F$ 是一个域,$K/F$ 是 $F$ 的一个伽罗瓦扩张。
    设 $H \leq \mathrm{Gal}(K/F)$ 是 $\mathrm{Gal}(K/F)$ 的一个子群,定义
    \[
        K^H = \{a\in K:\sigma(a)=a,\ \forall \sigma\in H\}.
    \]
    则 $K^H$ 是 $K/F$ 的一个子扩张域。
    反之,设 $E$ 是 $K/F$ 的一个子扩张域,定义
    \[
        \mathrm{Gal}(K/E) = \{\sigma\in \mathrm{Gal}(K/F):\sigma(a)=a,\ \forall a\in E\}.
    \]
    则 $\mathrm{Gal}(K/E)$ 是 $\mathrm{Gal}(K/F)$ 的一个子群。
    这两个映射 $H\mapsto K^H$ 和 $E\mapsto \mathrm{Gal}(K/E)$ 互为逆映射,并且它们之间存在包含关系的反转。
    也即,如果 $H_1 \leq H_2 \leq \mathrm{Gal}(K/F)$,则 $K^{H_2} \leq K^{H_1}$;
    如果 $E_1 \leq E_2$ 是 $K/F$ 的两个子扩张域,则 $\mathrm{Gal}(K/E_2) \leq \mathrm{Gal}(K/E_1)$。
    这就是伽罗瓦对应。
    \label{def:galois_correspondence}
\end{definition}

\begin{note}

\end{note}

\begin{theorem}[伽罗瓦基本定理 Fundamental Theorem of Galois Theory]
    设 $F$ 是一个域,$K/F$ 是 $F$ 的一个伽罗瓦扩张。
    则 $K/F$ 的子扩张域与 $\mathrm{Gal}(K/F)$ 的子群之间存在一一对应关系,这个对应关系是包含关系的反转。
    \label{thm:FundamentalTheoremOfGaloisTheory}
\end{theorem}

\begin{note}
    
\end{note}
\vspace{1em}

\begin{definition}[可解群 Solvable Group]
    设 $G$ 是一个群,如果存在一列子群
    \[
        \{e\} = G_0 \triangleleft G_1 \triangleleft G_2 \triangleleft \cdots \triangleleft G_n = G,
    \]
    使得每个商群 $G_{i+1}/G_i$ 都是阿贝尔群,则称 $G$ 是一个可解群。
    \label{def:solvable_group}
\end{definition}

\begin{theorem}[多项式方程根式可解定理 Polynomial Equation Solvability by Radicals Theorem]
    设 $F$ 是一个域,$f(x)\in F[x]$ 是一个正次数多项式,$K$ 是 $f(x)$ 在 $F$ 上的一个分裂域。
    则 $f(x)$ 的根可以表示为 $F$ 中元素和根式的有限次运算的结果,当且仅当,伽罗瓦群 $\mathrm{Gal}(K/F)$ 是一个可解群。
    \label{thm:PolynomialEquationSolvabilityByRadicals}
\end{theorem}

\begin{theorem}[阿贝尔-伽罗瓦定理 Abel Galois Theorem]
    一般五次及以上的多项式方程没有根式解。
    \label{thm:Abel_Galois_Theorem}
\end{theorem}

\newpage