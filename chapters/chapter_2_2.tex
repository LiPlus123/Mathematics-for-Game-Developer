\section{函数}
\begin{definition}[函数 Function]
    设 $ X, Y $ 为集合,$ f $ 为 $ X, Y $ 上的二元关系,$f$ 称为 $ X $ 到 $ Y $ 的一个\textbf{函数(Function)},当且仅当,$f$ 满足:
    \begin{enumerate}
        \item 唯一性:$ \forall x \in \mathrm{dom} f $,如果 $ (x,y_1)\in f  $ 且 $ (x,y_2)\in f $,则 $ y_1 = y_2 $
    \end{enumerate}
    记作 $ f: X \to Y $,记 $xfy$ 为 $y = f(x) $。定义域 $\mathrm{dom}f$ 也称为函数的原像(Preimage),值域 $\mathrm{ran}f$ 也称为函数的像(Image),也常记为 $\mathrm{im}f$ 或 $f(X)$。
\end{definition}

\begin{note}
    函数是一种特殊的二元关系,其唯一性说明,若对任意 $ x\in X $,在 $ f $ 中存在唯一的 $ y\in Y $,使得 $ (x,y)\in f $。不存在一个 $ x\in X $,对应多个 $ y\in Y $ 的情况。
    函数的复合的复合也即二元关系的复合。函数的逆不一定是函数,因为可能存在 $ y\in Y $,对应多个 $ x\in X $ 的情况,不满足唯一性。
\end{note}
\vspace{1em}

\begin{definition}
    设 $ f: X \to Y $ 为函数,若
    \begin{enumerate}
        \item \textbf{满射(Surjection)}:如果 $ \mathrm{ran}f = Y $,则称 $ f $ 为满射。
        \item \textbf{单射(Injection)}:如果 $ \forall y \in \mathrm{ran} f $,$ (x_1,y)\in f $ 且 $ (x_2,y)\in f $,$ x_1 = x_2 $,则称 $ f $ 为单射。
        \item \textbf{双射(Bijection)}:如果 $ f $ 同时为单射与满射,则称 $ f $ 为双射。
    \end{enumerate}
\end{definition}
\vspace{0.5em}

\subsection{集合的势}

对于一个有穷集合,我们可以用一个自然数表示集合的大小,并根据自然数的大小比较集合的大小。
这是对集合“大小”非常朴素的认识,并没有严格定义,而且推广到无穷集时,这一方法就失效了。
为了使无穷集之间也能比“大小”,通过函数定义集合的势(Cardinality)。
\vspace{0.5em}

\begin{definition}[集合的势 Cardinality]
    设 $ X, Y $ 为集合,
    \begin{enumerate}
        \item $ X $ 与 $ Y $ \textbf{等势},当且仅当,存在 $ X $ 到 $ Y $ 的双射,记为 $ |X| = |Y| $;
        \item $ Y $ \textbf{优势于} $ X $ ,当且仅当,存在 $ X $ 到 $ Y $ 的单射,记为 $ |X| \leq |Y| $;
        \item $ Y $ \textbf{严格优势于} $ X $,当且仅当,存在 $ X $ 到 $ Y $ 的单射,但不存在 $ X $ 到 $ Y $ 的双射,记为 $ |Y| < |X| $。
    \end{enumerate}
\end{definition}
\vspace{0.5em}

\begin{proposition}
    等势具有自反性、对称性和传递性:
    \begin{enumerate}
        \item 自反性:对于任意集合 $ X $,有 $ |X| = |X| $。
        \item 对称性:如果 $ |X| = |Y| $,则 $ |Y| = |X| $。
        \item 传递性:如果 $ |X| = |Y| $ 且 $ |Y| = |Z| $,则 $ |X| = |Z| $。
    \end{enumerate}
\end{proposition}
\vspace{0.5em}

\begin{proposition}
    优势关系具有自反性、反对称性和传递性:
    \begin{enumerate}
        \item 自反性:对于任意集合 $ X $,有 $ |X| \leq |X| $。
        \item 反对称性:如果 $ |X| \leq |Y| $ 且 $ |Y| \leq |X| $,则 $ |X| = |Y| $。
        \item 传递性:如果 $ |X| \leq |Y| $ 且 $ |Y| \leq |Z| $,则 $ |X| \leq |Z| $。
    \end{enumerate}
\end{proposition}
\vspace{0.5em}

\begin{definition}[有限集与无限集]
    设 $ X $ 为集合,
    \begin{enumerate}
        \item 如果 $ X $ 与某个自然数 $ n $ 等势,则称 $ X $ 为\textbf{有限集(Finite Set)},记为 $ |X| = n $;
        \item 如果 $ X $ 为空集,则称 $ X $ 为\textbf{空集(Empty Set)},记为 $ |X| = 0 $;
        \item 如果 $ X $ 既不是空集也不是有限集,则称 $ X $ 为\textbf{无限集(Infinite Set)}。
        \item 如果 $ X $ 与自然数集 $ \mathbb{N} $ 等势,则称 $ X $ 为\textbf{可数无限集(Countably Infinite Set)}
    \end{enumerate}
\end{definition}
\vspace{0.5em}

\begin{example}
    对于常见的数集,$\mathbb{N} \subset \mathbb{Z} \subset \mathbb{Q} $ 都是可数无限集,$ |\mathbb{N}| = |\mathbb{Z}| = |\mathbb{Q}| $,因为可以找到双射函数
    \begin{enumerate}
        \item 设 $ f:\mathbb{N}\to \mathbb{Z} $,将 $ \mathbb{Z} $ 中的整数按如下方式排列:
        \[
            0, 1, -1, 2, -2, 3, -3, \cdots  
        \]
        这样就可以定义双射 $ f(n) $ 为上述排列中第 $ n $ 个整数
        \item 设 $ g:\mathbb{N}\to \mathbb{Q} $,将 $ \mathbb{Q} $ 中的有理数按如下方式排列:
        \[
            0, 1, -1, \frac{1}{2}, -\frac{1}{2}, 2, -2, \frac{1}{3}, -\frac{1}{3}, \frac{2}{3}, -\frac{2}{3}, 3, -3, \cdots
        \]
        这样就可以定义双射 $ g(n) $ 为上述排列中第 $ n $ 个有理数
    \end{enumerate}
    实数集 $ \mathbb{R} $ 是不可数无限集,$ |\mathbb{R}| > |\mathbb{N}| $,可以通过 Cantor 对角线论证法证明。
\end{example}
\vspace{0.5em}


\newpage