\section{四元数}

从自然数到复数的扩充非常成功,整数的引入定义了减法,有理数的引入定义了除法,实数的引入定义了极限运算,复数的引入解决了多项式方程的根问题。
复数的乘法可以表示二元平面内的旋转,为了表示更高维的旋转,需要引入更高维的数系,而四元数常用来表示三维空间内的旋转。
\begin{note}
    之所以用四元数表示三维旋转,而不是用三元数,是因为三元数无法定义封闭的乘法运算,所以不能成为一个有用的数系。
    即便如此,四元数也不得不放弃乘法交换律。八元数甚至无法定义乘法结合律。
\end{note}

\vspace{1em}

\subsection{四元数的定义}
\begin{definition}[四元数 Quaternion]
    有序数组 $(a, b, c, d)\in\mathbb{R}^4$ 记为 $q = a + bi + cj + dk$,其中,
    \begin{enumerate}
        \item 约定:$i^2 = j^2 = k^2 = ijk = -1$;
        \item 等于关系:$a + bi + cj + dk = a' + b'i + c'j + d'k \iff a = a', b = b', c = c', d = d'$;
        \item 等价类:$[a + bi + cj + dk]_{=}=\{(e,f,g,h)\in\mathbb{R}^4 : a + bi + cj + dk = e + fi + gj + hk\}$。
    \end{enumerate}
    称 $\mathbb{R}^4$ 关于 $=$ 关系的商集 $\mathbb{R}^4/_{=}= \{[a + bi + cj + dk]_{=} : a + bi + cj + dk\in\mathbb{R}^4\}$ 为四元数集,记为 $\mathbb{H}$。
    四元数集中的元素称为四元数。
\end{definition}

\begin{note}
    四元数是一个等价类,但集合中只有一个有序数组作为代表元,所以可以直接用 $a + bi + cj + dk$ 来表示四元数。
    在四元数中,实数 $a$ 称为四元数的实部,$bi + cj + dk$ 称为四元数的虚部。四元数可以用一个标量和一个三维向量来表示,
    即 $q = (a, \mathbf{v})$,其中 $\mathbf{v} = (b, c, d)$。观察四元数定义中的虚部的约定,可得下表:
    \[
        \begin{array}{|c|cccc|}
            \hline
           \times & 1 & i & j & k \\ 
           \hline
            1 & 1 & i & j & k \\ 
            i & i & -1 & k & -j \\ 
            j & j & -k & -1 & i \\ 
            k & k & j & -i & -1\\
            \hline
        \end{array}
    \]
\end{note}
\vspace{1em}

\begin{definition}[四元数的加法]
    设 $q_1 = a_1 + b_1 i + c_1 j + d_1 k,\ q_2 = a_2 + b_2 i + c_2 j + d_2 k\in\mathbb{H}$,定义四元数的加法:
    \[
        q_1 + q_2 = (a_1 + a_2) + (b_1 + b_2)i + (c_1 + c_2)j + (d_1 + d_2)k
    \]
\end{definition}
\begin{definition}[四元数加法逆元与减法]
    设 $q = a + bi + cj + dk\in\mathbb{H}$,四元数的加法逆元记为 $-q$,满足:
    \[
        q + (-q) = 0 + 0i + 0j + 0k
    \]
    那么
    \[
        -q := -a - bi - cj - dk
    \]
    四元数的减法定义为:
    \[
        q_1 - q_2 = q_1 + (-q_2)
    \]
\end{definition}
\vspace{1em}

\begin{definition}[四元数的乘法]
    设 $q_1 = a_1 + b_1 i + c_1 j + d_1 k,\ q_2 = a_2 + b_2 i + c_2 j + d_2 k\in\mathbb{H}$,定义四元数的乘法:
    \begin{align*}
        q_1 q_2 &= (a_1 + b_1 i + c_1 j + d_1 k)(a_2 + b_2 i + c_2 j + d_2 k) \\
        % &= a_1 a_2 + a_1 b_2 i + a_1 c_2 j + a_1 d_2 k \\
        % &\quad + b_1 a_2 i + b_1 b_2 i^2 + b_1 c_2 ij + b_1 d_2 ik \\
        % &\quad + c_1 a_2 j + c_1 b_2 ji + c_1 c_2 j^2 + c_1 d_2 jk \\
        % &\quad + d_1 a_2 k + d_1 b_2 ki + d_1 c_2 kj + d_1 d_2 k^2 \\
        &= (a_1 a_2 - b_1 b_2 - c_1 c_2 - d_1 d_2) \\
        &\quad + (a_1 b_2 + b_1 a_2 + c_1 d_2 - d_1 c_2)i \\
        &\quad + (a_1 c_2 - b_1 d_2 + c_1 a_2 + d_1 b_2)j \\
        &\quad + (a_1 d_2 + b_1 c_2 - c_1 b_2 + d_1 a_2)k
    \end{align*}
    四元数的乘法还可以写成矩阵的形式:
    \[
        q_1 q_2 = 
        \begin{bmatrix}
            a_1 & -b_1 & -c_1 & -d_1 \\
            b_1 & a_1 & -d_1 & c_1 \\
            c_1 & d_1 & a_1 & -b_1 \\
            d_1 & -c_1 & b_1 & a_1
        \end{bmatrix}
        \begin{bmatrix}
            a_2 \\ b_2 \\ c_2 \\ d_2
        \end{bmatrix}
    \]
    如果用 $q_1 = (a_1, \mathbf{v}_1),\ q_2 = (a_2, \mathbf{v}_2)$ 来表示四元数,则有:
    \[
        q_1 q_2 = (a_1 a_2 - \mathbf{v}_1 \cdot \mathbf{v}_2, a_1 \mathbf{v}_2 + a_2 \mathbf{v}_1 + \mathbf{v}_1 \times \mathbf{v}_2)
    \]
    其中, $\mathbf{v}_1 \cdot \mathbf{v}_2$ 是三维向量的点积,$\mathbf{v}_1 \times \mathbf{v}_2$ 是三维向量的叉积。
    \[
        \mathbf{v}_1 \cdot \mathbf{v}_2 = b_1 b_2 + c_1 c_2 + d_1 d_2
    \]
    \[
        \mathbf{v}_1 \times \mathbf{v}_2 =
        \begin{vmatrix}
            i & j & k \\
            b_1 & c_1 & d_1 \\
            b_2 & c_2 & d_2
        \end{vmatrix}
        = (c_1 d_2 - d_1 c_2)i + (d_1 b_2 - b_1 d_2)j + (b_1 c_2 - c_1 b_2)k
    \]
\end{definition}
\vspace{1em}
\begin{note}
    四元数的加法和乘法都是良定义的,即封闭,且与代表元的选择无关。
    四元数加法逆元是良定义的,任意一个四元数都有唯一的加法逆元。
    四元数的加法满足交换律和结合律,但乘法不满足交换律,只满足结合律。
    在定义四元数除法之前,需要先定义四元数的模和共轭。
\end{note}
\vspace{1em}

\begin{definition}[四元数的共轭 Conjugate]
    设 $q = a + bi + cj + dk\in\mathbb{H}$,四元数的共轭记为 $\bar{q}$ 或 $q^*$,那么:
    \[
        \bar{q} = a - bi - cj - dk
    \]
\end{definition}

\begin{definition}[四元数的模 Modulus]
    设 $q = a + bi + cj + dk\in\mathbb{H}$,四元数的模记为 $|q|$,那么:
    \[
        |q| = \sqrt{a^2 + b^2 + c^2 + d^2}
    \]
\end{definition}

\begin{proposition}[四元数的共轭与模的关系]
    设 $q = a + bi + cj + dk\in\mathbb{H}$,那么:
    \[
        q \bar{q} = \bar{q} q =  |q|^2
    \]
\end{proposition}

\begin{definition}[四元数的度量]
    设 $q_1, q_2\in\mathbb{H}$,定义四元数的度量:
    \[
        d(q_1, q_2) = |q_1 - q_2|
    \]
\end{definition}

\vspace{1em}

\begin{definition}[四元数的逆元与除法]
    设 $q = a + bi + cj + dk\in\mathbb{H},\ q \neq 0$,四元数的逆元记为 $q^{-1}$,满足:
    \[
        q q^{-1} = q^{-1} q = 1
    \]
    由此可得:
    \[
        q^{-1} = \frac{\bar{q}}{|q|^2} = \frac{\bar{q}}{q\bar{q}} = \frac{a - bi - cj - dk}{a^2 + b^2 + c^2 + d^2}
    \]
    设 $q_1, q_2\in\mathbb{H},\ q_2 \neq 0$,四元数的除法定义为:
    \[
        q_1 / q_2 = q_1 q_2^{-1} = \frac{q_1 \bar{q_2}}{q_2 \bar{q_2}}
    \]
\end{definition}
\vspace{1em}

\subsection{四元数与旋转}
\begin{definition}[单位四元数 Unit Quaternion]
    设 $q = a + bi + cj + dk\in\mathbb{H}$,如果 $|q| = 1$,则称 $q$ 为单位四元数。
\end{definition}

% \begin{definition}[四元数的旋转]
%     设 $q = \cos\frac{\theta}{2} + (u_x i + u_y j + u_z k)\sin\frac{\theta}{2}$ 是一个单位四元数,其中 $\mathbf{u} = (u_x, u_y, u_z)$ 是一个单位向量,$\theta$ 是旋转角度。
%     设 $\mathbf{p} = (p_x, p_y, p_z)$ 是三维空间中的一个点,记为纯四元数 $p = 0 + p_x i + p_y j + p_z k$。
%     那么点 $\mathbf{p}$ 绕单位向量 $\mathbf{u}$ 旋转 $\theta$ 角度后得到的点 $\mathbf{p'}$ 可以表示为纯四元数 $p' = 0 + p'_x i + p'_y j + p'_z k$,其中:
%     \[
%         p' = q p q^{-1}
%     \]
% \end{definition}


\subsection{对偶四元数}
\begin{definition}[对偶四元数 Dual Quaternion]
    有序数组 $(a, b, c, d, e, f, g, h)\in\mathbb{R}^8$ 记为 $Q = a + bi + cj + dk + \epsilon(e + fi + gj + hk)$,其中,
    \begin{enumerate}
        \item 约定:$i^2 = j^2 = k^2 = ijk = -1,\ \epsilon^2 = 0,\ \epsilon \neq 0$;
        \item 等于关系:$a + bi + cj + dk + \epsilon(e + fi + gj + hk) = a' + b'i + c'j + d'k + \epsilon(e' + f'i + g'j + h'k)$
        \[
            \iff a = a', b = b', c = c', d = d', e = e', f = f', g = g', h = h'
        \]
        \item 等价类:$[a + bi + cj + dk + \epsilon(e + fi + gj + hk)]_{=}=\{(l,m,n,o,p,q,r,s)\in\mathbb{R}^8 : a + bi + cj + dk + \epsilon(e + fi + gj + hk) = l + mi + nj + ok + \epsilon(p + qi + rj + sk)\}$。
    \end{enumerate}
    称 $\mathbb{R}^8$ 关于 $=$ 关系的商集 $\mathbb{R}^8/_{=}=$ 为对偶四元数集,记为 $\mathbb{DQ}$。
    对偶四元数集中的元素称为对偶四元数。
    
\end{definition}

\begin{note}
    对偶四元数可以表示三维空间中任何刚体变换,包括旋转和平移。根据四元数虚部的约定,可得下表:
    \[
        \begin{array}{|c|cccccccc|}
            \hline
            \times & 1 & i & j & k & \epsilon & \epsilon i & \epsilon j & \epsilon k \\
            \hline
            1 & 1 & i & j & k & \epsilon & \epsilon i & \epsilon j & \epsilon k \\
            i & i & -1 & k & -j & \epsilon i & -\epsilon & \epsilon k & -\epsilon j \\
            j & j & -k & -1 & i & \epsilon j & -\epsilon k & -\epsilon & \epsilon i \\
            k & k & j & -i & -1 & \epsilon k & \epsilon j & -\epsilon i & -\epsilon \\
            \epsilon  & \epsilon & \epsilon i & \epsilon j & \epsilon k & 0 & 0 & 0 & 0 \\
            \epsilon i & \epsilon i & -\epsilon & \epsilon k & -\epsilon j & 0 & 0 & 0 & 0 \\
            \epsilon j & \epsilon j & -\epsilon k & -\epsilon & \epsilon i & 0 & 0 & 0 & 0 \\
            \epsilon k & \epsilon k & \epsilon j & -\epsilon i & -\epsilon & 0 & 0 & 0 & 0 \\
            \hline
        \end{array}
    \]

\end{note}