\section{整数}

\begin{definition}[整数 Integer]
    有序对 $(a,b)\in \mathbb{N}^2$ 记为 $a-b$,其中,
    \begin{enumerate}
        \item 等于关系 :$a-b=c-d \iff a+d=b+c$
        \item 等价类:任意 $a-b\in\mathbb{N}^2$ 的等价类为
        \[
            [a-b]_{=} = \{(c,d)\in\mathbb{N}^2 : a+d=b+c\} 
        \]
    \end{enumerate}
    称 $\mathbb{N}^2$ 关于 $=$ 关系的商集 $\mathbb{N}^2/_{=}= \{[a-b]_{=} : a-b \in \mathbb{N}^2\}= $ 为整数集,记为 $\mathbb{Z}$。
    整数集中的元素称为整数。
\end{definition}
\begin{note}
    在 $a-b$ 中,$-$ 不是减号,而是一个符号,此时还没有定义减法,仅表示这是一个有序对 $(a,b)$。
    整数也不是有序对,而是有序对关于等于关系的等价类,比如, $3-5$ 和 $1-3$ 是不同的有序对,但它们属于同一个等价类 $[3-5]_{=}=[1-3]_{=}$,表示同一个整数。
    全体自然数可以嵌入到整数集中,即 $\forall n\in\mathbb{N}$,有 $n = [n-0]_{=}\in\mathbb{Z}$。
    例如,$0 = [0-0]_{=}$,$1 = [1-0]_{=}$,$2 = [2-0]_{=}$,$3 = [3-0]_{=}$,$\cdots$。$\complement_{\mathbb{Z}}\mathbb{N}$ 中的元素称为负整数。
\end{note}
\vspace{1em}

\begin{definition}[整数的加法 Addition of Integers]
    设 $[a-b]_{=},[c-d]_{=}\in\mathbb{Z}$,定义加法为
    \[
        [a-b]_{=} + [c-d]_{=} = [(a+c)-(b+d)]_{=}
    \]
\end{definition}

\begin{definition}[整数的乘法 Multiplication of Integers]
    设 $[a-b]_{=},[c-d]_{=}\in\mathbb{Z}$,定义乘法为
    \[
        [a-b]_{=} \cdot [c-d]_{=} = [(ac+bd)-(ad+bc)]_{=}
    \]
\end{definition}

\begin{note}
    整数的加法和乘法都是良定义(Well-defined)的,即不依赖于代表元的选取,而且是封闭的。证明略。自然数的加法和乘法在整数集中保持不变。
    例如,$[2-0]_{=} + [3-0]_{=} = [(2+3)-(0+0)]_{=} = [5-0]_{=}$,即 $2+3=5$。
\end{note}

\vspace{1em}

\begin{definition}[整数加法的逆元 Additive Inverse of Integers]
    设 $[a-b]_{=}\in\mathbb{Z}$,定义其加法逆元为
    \[
        -[a-b]_{=} = [b-a]_{=}
    \]
    使得,$[a-b]_{=} + [-[a-b]_{=} = [0-0]_{=}$。
\end{definition}

\begin{definition}[整数的减法 Subtraction of Integers]
    设 $[a-b]_{=},[c-d]_{=}\in\mathbb{Z}$,定义减法为
    \[
        [a-b]_{=} - [c-d]_{=} = [a-b]_{=} + (-[c-d]_{=}) = [(a+d)-(b+c)]_{=}
    \]
\end{definition}

\begin{note}
    可以证明,任何一个整数都有唯一的加法逆元。因此,整数的减法也是良定义的,即不依赖于代表元的选取,而且是封闭的。
    自然数集中,不存在加法逆元,因此也不存在减法。整数集相比自然数集,扩充了加法逆元,使得加法和减法都能良定义。
\end{note}

\vspace{1em}

\begin{theorem}[整数代数运算的性质]
    设 $x,y,z\in\mathbb{Z}$,则
    \begin{enumerate}
        \item 交换律:$x+y=y+x$,$xy=yx$
        \item 结合律:$(x+y)+z=x+(y+z)$,$(xy)z=x(yz)$
        \item 分配律:$x(y+z)=xy+xz$
        \item 存在加法单位元 $0$,使得 $\forall x\in\mathbb{Z}$,有 $x+0=0+x=x$
        \item 存在乘法单位元 $1$,使得 $\forall x\in\mathbb{Z}$,有 $x\cdot 1 = 1\cdot x = x$
        \item 存在加法逆元 $-x$,使得 $\forall x\in\mathbb{Z}$,有 $x+(-x)=(-x)+x=0$
        \item 乘法对加法的零因子:$\forall x,y\in\mathbb{Z}$,若 $xy=0$,则 $x=0$ 或 $y=0$
    \end{enumerate}
\end{theorem}

\vspace{1em}

\begin{definition}[整数的序关系 Order Relation on Integers]
    设 $[a-b]_{=},[c-d]_{=}\in\mathbb{Z}$,定义序关系为
    \[
        [a-b]_{=} \leq [c-d]_{=} \iff a+d \leq b+c
    \]    
\end{definition}

\vspace{1em}

\begin{definition}[整数的绝对值 Absolute Value of Integers]
    设 $[a-b]_{=}\in\mathbb{Z}$,定义其绝对值为
    \[
        |[a-b]_{=}| = 
        \begin{cases}
            [a-b]_{=} & a\geq b \\
            -[a-b]_{=} & a < b
        \end{cases}
    \]
\end{definition}

\begin{definition}[整数的绝对值距离 Distance of Absolute Value of Integers]
    设 $x,y\in\mathbb{Z}$,定义它们的绝对值距离为
    \[
        d(x,y) = |x-y|
    \]
\end{definition}

\begin{theorem}[整数绝对值的性质 Properties of Absolute Value of Integers]
    设 $x,y\in\mathbb{Z}$,则
    \begin{enumerate}
        \item 非负性:$|x|\geq 0$,且 $|x|=0 \iff x=0$
        \item 积的绝对值:$|xy|=|x||y|$
        \item 三角不等式:$|x+y|\leq |x|+|y|$
        \item 反三角不等式:$||x| - |y|| \leq |x - y|$
        \item 距离的非负性:$d(x,y) \geq 0$,且 $d(x,y)=0 \iff x=y$
        \item 距离的三角不等式:$d(x,z) \leq d(x,y) + d(y,z)$
        \item 距离的对称性:$d(x,y) = d(y,x)$
    \end{enumerate}
\end{theorem}

\vspace{1em}

\begin{note}
    三元组 $(\mathbb{Z},+,\cdot)$ 是一个交换整环。二元组 $(\mathbb{Z},\leq)$ 是一个全序集。二元组 $(\mathbb{Z},d)$ 是一个度量空间。
\end{note}

\newpage