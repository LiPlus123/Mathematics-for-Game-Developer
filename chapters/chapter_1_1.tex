\chapter{集合论简介}
\section{朴素集合论}
\textbf{集合(Set)}是由一些事物汇集一起组成的整体,这些事物称为集合的\textbf{元素(Element)}。集合通常用大写英文字母表示,集合中的元素用小写字母表示。不含任何元素的集合称为空集,记为 $\varnothing$。集合有如下三个特性:
\begin{enumerate}
\item \textbf{无序性}:集合中每个元素的地位是相同的,元素之间是无序的。当然,我们可以在集合上定义偏序关系之后,元素之间可以依次排序。但就集合本身,元素之间是没有必然的序的。 
\item \textbf{互异性}:集合中的元素是互不相同的,每个元素只出现一次。
\item \textbf{确定性}:给定一个集合 $A$,任何一个对象要么属于集合 $A$,要么不属于集合 $A$,二者必居其一,不允许有模棱两可的情况出现。
\end{enumerate}

\begin{axiom}[概括公理模式 Axiom Schema of Comprehension]
    假设任意属性 $P(x)$,存在一个集合 $A$,使得对于任意对象 $x$, $x\in A$ 当且仅当 $P(x)$ 为真。集合 $A$ 记为 $\{x|P(x)\}$ 或 $\{x:P(A)\}$。
\end{axiom}

\begin{note}
    公理(Axiom)是一条命题。公理模式(Axiom Schema)中的谓词 $P$ 是可变的,公理模式实际上包含无穷多条公理。选择具体的一个谓词,就得到一条命题。
\end{note}
\vspace{1em}

概括公理模式是朴素集合论中构造集合的基本方法,换句话说,存在一个包含万事万物的集合,任选一条属性,都能从中筛选元素建构新的集合。概括公理模式限制太少,所以蕴含一个经典的悖论——\textbf{罗素悖论(Russell's Paradox)},若存在一个集合 $R$,它是由“所有不包括自身的集合”所组成,也即
    \[
        R = \{x:x\notin x, \text{x is a set}\}
    \]

那么,$R$ 是否包含自身呢?如果 $R\in R$,那么根据 $R$ 的定义,$R\notin R$;反之,如果 $R\notin R$,那么根据 $R$ 的定义,$R\in R$。无论哪种情况,都会导致矛盾。消解罗素悖论需要提出新的公理。\textbf{ZFC 公理系统}是通过限制概括公理模式来规避罗素悖论。在 \textbf{NBG 公理系统}中,引入比集合更高阶的“类(Class)”,从而避免讨论“包含所有集合的集合”,而是“包含所有集合的类”。下面主要介绍 ZFC 公理系统。
