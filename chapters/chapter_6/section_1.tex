\chapter{点集拓扑}

\section{拓扑空间的定义}

\subsection{开集与闭集}
\begin{definition}[拓扑空间 Topological Space]
    设$ X $是一个集合,$ \mathcal{T}(X) \subseteq \mathcal{P}(X) $ 是 $ X $ 的子集族,二元组 $ (X,\mathcal{T}) $ 称为\textbf{拓扑空间},当且仅当,$ \mathcal{T} $ 满足:
    \begin{enumerate}
        \item 空集与全集:$ \varnothing \in \mathcal{T} $ 且 $ X \in \mathcal{T} $;
        \item 对任意并封闭:任意 $ \{U_i:i\in I\} \subseteq \mathcal{T} $,有 $ \bigcup_{i\in I} U_i \in \mathcal{T} $;
        \item 对有限交封闭:有限个 $ U_1,U_2,\ldots,U_n\in \mathcal{T} $,有 $ U_1\cap U_2\cap \cdots \cap U_n \in \mathcal{T} $。
    \end{enumerate}
    子集族 $ \mathcal{T} $ 称为 $ X $ 上的\textbf{拓扑 Topology},$ \mathcal{T} $ 中的元素称为\textbf{开集 Open Set}。
    \label{definition:topological_space}
\end{definition}

\begin{note}
    根据拓扑空间的的定义,幂集 $ \mathcal{P}(X) $ 和 $ \{\varnothing,X\} $ 都是 $ X $ 上的拓扑,分别称为\textbf{离散拓扑 Discrete Topology}和\textbf{平凡拓扑 Trivial Topology}。
    给定集合 $X$,$X$ 的任意子集族并非都是 $X$ 上的拓扑,$X$ 的拓扑也非唯一。比如,设 $ X=\{a,b,c\} $,那么下面的子集族都是 $ X $ 上的拓扑:
    \begin{align*}
        &\mathcal{T}_1(X)=\{\varnothing,X\},\\
        &\mathcal{T}_2(X)=\mathcal{P}(X),\\
        &\mathcal{T}_3(X)=\{\varnothing,X,\{a\}\},\\
        &\mathcal{T}_4(X)=\{\varnothing,X,\{a,b\}\},\\
        &\mathcal{T}_5(X)=\{\varnothing,X,\{a\},\{a,b\}\},\\
        &\mathcal{T}_6(X)=\{\varnothing,X,\{a,b\},\{b\},\{b,c\}\}.
    \end{align*}
    下面的子集族不是 $ X $ 上的拓扑:
    \begin{align*}
        &\mathcal{T}_7(X)=\{\varnothing,X,\{a\},\{b\}\},\\
        &\mathcal{T}_8(X)=\{\varnothing,X,\{a,b\},\{b,c\}\}.
    \end{align*}
\end{note}
\vspace{1em}

\begin{definition}[闭集 Closed Set]
    设 $ (X,\mathcal{T}) $ 是拓扑空间,$ X $ 的子集 $ A\subseteq X $ 称为\textbf{闭集},当且仅当 $ \complement_X A \in \mathcal{T} $。
    \label{definition:closed_set}
\end{definition}

\begin{proposition}[闭集的性质]
    设 $ (X,\mathcal{T}(X)) $ 为拓扑空间,则有:
    \begin{enumerate}
        \item 空集与全集:$ \varnothing $ 与 $ X $ 是闭集;
        \item 对任意交封闭:$ \bigcap_{i\in I} A_i $ 是闭集,其中 $ \{A_i:i\in I\}$ 是闭集;
        \item 对有限并封闭:$ A_1\cup A_2\cup \cdots \cup A_n $ 是闭集,其中 $ A_1,A_2,\ldots,A_n $ 是闭集。
    \end{enumerate}
\end{proposition}

\begin{note}
    在拓扑空间,开集的补集就是闭集,闭集的补集是开集。
    开集是对几何空间中开区间、开球等概念的抽象,可以用来定义点的邻域、连续性等概念。
    闭集是对几何空间中闭区间、闭球等概念的抽象,可以用来定义点的极限、闭包等概念。
    空集和全集既是开集又是闭集,当然也存在既非开集也非闭集的集合,在后面会看到这样的例子。
\end{note}

\vspace{1em}
\subsection{领域和聚点}

\begin{definition}[领域 Neighborhood]
    设 $(X,\mathcal{T})$ 是拓扑空间,$x\in X$ 的\textbf{领域}记为 $N(x)\subseteq X$,定义为:存在开集 $U\in\mathcal{T}$,使得:
    \[
        x\in U \subseteq N(x) \subseteq X
    \]
    领域 $N(x)$ 不一定是开集。如果 $N(x)\in\mathcal{T}$ 是开集,则称为\textbf{开领域}。
    \label{def:neighborhood}
\end{definition}
\begin{example}
    
\end{example}

\begin{note}
    领域就是拓扑空间中对点“附近”的一种抽象。
\end{note}
\vspace{1em}

\begin{definition}[聚点 Accumulation Point]
    设 $(X,\mathcal{T})$ 是拓扑空间,$A\subseteq X$,点 $x\in X$ 称为 $A$ 的\textbf{聚点},当且仅当,$x$ 的每个领域 $N(x)$ 都与 $A$ 有交集,且
    \[
        N(x) \cap (A\setminus\{x\}) \neq \varnothing.
    \]
    \label{def:accumulation_point}
\end{definition}

\begin{example}
    
\end{example}

\begin{note}
    $x$ 是 $A$ 的聚点,并不需要 $x\in A$,只需要 $x$ 的每个领域与 $A\setminus\{x\}$ 有交集即可。
\end{note}

\begin{proposition}[闭集判定的充要条件]
    设 $(X,\mathcal{T})$ 是拓扑空间,$A\subseteq X$,则 $A$ 是闭集当且仅当 $A$ 的每个聚点都属于 $A$。
\end{proposition}
\begin{proof}
    必要性:设 $A$ 是闭集,$x$ 是 $A$ 的聚点,若 $x\notin A$,则 $x\in \complement_X A$。因为 $A$ 是闭集,所以 $\complement_X A$ 是开集,因此,存在开集 $U\in \mathcal{T}$,使得
    \[
        x\in U \subseteq \complement_X A.
    \]
    这就说明 $U$ 与 $A\setminus\{x\}$ 没有交集,这与 $x$ 是 $A$ 的聚点矛盾,所以 $x\in A$。

    充分性:设 $A$ 的每个聚点都属于 $A$,若 $A$ 不是闭集,则 $\complement_X A$ 不是开集,所以存在点 $x\in \complement_X A$,使得对任意开集 $U\in \mathcal{T}$,只要 $x\in U$,就有
    \[
        U \cap A \neq \varnothing.
    \]
    这就说明 $x$ 的每个领域都与 $A\setminus\{x\}$ 有交集,所以 $x$ 是 $A$ 的聚点,由假设可知,$x\in A$,这与 $x\in \complement_X A$ 矛盾,所以 $A$ 是闭集。
\end{proof}

\vspace{1em}

\subsection{集合的内外与边界}

\vspace{1em}


\newpage