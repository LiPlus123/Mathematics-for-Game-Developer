\chapter{点集拓扑}

\section{拓扑空间的定义}

\subsection{开集与闭集}
\begin{definition}[拓扑空间 Topological Space]
    设 $ X $ 是一个集合,$ \mathcal{T}(X) \subseteq \mathcal{P}(X) $ 是 $ X $ 的子集族,二元组 $ (X,\mathcal{T}) $ 称为\textbf{拓扑空间},当且仅当,$ \mathcal{T} $ 满足:
    \begin{enumerate}
        \item 空集与全集:$ \varnothing \in \mathcal{T} $ 且 $ X \in \mathcal{T} $;
        \item 对任意并封闭:任意 $ \{U_i:i\in I\} \subseteq \mathcal{T} $,有 $ \bigcup_{i\in I} U_i \in \mathcal{T} $;
        \item 对有限交封闭:有限个 $ U_1,U_2,\ldots,U_n\in \mathcal{T} $,有 $ U_1\cap U_2\cap \cdots \cap U_n \in \mathcal{T} $。
    \end{enumerate}
    子集族 $ \mathcal{T} $ 称为 $ X $ 上的\textbf{拓扑 Topology},$ \mathcal{T} $ 中的元素称为\textbf{开集 Open Set}。
    \label{definition:topological_space}
\end{definition}

\begin{note}
    根据拓扑空间的的定义,幂集 $ \mathcal{P}(X) $ 和 $ \{\varnothing,X\} $ 都是 $ X $ 上的拓扑,分别称为\textbf{离散拓扑 Discrete Topology}和\textbf{平凡拓扑 Trivial Topology}。
    给定集合 $X$,$X$ 的任意子集族并非都是 $X$ 上的拓扑,$X$ 的拓扑也非唯一。比如,设 $ X=\{a,b,c\} $,那么下面的子集族都是 $ X $ 上的拓扑:
    \begin{align*}
        &\mathcal{T}_1(X)=\{\varnothing,X\},\\
        &\mathcal{T}_2(X)=\mathcal{P}(X),\\
        &\mathcal{T}_3(X)=\{\varnothing,X,\{a\}\},\\
        &\mathcal{T}_4(X)=\{\varnothing,X,\{a,b\}\},\\
        &\mathcal{T}_5(X)=\{\varnothing,X,\{a\},\{a,b\}\},\\
        &\mathcal{T}_6(X)=\{\varnothing,X,\{a,b\},\{b\},\{b,c\}\}.
    \end{align*}
    下面的子集族不是 $ X $ 上的拓扑:
    \begin{align*}
        &\mathcal{T}_7(X)=\{\varnothing,X,\{a\},\{b\}\},\\
        &\mathcal{T}_8(X)=\{\varnothing,X,\{a,b\},\{b,c\}\}.
    \end{align*}
\end{note}
\vspace{1em}

\begin{definition}[闭集 Closed Set]
    设 $ (X,\mathcal{T}) $ 是拓扑空间,$ X $ 的子集 $ A\subseteq X $ 称为\textbf{闭集},当且仅当 $ \complement_X A \in \mathcal{T} $。
    \label{definition:closed_set}
\end{definition}

\begin{proposition}[闭集的性质]
    设 $ (X,\mathcal{T}(X)) $ 为拓扑空间,则有:
    \begin{enumerate}
        \item 空集与全集:$ \varnothing $ 与 $ X $ 是闭集;
        \item 对任意交封闭:$ \bigcap_{i\in I} A_i $ 是闭集,其中 $ \{A_i:i\in I\}$ 是闭集;
        \item 对有限并封闭:$ A_1\cup A_2\cup \cdots \cup A_n $ 是闭集,其中 $ A_1,A_2,\ldots,A_n $ 是闭集。
    \end{enumerate}
\end{proposition}
\vspace{1em}

\begin{example}[实数集的拓扑]
    设 $(\mathbb{R},\mathcal{T})$,则实数集上的拓扑为:
    \[
        \mathcal{T}(\mathbb{R}) = \{U \subseteq \mathbb{R} : U \text{ 是开区间的任意并集}\} \cup \{\varnothing,\mathbb{R}\}.
    \]
    其中,
    \begin{enumerate}
        \item 任意开区间 $ (a,b) $ 是开集;$(a,+\infty)$ 和 $(-\infty,b)$ 也是开集;
        \item 任意闭区间 $ [a,b] $ 是闭集,它是开区间 $(-\infty,b)\cap(a,+\infty)$ 的补集;
        \item 有限集是闭集,比如 $\{a\}$ 是闭集,因为它是开集 $(-\infty,a)\cap(a,+\infty)$ 的补集;
        \item $\mathbb{R}$ 和 $\varnothing$ 既是开集也是闭集
        \item 任意半开半闭区间 $ [a,b) $ 和 $ (a,b] $ 既不是开集也不是闭集。
    \end{enumerate}
    \label{ex:topology_on_real_numbers}
\end{example}

\begin{note}
    在拓扑空间,开集的补集就是闭集,闭集的补集是开集。
    开集是对几何空间中开区间、开球等概念的抽象;
    闭集是对几何空间中闭区间、闭球等概念的抽象。
    空集和全集既是开集又是闭集,当然也存在既非开集也非闭集的集合。
\end{note}

\vspace{1em}
\subsection{领域和聚点}

\begin{definition}[领域 Neighborhood]
    设 $(X,\mathcal{T})$ 是拓扑空间,$x\in X$ 的\textbf{领域}记为 $N(x)\subseteq X$,定义为:存在开集 $U\in\mathcal{T}$,使得:
    \[
        x\in U \subseteq N(x) \subseteq X
    \]
    领域 $N(x)$ 不一定是开集。如果 $N(x)\in\mathcal{T}$ 是开集,则称为\textbf{开领域}。
    \label{def:neighborhood}
\end{definition}
\begin{example}
    在 $(\mathbb{R},\mathcal{T})$ 中,设 $x=0$,则
    \begin{enumerate}
        \item $N_1(0)=(-1,1)$ 是 $0$ 的开领域,因为 $(-1,1)$ 是开集,且 $0\in(-0.5,0.5)\subseteq(-1,1)$;
        \item $N_2(0)=[-1,1]$ 是 $0$ 的领域,因为存在开集 $(-0.5,0.5)$,使得 $0\in(-0.5,0.5)\subseteq[-1,1]$,但 $[-1,1]$ 不是开集,所以不是开领域;
        \item $N_3(0)=\{0\}\cup(1,2)$ 不是 $0$ 的领域,因为不存在开集 $U$,使得 $0\in U \subseteq \{0\}\cup(1,2)$。
    \end{enumerate}
\end{example}

\begin{note}
    领域就是拓扑空间中对点“附近”的一种抽象描述。
\end{note}
\vspace{1em}

\begin{definition}[聚点 Accumulation Point]
    设 $(X,\mathcal{T})$ 是拓扑空间,$A\subseteq X$,点 $x\in X$ 称为 $A$ 的\textbf{聚点},当且仅当,$x$ 的每个领域 $N(x)$ 都与 $A$ 有交集,且
    \[
        N(x) \cap (A\setminus\{x\}) \neq \varnothing.
    \]
    \label{def:accumulation_point}
\end{definition}

\begin{example}
    在 $(\mathbb{R},\mathcal{T})$ 中,设 $A=\{1/n:n\in\mathbb{Z}^+\}$,则
    \begin{enumerate}
        \item $0$ 是 $A$ 的聚点,因为 $0$ 的每个领域都与 $A$ 有交集;
        \item $1$ 不是 $A$ 的聚点,因为存在开领域 $(0.5,1.5)$,使得 $(0.5,1.5)\cap (A\setminus\{1\})=\varnothing$;
        \item $1/2$ 不是 $A$ 的聚点,因为存在开领域 $(0.4,0.6)$,使得 $(0.4,0.6)\cap (A\setminus\{1/2\})=\varnothing$。
    \end{enumerate}
    在实数集中,柯西序列的极限都是聚点。
\end{example}

\begin{note}
    $x$ 是 $A$ 的聚点,并不需要 $x\in A$,只需要 $x$ 的每个领域与 $A\setminus\{x\}$ 有交集即可。
    比如,在例 \ref{ex:sqrt2-approx} 中,设 $A=\{x\in\mathbb{Q}:x=1,1.4,1.41,1.414,1.4142,\ldots\}$ 是一个有理柯西序列,
    $\sqrt{2}$ 是 $A$ 的聚点,但 $\sqrt{2}\notin A$。
\end{note}
\vspace{1em}

\begin{proposition}[闭集判定的充要条件]
    设 $(X,\mathcal{T})$ 是拓扑空间,$A\subseteq X$,则 $A$ 是闭集当且仅当 $A$ 的每个聚点都属于 $A$。
    \label{prop:closed_set_criteria_1}
\end{proposition}

\begin{proof}
    必要性:设 $A$ 是闭集,$x$ 是 $A$ 的聚点,若 $x\notin A$,则 $x\in \complement_X A$。因为 $A$ 是闭集,所以 $\complement_X A$ 是开集,因此,存在开集 $U\in \mathcal{T}$,使得
    \[
        x\in U \subseteq \complement_X A.
    \]
    这就说明 $U$ 与 $A\setminus\{x\}$ 没有交集,这与 $x$ 是 $A$ 的聚点矛盾,所以 $x\in A$。

    充分性:设 $A$ 的每个聚点都属于 $A$,若 $A$ 不是闭集,则 $\complement_X A$ 不是开集,所以存在点 $x\in \complement_X A$,使得对任意开集 $U\in \mathcal{T}$,只要 $x\in U$,就有
    \[
        U \cap A \neq \varnothing.
    \]
    这就说明 $x$ 的每个领域都与 $A\setminus\{x\}$ 有交集,所以 $x$ 是 $A$ 的聚点,由假设可知,$x\in A$,这与 $x\in \complement_X A$ 矛盾,所以 $A$ 是闭集。
\end{proof}

\vspace{1em}

\subsection{集合的内部与闭包}

\begin{definition}[集合的内部与闭包]
    设 $(X,\mathcal{T})$ 是拓扑空间,$A\subseteq X$,$A$ 的内部记为 $\operatorname{int}(A)$,定义为:
    \[
        \operatorname{int}(A) = \bigcup \{U \in \mathcal{T} : U \subseteq A\} 
    \]
    也即 $A$ 的内部是所有包含于 $A$ 的开集的并。
    \label{def:set_interior}
\end{definition}

\begin{proposition}
    设 $(X,\mathcal{T})$ 是拓扑空间,$A\subseteq X$,则 $\operatorname{int}(A)$ 是包含于 $A$ 的最大开集。
\end{proposition}
\vspace{1em}

\begin{definition}[集合的闭包 Closure]
    设 $(X,\mathcal{T})$ 是拓扑空间,$A\subseteq X$,$A$ 的闭包记为 $\operatorname{cl}(A)$ 或 $\overline{A}$,定义为:
    \[
        \operatorname{cl}(A) = \bigcap \{F \subseteq X : F \text{ 是闭集且 } A \subseteq F\}
    \]
    也即 $A$ 的闭包是所有包含 $A$ 的闭集的交。
    \label{def:set_closure}
\end{definition}

\begin{proposition}
    设 $(X,\mathcal{T})$ 是拓扑空间,$A\subseteq X$,则 $\operatorname{cl}(A)$ 是包含 $A$ 的最小闭集。
\end{proposition}
\vspace{1em}

\begin{proposition}[开集和闭集判定的充要条件]
    设 $(X,\mathcal{T})$ 是拓扑空间,$A\subseteq X$,则
    \begin{enumerate}
        \item $A$ 是开集当且仅当 $\operatorname{int}(A) = A$;
        \item $A$ 是闭集当且仅当 $\operatorname{cl}(A) = A$。
    \end{enumerate}
    \label{prop:open_closed_set_criteria_1}
\end{proposition}

\begin{proposition}[集合内部与闭包的关系]
    设 $(X,\mathcal{T})$ 是拓扑空间,$A\subseteq X$,则
    \[
        \operatorname{cl}(A) = A \cup \{\text{$A$ 的所有聚点}\}.
    \]
\end{proposition}

\begin{proof}
    设 $F = A \cup \{\text{$A$ 的所有聚点}\}$,则 $A \subseteq F$。若 $x \in F$ 的补集 $\complement_X F$,则 $x \notin A$ 且 $x$ 不是 $A$ 的聚点,所以存在开集 $U \in \mathcal{T}$,使得
    \[
        x \in U \subseteq \complement_X A.
    \]
    这就说明 $\complement_X F$ 是开集,所以 $F$ 是闭集,因此,$\operatorname{cl}(A) \subseteq F$。若 $F \not\subseteq \operatorname{cl}(A)$,则存在 $x \in F$,使得 $x \notin \operatorname{cl}(A)$,所以 $x \in \complement_X \operatorname{cl}(A)$。因为 $\operatorname{cl}(A)$ 是闭集,所以 $\complement_X \operatorname{cl}(A)$ 是开集,因此,存在开集 $V \in \mathcal{T}$,使得
    \[
        x \in V \subseteq \complement_X \operatorname{cl}(A).
    \]
    这就说明 $V$ 与 $A\setminus\{x\}$ 没有交集,所以 $x$ 不是 $A$ 的聚点,由假设可知,$x\in A$,这与 $x\notin \operatorname{cl}(A)$ 矛盾,
    所以 $F \subseteq \operatorname{cl}(A)$。综上所述,$\operatorname{cl}(A) = F = A \cup \{\text{$A$ 的所有聚点}\}$。
\end{proof}
\vspace{1em}

\begin{example}
    在 $(\mathbb{R},\mathcal{T})$ 中,设 $A=(1,2]$,则
    \begin{enumerate}
        \item $A$ 的内部为 $\operatorname{int}(A)=(1,2)$,它是包含于 $A$ 的最大开集;
        \item $A$ 的闭包为 $\operatorname{cl}(A)=(1,2) \cup \{1,2\} = [1,2]$,它是包含 $A$ 的最小闭集。
    \end{enumerate}
\end{example}

\begin{note}
    集合内部中的点称为集合的\textbf{内点},根据命题 \ref{prop:open_closed_set_criteria_1} 可知,开集中的点都是内点。
\end{note}

\vspace{1em}

\subsection{集合的边界与外部}
\begin{definition}[集合的边界 Boundary]
    设 $(X,\mathcal{T})$ 是拓扑空间,$A\subseteq X$,$A$ 的\textbf{边界}记为 $\partial A$,定义为:
    \[
        \partial A = \operatorname{cl}(A) \setminus \operatorname{int}(A)
    \]
    \label{def:set_boundary}
\end{definition}

\begin{proposition}[闭包分解]
    设 $(X,\mathcal{T})$ 是拓扑空间,$A\subseteq X$,则 $\operatorname{cl}(A) = \operatorname{int}(A) \cup \partial(A) = A \cup \partial(A) $。
    \label{prop:closure_decomposition}
\end{proposition}
\begin{proof}
    因为 $\operatorname{int}(A)\subseteq A \subseteq \operatorname{cl}(A)$,所以只需证明 $\operatorname{cl}(A) = \operatorname{int}(A) \cup \partial(A)$。
    根据定义,$\partial A = \operatorname{cl}(A) \setminus \operatorname{int}(A) = \complement_{\operatorname{cl}(A)} \operatorname{int}(A)$,所以,
    \[
        \operatorname{int}(A) \cup \partial(A) = \operatorname{int}(A) \cup (\complement_{\operatorname{cl}(A)} \operatorname{int}(A)) = \operatorname{cl}(A)
    \]
\end{proof}
\vspace{1em}

\begin{definition}[集合的外部 Exterior]
    设 $(X,\mathcal{T})$ 是拓扑空间,$A\subseteq X$,$A$ 的\textbf{外部}记为 $\operatorname{ext}(A)$,定义为:
    \[
        \operatorname{ext}(A) = \complement_X \operatorname{cl}(A).
    \]
    \label{def:set_exterior}
\end{definition}

\begin{proposition}[集合的外部是开集]
    设 $(X,\mathcal{T})$ 是拓扑空间,$A\subseteq X$,则 $\operatorname{ext}(A)$ 是开集,并且
    \[
        \operatorname{ext}(A) = \complement_X \operatorname{cl}(A) = \operatorname{int}(\complement_X A)
    \]
    \label{prop:exterior_is_open}
\end{proposition}
\begin{proof}
    因为 $\operatorname{cl}(A)$ 是闭集,所以 $\complement_X \operatorname{cl}(A)$ 是开集,因此,$\operatorname{ext}(A)$ 是开集。
    根据集合外部的定义,$\operatorname{ext}(A) = \complement_X \operatorname{cl}(A) = \{x\in X : x \notin A \text{并且} x\notin \{A \text{\ 的所有聚点}\}\}$。
    因为 $\forall x \in \operatorname{int}(\complement_X A) \subseteq \complement_X A$,所以 $x \notin A$;又因为,$\forall x\in \operatorname{int}(\complement_X A)$,
    存在 $ x\in U \subseteq \complement_X A$,根据集合的矛盾律 \ref{prop:sets_laws} $ \complement_X A \cap A = \varnothing$ 所以 $U \cap A = \varnothing$,说明 $x$ 不是 $A$ 的聚点。
    那么 $\forall x\in \operatorname{int}(\complement_X A),\ x\in \{x\in X : x \notin A \text{并且} x\notin \{A \text{\ 的所有聚点}\}\} = \complement_X \operatorname{cl}(A)$,
\end{proof}
\vspace{1em}

\begin{proposition}
    设 $(X,\mathcal{T})$ 是拓扑空间,$A\subseteq X$,则
    \[
        X = \operatorname{int}(A) \cup \partial(A) \cup \operatorname{ext}(A)
    \]
    并且,
    \begin{align*}
        \operatorname{int}(A) \cap \partial(A) &= \varnothing,\\
        \operatorname{int}(A) \cap \operatorname{ext}(A) &= \varnothing,\\
        \partial(A) \cap \operatorname{ext}(A) &= \varnothing
    \end{align*}
    \label{prop:topology_space_decomposition}
\end{proposition}
\begin{proof}
    根据边界 \ref{def:set_boundary} 和外部 \ref{def:set_exterior} 的定义,以及命题 \ref{prop:closure_decomposition} 和 \ref{prop:exterior_is_open} 可以非常容易证明。
\end{proof}

\begin{note}
    集合的内部,边界和外部,将全集划分为三个互不相交的部分。
\end{note}
\vspace{1em}

\begin{example}
    在 $(\mathbb{R},\mathcal{T})$ 中,设 $A=(1,2]$,则
    \begin{enumerate}
        \item $A$ 的边界为 $\partial A= [1,2]\setminus (1,2) = \{1,2\}$;
        \item $A$ 的外部为 $\operatorname{ext}(A) = (-\infty,+\infty) \setminus [1,2] = (-\infty,1)\cup(2,+\infty)$。
    \end{enumerate}
\end{example}
\vspace{1em}

\begin{proposition}
    设 $(X,\mathcal{T})$ 是拓扑空间,$A\subseteq X$,则有:
    \[
        \complement_X(\operatorname{int}(A)) = \operatorname{cl}(\complement_X A)
    \]
    \label{prop:complement_interior_closure}
\end{proposition}

\begin{proof}
    根据内部 \ref{def:set_interior} 的定义,$\operatorname{int}(A) = \bigcup \{U \in \mathcal{T} : U \subseteq A\}$。根据德摩根律 \ref{prop:sets_laws} 有:
    \[
        \complement_X(\operatorname{int}(A)) = \complement_X\left(\bigcup \{U \in \mathcal{T} : U \subseteq A\}\right) = \bigcap \{\complement_X U : U \in \mathcal{T} \text{且} U \subseteq A\}.
    \]
    因为 $U$ 是开集,所以 $\complement_X U$ 是闭集,并且 $A\subseteq U$ 等价于 $\complement_X U \subseteq \complement_X A$,根据闭包 \ref{def:set_closure} 的定义,有:
    \[
        \complement_X(\operatorname{int}(A)) = \bigcap \{\complement_X U : U \in \mathcal{T} \text{且} A\subseteq U\}  = \operatorname{cl}(\complement_X A).
    \]
\end{proof}

\begin{corollary}
    设 $(X,\mathcal{T})$ 是拓扑空间,$A\subseteq X$,则 。
    \begin{enumerate}
        \item 集合的边界等于补集的边界:$\partial A = \partial (\complement_X A)$
        \item 集合的边界等于闭包与补集闭包的交:$\partial A = \operatorname{cl}(A) \cap \operatorname{cl}(\complement_X A)$;
    \end{enumerate}
\end{corollary}
\begin{proof}
    推论 1:根据命题 \ref{prop:topology_space_decomposition} 有:
    \[
        \complement_X(\operatorname{int}(A)) = \partial(A) \cup \operatorname{ext}(A)
    \]
    根据命题 \ref{prop:closure_decomposition} 有:
    \[
        \operatorname{cl}(\complement_X A) = \partial(\complement_X A) \cup \operatorname{int}(\complement_X A)
    \]
    根据命题 \ref{prop:complement_interior_closure} $\complement_X(\operatorname{int}(A)) = \operatorname{cl}(\complement_X A)$,那么:
    \[
        \partial(A) \cup \operatorname{ext}(A) = \partial(\complement_X A) \cup \operatorname{int}(\complement_X A)
    \]
    根据命题 \ref{prop:exterior_is_open} $\operatorname{ext}(A) = \operatorname{int}(\complement_X A)$。
    并且 $\partial A \cap \operatorname{ext}(A) = \varnothing,\ \partial(\complement_X A) \cap \operatorname{int}(\complement_X A) = \varnothing$,所以:
    \[
        \partial A = \partial (\complement_X A) = \partial (\complement_X A)
    \]
    推论 2:
    \[
        \operatorname{cl}(A) \cap \operatorname{cl}(\complement_X A) =\operatorname{cl}(A) \cap \complement_X \operatorname{int}(A) = \operatorname{cl}(A) \cap [\partial(A) \cup \operatorname{ext}(A)]
    \]
    根据集合交并的分配律 \ref{prop:sets_laws} 有:
    \[
        \operatorname{cl}(A) \cap [\partial(A) \cup \operatorname{ext}(A)] = [\operatorname{cl}(A) \cap \partial(A)] \cup [\operatorname{cl}(A) \cap \operatorname{ext}(A)]
    \]
    其中,$\operatorname{cl}(A) \cap \partial(A) = \varnothing,\ \operatorname{cl}(A) \cap \partial(A) = \partial A$,所以:
    \[
        \operatorname{cl}(A) \cap \operatorname{cl}(\complement_X A) = [\operatorname{cl}(A) \cap \partial(A)] \cup [\operatorname{cl}(A) \cap \operatorname{ext}(A)] = \partial A
    \]
\end{proof} 

\begin{note}
    总结一下,在拓扑空间 $(X,\mathcal{T})$ 中,子集 $A\subseteq X$ 的内部、闭包、边界和外部的关系如下:
    \begin{enumerate}
        \item 内部:$\operatorname{int}(A)$ 是包含于 $A$ 的最大开集;
        \item 闭包:$\operatorname{cl}(A)$ 是包含 $A$ 的最小闭集,并且 $\operatorname{cl}(A) = A \cup \{\text{$A$ 的所有聚点}\}$;
        \item 边界:$\partial A = \operatorname{cl}(A) \setminus \operatorname{int}(A) = \operatorname{cl}(A) \cap \operatorname{cl}(\complement_X A)  = \partial (\complement_X A)$;
        \item 外部:$\operatorname{ext}(A) = \complement_X \operatorname{cl}(A) = \operatorname{int}(\complement_X A)$;
        \item 补集:$\complement_X(\operatorname{int}(A)) = \operatorname{cl}(\complement_X A)$;
        \item 分解:$X = \operatorname{int}(A) \cup \partial(A) \cup \operatorname{ext}(A)$,并且 $\operatorname{int}(A)$、$\partial(A)$ 和 $\operatorname{ext}(A)$ 互不相交。
    \end{enumerate}
    比如,在 $(\mathbb{R},\mathcal{T})$ 中,设 $A=(1,2]$,则
    \begin{enumerate}
        \item $A$ 的补集为 $\complement_{\mathbb{R}} A = (-\infty,1] \cup (2,+\infty)$;
        \item $A$ 的内部为 $\operatorname{int}(A)=(1,2)$;
        \item $A$ 的闭包为 $\operatorname{cl}(A)=[1,2]$;
        \item $A$ 的边界为 $\partial A= [1,2]\setminus (1,2) = \{1,2\}$;
        \item $A$ 的外部为 $\operatorname{ext}(A) = (-\infty,+\infty) \setminus [1,2] = (-\infty,1)\cup(2,+\infty)$。
    \end{enumerate}
\end{note}

\vspace{1em}

\subsection{拓扑基}

\vspace{1em}
\subsection{子空间拓扑}

\newpage