\section{实数}

从自然数到有理数的扩充都非常的自然,然而有理数并不能满足我们的所有需求。比如在上一节的末尾介绍有理数稠密性时提到,不存在有理数 $x$ 使得 $x^2=2$。为了满足这样的需求,我们需要对数系进行进一步的扩充,得到实数系。
\vspace{1em}

\begin{example}
    通过迭代法,我们可以找到一个有理数序列 $a_n$,使得 $(a_n)^2$ 趋近于 $2$:
    \begin{equation}
        a_{n+1} = \frac{1}{2}(a_n+\frac{2}{a_n}),\ n\ge 0
        \label{eq:sqrt2-approx}
    \end{equation}
    令 $a_0=1$,得到一个有理数序列:
    \begin{align*}
        a_1 &= \frac{1}{2}(1+2) = \frac{3}{2} = 1.5\\
        a_2 &= \frac{1}{2}(\frac{3}{2}+\frac{4}{3})=\frac{17}{12} = 1.41666666666\cdots\\
        a_3 &= \frac{1}{2}(\frac{17}{12}+\frac{24}{17})=\frac{577}{408}= 1.414215686274\cdots\\
        a_4 &= \frac{1}{2}(\frac{577}{408}+\frac{816}{577})=\frac{665857}{470832}=1.414213562374\cdots\\
        \vdots
    \end{align*}
    无限递归执行下去,会得到一个有理数的数列。随着 $n$ 的增大,$a_n$ 之间的间隔会越来越小。换一个首项,令 $a^{\prime}_0=1.4$,同样使用迭代式 \ref{eq:sqrt2-approx},得到另一个有理数数列:
    \begin{align*}
        a_1^{\prime} &= \frac{1}{2}(\frac{7}{5}+\frac{10}{7}) = \frac{99}{70} = 1.4142857142857\cdots\\
        a_2^{\prime} &= \frac{1}{2}(\frac{99}{70}+\frac{140}{99})=\frac{19601}{13860}=1.4142135642135642\cdots\\
        a_3^{\prime} &= \frac{1}{2}(\frac{19601}{13860}+\frac{27720}{19601})=\frac{768398401}{543339720}=1.41421356237309504\cdots\\
        a_4^{\prime} &=\frac{1}{2}(\frac{768398401}{543339720}+\frac{1086679440}{768398401})=\frac{1180872205318713601}{835002744095575440}=1.414213562373095048801\cdots\\
        \vdots
    \end{align*}
    无限递归执行下去,会得到另一个有理数的数列。随着 $ n $ 的增加,数列中项与项之间的间隔会也会越来越小。同时,随着序号 $ n $ 的增加,数列 $ a_n $ 和数列 $ a_n^{\prime} $ 中项的差 $ |a_n-a_n^{\prime}| $ 也会越来越小。
    \label{ex:sqrt2-approx}
\end{example}

\begin{note}
    我们知道,这两个序列最终都会收敛到无理数 $ \sqrt{2} $,但是 $ \sqrt{2}\notin\mathbb{Q} $。
    我们称数列 $ \{a_n\in\mathbb{Q}\} $ 和数列 $ \{a_n^{\prime}\in\mathbb{Q}\} $ 为\textbf{柯西序列},$ \sqrt{2} $ 为这两个\textbf{柯西序列的极限}。
    在有理数集中,柯西序列的极限不一定是有理数。因此我们需要对有理数集打上补丁,让这些序列的极限也在这个新集合中。
\end{note}
\vspace{1em}

\subsection{柯西序列与实数}

% 历史上,实数的定义有多种方式,一种是通过柯西序列的方式,另一种是戴德金分割,下面主要介绍柯西序列定义实数。
% \vspace{1em}

\begin{definition}[有理数序列 sequence of rational numbers]
    函数 $f:\mathbb{N}\to\mathbb{Q}$ 为有理数序列,记为 $\{a_n\}$,其中 $a_n=f(n),\ n\ge m,\cdots$。
\end{definition}
\begin{note}
    有理数序列说明,对于每个大于等于 $m$ 的自然数,都有唯一一个确定的有理数与之对应。例如:
    \begin{enumerate}
        \item 等比数列:$a_n = a_1q^{n-1},\ a_1\neq 0,\ q\neq 0,\ n=1,2,3,\cdots$
        \item 等差数列:$a_n = a_1 + (n-1)d,\ n=1,2,3\cdots$
        \item 斐波那契数列:$a_1=1,\ a_2=1,\ a_n=a_{n-1}+a_{n-2},\ n\ge 3$
        \item 素数数列:$a_1=2,\ a_2=3,\ a_3=5,\ a_4=7,\ a_5=11,\ a_6=13,\cdots$
    \end{enumerate}
\end{note}
\vspace{1em}

\begin{definition}[柯西序列 Cauchy sequence]
    设 $\{a_n\}$ 是有理数序列。$\{a_n\}$ 是柯西序列,当且仅当,对于任意给定的有理数 $\epsilon>0$,都存在一个正整数 $N$,使得当 $i,j>N$ 时,有
    \[
        |a_i - a_j| \le \epsilon
    \]
    柯西序列也称为\textbf{基本序列}。
\end{definition}
\begin{note}
    柯西序列是一种特殊的有理数序列,它要求随着序号 $n$ 的增大,序列中项与项之间的间隔越来越小。
    例如,数列 $a_n = \frac{1}{n},\ n\ge 1$ 和 $a_{n+1} = \frac{1}{2}(a_n+\frac{2}{a_n}),\ n\ge 0$ 都是柯西序列,
\end{note}
\vspace{1em}

\begin{definition}[有界序列 bounded sequence]
    设 $\{a_n\}$ 是有理数序列。如果存在有理数 $M>0$,使得对任意的 $n$ 都有 $|a_n|\le M$,则称 $\{a_n\}$ 是有界序列。
\end{definition}

\begin{theorem}
    柯西序列都是有界序列;有界序列不一定是柯西序列。
\end{theorem}

\begin{note}
    一个序列是柯西序列蕴含该序列是有界序列,但有界序列不一定是柯西序列,比如,$a_n=(-1)^n,\ n\ge 1$ 是有界序列,但不是柯西序列。
\end{note}
\vspace{1em}

\begin{definition}[实数 Real Number] 
    设 $S$ 是所有有理柯西序列的集合
    \begin{enumerate}
        \item 等于关系:对任意有理柯西序列 $\{a_n\},\{b_n\}\in S$,$\{a_n\} = \{b_n\}$ 当且仅当,对于任意有理数 $\epsilon>0$,都存在一个正整数 $N$,使得当 $n>N$ 时,有
        \[
            |a_n - b_n| \le \epsilon
        \] 
        \item 等价类:任意 $\{a_n\}\in S$ 的等价类为
        \[
            [\{a_n\}]_{=} = \{\{b_n\}\in S : \{a_n\} = \{b_n\}\}
        \]
    \end{enumerate}
    称 $S$ 关于 $=$ 关系的商集 $S/_{=}= \{[\{a_n\}]_{=} : \{a_n\}\in S\}$ 为实数集,记为 $\mathbb{R}$。
    实数集中的元素称为实数。
\end{definition}
\begin{note}
    实数是柯西序列关于等于关系的等价类,比如,数列 $a_n = \frac{1}{n},\ n\ge 1$ 和数列 $b_n = \frac{1}{n^2},\ n\ge 1$ 是不同的柯西序列,但它们满足等于关系,属于同一个等价类 $[\{a_n\}]_{=}=[\{b_n\}]_{=}$,表示同一个实数 $0$。
    再比如,在例 \ref{ex:sqrt2-approx} 中,数列 $a_n$ 和数列 $a_n^{\prime}$ 是不同的柯西序列,但它们满足等于关系,属于同一个等价类 $[\{a_n\}]_{=}=[\{a_n^{\prime}\}]_{=}$,表示同一个实数 $\sqrt{2}$。
    全体有理数可以嵌入到实数集中,即 $\forall q\in\mathbb{Q}$,有 $q = [\{q,q,q,\cdots\}]_{=}\in\mathbb{R}$。
    例如,$a_0=1,a_2=1,a_3=1,\cdots$ 和 $b_0=0,b_2=0.9,b_3=0.999,\cdots$ 都表示同一个实数 $1$。
\end{note}
\vspace{1em}

\begin{definition}[实数加法 Addition of Real Numbers]
    设 $[\{a_n\}]_{=},[\{b_n\}]_{=}\in\mathbb{R}$,定义加法为
    \[
        [\{a_n\}]_{=} + [\{b_n\}]_{=} = [\{a_n+b_n\}]_{=}
    \]
\end{definition}

\begin{proposition}
    实数加法的定义是良定义(Well-defined)的,即不依赖于代表元的选取,而且是封闭的。
\end{proposition}
\begin{proof}
    先证明实数加法的封闭性,任意两个实数相加依然是实数。因为 $\{a_n\},\{b_n\}$ 是柯西序列,
    根据柯西序列的定义,任意有理数 $\epsilon>0$,都存在一个正整数 $N_a$,使得当 $m,n>N_a$ 时,有 $|a_m - a_n| \le \epsilon/2$;
    同理,存在一个正整数 $N_b$,使得当 $m,n>N_b$ 时,有 $|b_m - b_n| \le \epsilon/2$;设序列 $\{c_n\}=\{a_n+b_n\}$,令 $N=\max(N_a,N_b)$,则当 $m,n>N$ 时,有
    \begin{align*}
        |c_m-c_n| &= |a_m+b_m - (a_n+b_n)| \\
        &= |(a_m-a_n) + (b_m-b_n)| \\
        &\le |a_m-a_n| + |b_m-b_n| \\
        &\le \varepsilon/2 +  \varepsilon/2 = \varepsilon
    \end{align*}
    所以 $\{c_n\}$ 是柯西序列。\\
    再证明实数加法与实数代表元的选择无关。
    设任意与 $\{a_n\},\{b_n\}$ 等价的柯西序列分别为 $\{a_n^{\prime}\},\{b_n^{\prime}\}$。
    那么,对于任意有理数 $\epsilon>0$,都存在一个正整数 $N_a$,使得当 $i>N_a$ 时,有 $|a_i - a_i^{\prime}| \le \epsilon/2$;
    同理,存在一个正整数 $N_b$,使得当 $j>N_b$ 时,有 $|b_j - b_j^{\prime}| \le \epsilon/2$。
    设序列 $\{c_n^{\prime}\}=\{a_n^{\prime}+b_n^{\prime}\}$,令 $N=\max(N_a,N_b)$,则当 $n>N$ 时,有
    \begin{align*}
        |c^{\prime}_n-c_n| &= |a_n^{\prime}+b_n^{\prime} - (a_n+b_n)| \\
        &= |(a_n^{\prime}-a_n)+(b_n^{\prime}-b_n)|\\
        &\le |a_n^{\prime}-a_n|+|b_n^{\prime}-b_n|\\
        &\le \varepsilon/2 + \varepsilon/2 = \varepsilon
    \end{align*}
    所以 $\{c_n^{\prime}\}$ 与 $\{c_n\}$ 等价,表示同一个实数。
\end{proof}

\vspace{1em}

\begin{definition}[实数乘法 Multiplication of Real Numbers]
    设 $[\{a_n\}]_{=},[\{b_n\}]_{=}\in\mathbb{R}$,定义乘法为
    \[
        [\{a_n\}]_{=} \cdot [\{b_n\}]_{=} = [\{a_nb_n\}]_{=}
    \]
\end{definition}

\begin{proposition}
    实数乘法的定义是良定义(Well-defined)的,即不依赖于代表元的选取,而且是封闭的。
\end{proposition}

\begin{proof}
    先证明实数乘法的封闭性,任意两个实数实数相乘依然是实数。因为 $\{a_n\},\{b_n\}$ 是柯西序列,也是有界序列,存在有理数 $M$,使得对任意的 $n$ 都有 $|a_n|\le M$ 和 $|b_n|\le M$。
    根据柯西序列的定义,任意有理数 $\epsilon>0$,都存在一个正整数 $N_a$,使得当 $m,n>N_a$ 时,有 $|a_m - a_n| \le \epsilon/2M$;
    同理,存在一个正整数 $N_b$,使得当 $m,n>N_b$ 时,有 $|b_m - b_n| \le \epsilon/2M$;
    设序列 $\{c_n\}=\{a_nb_n\}$,令 $N=\max(N_a,N_b)$,则当 $m,n>N$ 时,有:
    \begin{align*}
        |c_m-c_n| &= |a_m*b_m-a_n*b_n| \\
        &=|a_m*b_m - a_m*b_n + a_m*b_n-a_n*b_n|\\
        &\le |a_m*b_m - a_m*b_n| + |a_m*b_n-a_n*b_n|\\
        &= |a_m| |b_m-b_n|+|b_n||a_m-b_n|\\
        &\le M \frac{\varepsilon}{2M} + M\frac{\varepsilon}{2M}\\
        &= \varepsilon
    \end{align*}
    所以 $\{c_n\}$ 是柯西序列。\\
    再证明实数乘法与实数代表元的选择无关。
    设任意与 $\{a_n\},\{b_n\}$ 等价的柯西序列分别为 $\{a_n^{\prime}\},\{b_n^{\prime}\}$。
    那么,对于任意有理数 $\epsilon>0$,都存在一个正整数 $N_a$,使得当 $i>N_a$ 时,有 $|a_i - a_i^{\prime}| \le \epsilon/2M$;
    同理,存在一个正整数 $N_b$,使得当 $j>N_b$ 时,有 $|b_j - b_j^{\prime}| \le \epsilon/2M$。
    设序列 $\{c_n^{\prime}\}=\{a_n^{\prime}b_n^{\prime}\}$,令 $N=\max(N_a,N_b)$,则当 $n>N$ 时,有
    \begin{align*}
        |c_m^{\prime}-c_m| &= |a^{\prime}_m*b^{\prime}_m - a_m*b_m|\\
        & =|a^{\prime}_m*b^{\prime}_m - a^{\prime}_m*b_m + a^{\prime}_m*b_m - a_m*b_m|\\
        &\le |a^{\prime}_m*b^{\prime}_m - a^{\prime}_m*b_m|+|a^{\prime}_m*b_m - a_m*b_m|\\
        &= |a^{\prime}_m||b^{\prime}_m-b_m| + |b_m||a^{\prime}_m-a_m|\\
        & \le M \frac{\varepsilon}{2M} + M \frac{\varepsilon}{2M}\\
        &= \varepsilon
    \end{align*}
    所以 $\{c_n^{\prime}\}$ 与 $\{c_n\}$ 等价,表示同一个实数。
\end{proof}

\vspace{1em}

\begin{definition}[实数加法逆元与减法]
    设 $a=[\{a_n\}]_{=}\in\mathbb{R}$,加法逆元记为 $-a$,使得
    \[
        a + (-a) = [\{0\}]_{=}
    \]
    那么:
    \[
        -a:= [\{-a_n\}]_{=}
    \]
    设 $a_1,a_2\in\mathbb{R}$, 实数的减法定义为:
    \[
        a_1 - a_2 = a_1 + (-a_2)
    \]
\end{definition}

\begin{definition}[实数乘法逆元与除法]
    设 $a=[\{a_n\}]_{=}\in\mathbb{R}$,且 $a\neq 0$,乘法逆元记为 $a^{-1}$,使得
    \[
        a \cdot a^{-1} = [\{1\}]_{=}
    \]
    那么:
    \[
        a^{-1} := [\{a_n^{-1}\}]_{=}
    \]
    设 $a_1,a_2\in\mathbb{R},\ a_2\neq 0$, 实数的除法定义为:
    \[
        a_1 / a_2 = a_1 \cdot a_2^{-1}
    \]
\end{definition}
\vspace{1em}

\begin{proposition}[实数代数运算的性质]
    设 $a,b,c\in\mathbb{R}$,则实数的代数运算具有以下性质:
    \begin{enumerate}
        \item 交换律:$a+b=b+a$
        \item 加法结合律:$(a+b)+c=a+(b+c)$;
        \item 加法单位元:存在唯一的实数 $0$,使得 $a+0=a$;
        \item 加法逆元:任意实数 $a$ 存在唯一的实数 $-a$,使得 $a+(-a)=0$;
        \item 乘法交换律:$ab=ba$;
        \item 乘法结合律:$(ab)c=a(bc)$;
        \item 乘法单位元:存在唯一的实数 $1$,使得 $a\cdot 1=a$;
        \item 乘法逆元:任意非零实数 $a$ 存在唯一的实数 $a^{-1}$,使得 $a\cdot a^{-1}=1$;
        \item 乘法对加法的分配律:$a(b+c)=ab+ac$.
        \item 乘法零元:$a\cdot 0 = 0$.
    \end{enumerate}
\end{proposition}

\begin{note}
    实数的四则运算的定义都是良定义的,即不依赖于代表元的选取,而且是封闭的。实数集和加法乘法一起,构成一个数域,称为实数域,记为 $(\mathbb{R},+,\cdot)$。
\end{note}

\vspace{1em}

\begin{definition}[实数的序 Order of Real Numbers]
    设 $[\{a_n\}]_{=},[\{b_n\}]_{=}\in\mathbb{R}$,定义序为
    \[
        [\{a_n\}]_{=} \le [\{b_n\}]_{=} \iff \exists N,\ \forall n>N,\ a_n \le b_n
    \]
\end{definition}

\begin{definition}[实数的绝对值 Absolute Value of Real Numbers]
    设 $a\in\mathbb{R}$,定义其绝对值为
    \[
        |a| = 
        \begin{cases}
            a, & a \ge 0 \\
            -a, & a < 0
        \end{cases}
    \]
\end{definition}

\begin{definition}[实数的距离 Distance of Real Numbers]
    设 $a,b\in\mathbb{R}$,定义它们之间的距离为
    \[
        d(a,b) = |a - b|
    \]
\end{definition}
\vspace{1em}

\begin{proposition}[实数绝对值的性质]
    设 $a,b\in\mathbb{R}$,则有:
    \begin{enumerate}
        \item 非负性:$|a| \ge 0$,且 $|a| = 0 \iff a = 0$;
        \item 乘法:$|ab| = |a||b|$;
        \item 三角不等式:$|a+b| \le |a| + |b|$.
        \item 反三角不等式:$||a| - |b|| \le |a - b|$.
        \item 距离的非负性:$d(a,b) \ge 0$,且 $d(a,b) = 0 \iff a = b$;
        \item 距离的对称性:$d(a,b) = d(b,a)$;
        \item 距离的三角不等式:$d(a,c) \le d(a,b) + d(b,c)$.
    \end{enumerate}
\end{proposition}

\begin{note}
    二元组 $(\mathbb{R},\le)$ 构成一个全序集,实数绝对值运算同样满足非负性、三角不等式、反三角不等式等,二元组 $(\mathbb{R},d)$ 构成一个度量空间。
\end{note}

\vspace{1em}
\subsection{实数集的完备性}
实数相比有理数最大的区别是“连续”,准确的说,实数集是完备的。实数的完备性使得极限运算在实数集上是良定义的。
\vspace{1em}

\begin{definition}[实数序列 sequence of real numbers]
    函数 $f:\mathbb{N}\to\mathbb{R}$ 为实数序列,记为 $\{x_n\}$,其中 $x_n=f(n),\ n\ge m,\cdots$。
\end{definition}

\begin{definition}[实数柯西序列 Cauchy sequence of real numbers]
    设 $\{x_n\}$ 是实数序列。$\{x_n\}$ 是柯西序列,当且仅当,对于任意给定的实数 $\epsilon>0$,都存在一个正整数 $N$,使得当 $i,j>N$ 时,有
    \[
        d(x_i,x_j) = |x_i - x_j| \le \epsilon
    \]
\end{definition}

\begin{definition}[有界序列 bounded sequence of real numbers]
    设 $\{x_n\}$ 是实数序列。如果存在实数 $M>0$,使得对任意的 $n$ 都有 $|x_n|\le M$,则称 $\{x_n\}$ 是有界序列。
\end{definition}

\begin{theorem}
    实数柯西序列都是有界序列;有界序列不一定是柯西序列。
\end{theorem}

\begin{note}
    实数序列的定义同有理数序列,不过序列是由实数构成的。类似地可以定义实数柯西序列和有界序列。
    一个实数序列是柯西序列蕴含该序列是有界序列。
\end{note}
\vspace{1em}

\begin{definition}[收敛序列与序列极限 Convergent Sequence and Limit of Sequence]
    设 $\{x_n\}$ 是实数序列。$\{x_n\}$ 收敛于实数 $x$,记为 $\lim_{n\to\infty} x_n = x$,当且仅当,对于任意给定的实数 $\epsilon>0$,都存在一个正整数 $N$,使得当 $n>N$ 时,有
    \[
        d(x_n,x) = |x_n - x| \le \epsilon
    \]
    此时,称 $x$ 为序列 $\{x_n\}$ 的极限。
\end{definition}

\begin{proposition}
    收敛序列的极限是唯一。
\end{proposition}
\begin{proof}
    使用反证法,假设实数收敛序列 $\{x_n\}$ 收敛于极限 $x$ 和 $y$。
    那么设 $ \epsilon = \frac{|x-y|}{2} > 0 $,根据收敛序列的定义,存在正整数 $N_1$,使得当 $n>N_1$ 时,有 $|x_n - x| < \epsilon$;
    同理,存在正整数 $N_2$,使得当 $n>N_2$ 时,有 $|x_n - y| < \epsilon$;
    令 $N=\max(N_1,N_2)$,则当 $n>N$ 时,有
    \begin{align*}
        |x - y| &= |x - x_n + x_n - y| \\
        &\le |x - x_n| + |x_n - y| \\
        &< \epsilon + \epsilon = 2\epsilon = |x-y|
    \end{align*}
    这说明 $|x-y| < |x-y|$,矛盾,所以假设不成立,极限是唯一的。
\end{proof}
\vspace{1em}

下面不加证明的给出实数集完备性的两个等价定理。

\begin{theorem}[实数集的完备性 Completeness of Real Numbers]
    实数集具有完备性,当且仅当,所有的实数柯西序列都是收敛实数序列。
\end{theorem}

\begin{theorem}[柯西收敛准则 Cauchy Convergence Criterion]
    实数序列 $\{x_n\}$ 收敛的充分必要条件是:对于任给的实数 $\epsilon>0$,都存在正整数 $N$,使得当 $i,j>N$ 时,有 $d(x_i,x_j) = |x_i - x_j| \le \epsilon$。
\end{theorem}
% \begin{proof}
%     证明必要性。设实数序列 $\{x_n\}$ 收敛于实数 $x$,那么对于任意实数 $\varepsilon/2 > 0$,都存在正整数 $N$,使得当 $i,j>N$ 时,有:
%     \[
%         |x_i-x_j| = |x_i - x + x - x_j| \le |x_i - x| + |x_j - x| < \varepsilon/2 + \varepsilon/2 = \varepsilon
%     \]
%     证明充分性。设实数序列 $\{x_n\}$ 是柯西序列,那么 $\{x_n\}$ 是有界序列。定义尾部序列:
%     \[
%         \{E_k\} = \{x_n : n > k\},\ k=1,2,3,\cdots
%     \]
%     则 $\{E_k\}$ 非空且有界,根据确界存在定理,$\{E_k\}$ 有上确界 $a_k = \sup\{E_k\}$;同时,$\{E_k\}$ 有下确界 $b_k = \inf\{E_k\}$。
%     序列 $\{a_k\}$ 单调递增有上界,序列 $\{b_k\}$ 单调递减有下界。再根据确界存在定理,令:
%     \[
%         s = \sup \{a_k\},\ t = \inf \{b_k\}
%     \]
    
% \end{proof}
% \vspace{1em}

\begin{note}
    柯西收敛准则和实数完备性是两条条等价的定理,它们说明了实数集和有理数集的本质区别。
    稠密的有理数集对极限运算不封闭,一个有理数序柯西序列的极限可能不是有理数,
    例 \ref{ex:sqrt2-approx} 中的两个柯西序列在有理数集上没有极限,但在实数集上都收敛于 $\sqrt{2}$。
    完备性保证了有理数柯西列一定能收敛到一个实数,而不会收敛到一个不属于实数集的元素。
    换句话说,在实数集中,柯西序列与收敛序列是等价的,实数集对极限运算封闭。
\end{note}

% \begin{theorem}[确界存在定理 Existence of Supremum and Infimum]
%     设 $X\subseteq\mathbb{R}$,$X$ 非空且有上界,则 $X$ 有最小上界(即上确界);$X$ 非空且有下界,则 $X$ 有最大下界(即下确界)。
% \end{theorem}

% \begin{proof}
%     设 $X\subset \mathbb{R}$ 非空且有上界。则构造两个有理数集合:
%     \begin{align*}
%         A &= \{a\in\mathbb{Q}: \exists x\in X, x > a\} \\
%         B &= \{b\in\mathbb{Q}: \forall x\in X, x \le b\}
%     \end{align*}
%     其中,$A$ 表示所有不是 $X$ 上界的有理数,$B$ 表示所有是 $X$ 上界的有理数,因此,$A = \complement_{\mathbb{Q}}B$,且对任意 $a\in A$ 和 $b\in B$,都有 $a < b$。
%     称 $A$ 和 $B$ 构成一个对有理数集的\textbf{戴德金分割 Dedekind cut}。
%     序列 $\{a_n\}$ 和 $\{b_n\}$ 分别定义为:
%     \begin{align*}
%         a_n &= \sup\{a\in A: a \le b_{n-1}\},\ n\ge 1,\ a_0\in A \\
%         b_n &= \inf\{b\in B: b \ge a_n\},\ n\ge 1
%     \end{align*}
%     则 $\{a_n\}$ 和 $\{b_n\}$ 都是有理数柯西序列,且对任意 $n$,都有 $a_n \le b_n$。
%     设 $s = [\{a_n\}]_{=} = [\{b_n\}]_{=}$,则 $s$ 是实数集 $\mathbb{R}$ 中的一个实数。
%     对于任意 $x\in X$,因为 $b_n\in B$,所以存在 $N$,使得当 $n>N$ 时,有 $x \le b_n$,根据实数序的定义,$s \ge x$,
%     因此,$s$ 是 $X$ 的上界。再设 $t$ 是 $X$ 的任意上界,那么因为 $b_n\in B$,所以存在 $N$,使得当 $n>N$ 时,有 $t \ge b_n$,进而说明 $t \ge s$。
%     综上,$s$ 是 $X$ 的最小上界。\\
%     同理,可以证明 $X$ 非空且有下界,则 $X$ 有最大下界(即下确界)。
% \end{proof}

% \begin{note}
%     确界存在性是实数集完备性的重要定理。有理数集则不具备这一性质。例 例 \ref{ex:sqrt2-approx} 中的两个柯西序列在有理数集上界,比如 3、2、1.5 等都是序列的上界,但没有最小上界,因为 $\sqrt{2}\notin\mathbb{Q}$。
% \end{note}

\vspace{1em}

\subsection{实数的幂运算}

\begin{definition}[实数的自然数次幂]
    设 $x\in\mathbb{R}$ 是实数,$n\in\mathbb{N}$ 是自然数,$x$ 的 $n$ 次幂记为 $x^n$,递归地定义:
    \begin{enumerate}
        \item $x^0 = 1$
        \item $x^n = x \cdot x^{n-1},\ n\ge 1$
    \end{enumerate}
\end{definition}

\begin{definition}[实数的负整数次幂]
    设 $x\neq 0\in\mathbb{R}$ ,$n\in\mathbb{N}^+$ 是正整数,$x$ 的 $-n$ 次幂记为 $x^{-n}$,定义
    \[
        x^{-n} = \frac{1}{x^n}
    \]
\end{definition}
% \begin{theorem}[实数的整数次幂的性质]
%     设 $x,y\in\mathbb{R}$,$m,n\in\mathbb{Z}$,则
%     \begin{enumerate}
%         \item $x^m \cdot x^n = x^{m+n}$
%         \item $(x^m)^n = x^{mn}$
%         \item $(xy)^n = x^n y^n$
%         \item 若 $x\neq 0$,则 $x^m / x^n = x^{m-n}$
%     \end{enumerate}
% \end{theorem}
% \vspace{1em}

\begin{definition}[正实数的真分数次幂]
    设 $x>0\in\mathbb{R}$ 是正实数,$n\in\mathbb{N}^+$ 是正整数,$x$ 的 $\frac{1}{n}$ 次幂记为 $x^{\frac{1}{n}}$,定义
    \[
        x^{\frac{1}{n}} = \sup\{y\in\mathbb{R}: y\ge 0,\ y^n \le x\}
    \]
    $x^{\frac{1}{n}}$ 称为 $x$ 的\textbf{$n$ 次方根},也常记为 $\sqrt[n]{x}$。
\end{definition}

\begin{proposition}[n 次方根的存在性]
    设 $x>0\in\mathbb{R}$ 是正实数,$n\in\mathbb{N}^+$ 是正整数,则 $x$ 的 $n$ 次方根 $x^{\frac{1}{n}}$ 存在且唯一。
\end{proposition}

\begin{definition}[正实数的有理数次幂]
    设 $x>0\in\mathbb{R}$ 是正实数,$m\in\mathbb{Z}$ 是整数,$n\in\mathbb{N}^+$ 是正整数,$x$ 的 $\frac{m}{n}$ 次幂记为 $x^{\frac{m}{n}}$,定义
    \[
        x^{\frac{m}{n}} = (x^{\frac{1}{n}})^m
    \]
\end{definition}
\vspace{1em}

\begin{theorem}[正实数的有理数次幂的性质]
    设 $x,y>0\in\mathbb{R}$,$m,n\in\mathbb{N}^+$,$p,q\in\mathbb{Q}$ 则
    \begin{enumerate}
        \item $(x^{\frac{1}{n}})^n = x$
        \item $x^{\frac{1}{m}} = (x^{\frac{1}{n}})^{\frac{n}{m}}$
        \item $(xy)^{p} = x^{p} y^{p}$
        \item $(x^{p})^{q} = x^{pq}$
        \item $x^{p} \cdot x^{q} = x^{p+q}$
        \item $x^{p} / x^{q} = x^{p-q}$
        \item 如果 $y = x^{1/n}$,那么 $x = y^n$
        \item 若 $x\le y$,则 $x^{\frac{1}{n}} \le y^{\frac{1}{n}}$
    \end{enumerate}
\end{theorem}

\vspace{1em}
\begin{note}
    实数的有理数次幂要求底数为正实数,指数可以是任意有理数。
    在实数集中,负实数的有理数次幂没有定义,比如 $(-1)^{\frac{1}{2}}$ 是无意义的,也即不存在实数 $x$,使得 $x^2 = -1$。
    这说明实数集对开方运算不封闭,需要再次扩充数集,得到复数集。
\end{note}

\newpage