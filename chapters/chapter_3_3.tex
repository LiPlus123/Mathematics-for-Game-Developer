\section{实数}

从自然数到有理数的扩充都非常的自然,然而有理数并不能满足我们的所有需求。比如在上一节的末尾介绍有理数稠密性时提到,不存在有理数 $x$ 使得 $x^2=2$。为了满足这样的需求,我们需要对数系进行进一步的扩充,得到实数系。
\vspace{1em}

\begin{example}
    通过迭代法,我们可以找到一个有理数序列 $a_n$,使得 $(a_n)^2$ 趋近于 $2$:
    \begin{equation}
        a_{n+1} = \frac{1}{2}(a_n+\frac{2}{a_n}),\ n\ge 0
        \label{eq:sqrt2-approx}
    \end{equation}
    令 $a_0=1$,得到一个有理数序列:
    \begin{align*}
        a_1 &= \frac{1}{2}(1+2) = \frac{3}{2} = 1.5\\
        a_2 &= \frac{1}{2}(\frac{3}{2}+\frac{4}{3})=\frac{17}{12} = 1.41666666666\cdots\\
        a_3 &= \frac{1}{2}(\frac{17}{12}+\frac{24}{17})=\frac{577}{408}= 1.414215686274\cdots\\
        a_4 &= \frac{1}{2}(\frac{577}{408}+\frac{816}{577})=\frac{665857}{470832}=1.414213562374\cdots\\
        \vdots
    \end{align*}
    无限递归执行下去,会得到一个有理数的数列。随着 $n$ 的增大,$a_n$ 之间的间隔会越来越小。换一个首项,令 $a^{\prime}_0=1.4$,同样使用迭代式 \ref{eq:sqrt2-approx},得到另一个有理数数列:
    \begin{align*}
        a_1^{\prime} &= \frac{1}{2}(\frac{7}{5}+\frac{10}{7}) = \frac{99}{70} = 1.4142857142857\cdots\\
        a_2^{\prime} &= \frac{1}{2}(\frac{99}{70}+\frac{140}{99})=\frac{19601}{13860}=1.4142135642135642\cdots\\
        a_3^{\prime} &= \frac{1}{2}(\frac{19601}{13860}+\frac{27720}{19601})=\frac{768398401}{543339720}=1.41421356237309504\cdots\\
        a_4^{\prime} &=\frac{1}{2}(\frac{768398401}{543339720}+\frac{1086679440}{768398401})=\frac{1180872205318713601}{835002744095575440}=1.414213562373095048801\cdots\\
        \vdots
    \end{align*}
    无限递归执行下去,会得到另一个有理数的数列。随着 $ n $ 的增加,数列中项与项之间的间隔会也会越来越小。同时,随着序号 $ n $ 的增加,数列 $ a_n $ 和数列 $ a_n^{\prime} $ 中项的差 $ |a_n-a_n^{\prime}| $ 也会越来越小。
    \label{ex:sqrt2-approx}
\end{example}

\begin{note}
    我们知道,这两个序列最终都会收敛到无理数 $ \sqrt{2} $,但是 $ \sqrt{2}\notin\mathbb{Q} $。
    我们称数列 $ \{a_n\in\mathbb{Q}\} $ 和数列 $ \{a_n^{\prime}\in\mathbb{Q}\} $ 为\textbf{柯西序列},$ \sqrt{2} $ 为这两个\textbf{柯西序列的极限}。
    在有理数集中,柯西序列的极限不一定是有理数。因此我们需要对有理数集打上补丁,让这些序列的极限也在这个新集合中。
\end{note}
\vspace{1em}

\subsection{柯西序列与实数}

历史上,实数的定义有多种方式,一种是通过柯西序列的方式,另一种是戴德金分割,下面主要介绍柯西序列定义实数。
\vspace{1em}

\begin{definition}[有理数序列 sequence of rational numbers]
    函数 $f:\mathbb{N}\to\mathbb{Q}$ 为有理数序列,记为 $\{a_n\}$,其中 $a_n=f(n),\ n\ge m,\cdots$。
\end{definition}
\begin{note}
    有理数序列说明,对于每个大于等于 $m$ 的自然数,都有唯一一个确定的有理数与之对应。例如:
    \begin{enumerate}
        \item 等比数列:$a_n = a_1q^{n-1},\ a_1\neq 0,\ q\neq 0,\ n=1,2,3,\cdots$
        \item 等差数列:$a_n = a_1 + (n-1)d,\ n=1,2,3\cdots$
        \item 斐波那契数列:$a_1=1,\ a_2=1,\ a_n=a_{n-1}+a_{n-2},\ n\ge 3$
        \item 素数数列:$a_1=2,\ a_2=3,\ a_3=5,\ a_4=7,\ a_5=11,\ a_6=13,\cdots$
    \end{enumerate}
\end{note}
\vspace{1em}

\begin{definition}[柯西序列 Cauchy sequence]
    设 $\{a_n\}$ 是有理数序列。$\{a_n\}$ 是柯西序列,当且仅当,对于任意给定的有理数 $\epsilon>0$,都存在一个正整数 $N$,使得当 $i,j>N$ 时,有
    \[
        |a_i - a_j| \le \epsilon
    \]
    柯西序列也称为\textbf{基本序列}。
\end{definition}
\begin{note}
    柯西序列是一种特殊的有理数序列,它要求随着序号 $n$ 的增大,序列中项与项之间的间隔越来越小。
    例如,数列 $a_n = \frac{1}{n},\ n\ge 1$ 和 $a_{n+1} = \frac{1}{2}(a_n+\frac{2}{a_n}),\ n\ge 0$ 都是柯西序列,
\end{note}
\vspace{1em}

\begin{definition}[有界序列 bounded sequence]
    设 $\{a_n\}$ 是有理数序列。如果存在有理数 $M>0$,使得对任意的 $n$ 都有 $|a_n|\le M$,则称 $\{a_n\}$ 是有界序列。
\end{definition}

\begin{theorem}
    柯西序列都是有界序列;有界序列不一定是柯西序列。
\end{theorem}

\begin{note}
    一个序列是柯西序列蕴含该序列是有界序列,但有界序列不一定是柯西序列,比如,$a_n=(-1)^n,\ n\ge 1$ 是有界序列,但不是柯西序列。
\end{note}
\vspace{1em}

\begin{definition}[实数 Real Number] 
    设 $S$ 是所有有理柯西序列的集合
    \begin{enumerate}
        \item 等于关系:对任意有理柯西序列 $\{a_n\},\{b_n\}\in S$,$\{a_n\} = \{b_n\}$ 当且仅当,对于任意有理数 $\epsilon>0$,都存在一个正整数 $N$,使得当 $n>N$ 时,有
        \[
            |a_n - b_n| \le \epsilon
        \] 
        \item 等价类:任意 $\{a_n\}\in S$ 的等价类为
        \[
            [\{a_n\}]_{=} = \{\{b_n\}\in S : \{a_n\} = \{b_n\}\}
        \]
    \end{enumerate}
    称 $S$ 关于 $=$ 关系的商集 $S/_{=}= \{[\{a_n\}]_{=} : \{a_n\}\in S\}$ 为实数集,记为 $\mathbb{R}$。
    实数集中的元素称为实数。
\end{definition}
\begin{note}
    实数是柯西序列关于等于关系的等价类,比如,数列 $a_n = \frac{1}{n},\ n\ge 1$ 和数列 $b_n = \frac{1}{n^2},\ n\ge 1$ 是不同的柯西序列,但它们满足等于关系,属于同一个等价类 $[\{a_n\}]_{=}=[\{b_n\}]_{=}$,表示同一个实数 $0$。
    再比如,在例 \ref{ex:sqrt2-approx} 中,数列 $a_n$ 和数列 $a_n^{\prime}$ 是不同的柯西序列,但它们满足等于关系,属于同一个等价类 $[\{a_n\}]_{=}=[\{a_n^{\prime}\}]_{=}$,表示同一个实数 $\sqrt{2}$。
    全体有理数可以嵌入到实数集中,即 $\forall q\in\mathbb{Q}$,有 $q = [\{q,q,q,\cdots\}]_{=}\in\mathbb{R}$。
    例如,$a_0=1,a_2=1,a_3=1,\cdots$ 和 $b_0=0,b_2=0.9,b_3=0.999,\cdots$ 都表示同一个实数 $1$。
\end{note}
\vspace{1em}

\begin{lemma}
    设 $\{a_n\},\{b_n\}$ 是柯西序列,则 $\{a_n+b_n\}$ 也是柯西序列。
\end{lemma}

\begin{definition}[实数加法 Addition of Real Numbers]
    设 $[\{a_n\}]_{=},[\{b_n\}]_{=}\in\mathbb{R}$,定义加法为
    \[
        [\{a_n\}]_{=} + [\{b_n\}]_{=} = [\{a_n+b_n\}]_{=}
    \]
\end{definition}

\begin{proposition}
    实数加法的定义是良定义(Well-defined)的,即不依赖于代表元的选取,而且是封闭的。
\end{proposition}

\begin{lemma}
    设 $\{a_n\},\{b_n\}$ 是柯西序列,则 $\{a_nb_n\}$ 也是柯西序列。
\end{lemma}

\begin{definition}[实数乘法 Multiplication of Real Numbers]
    设 $[\{a_n\}]_{=},[\{b_n\}]_{=}\in\mathbb{R}$,定义乘法为
    \[
        [\{a_n\}]_{=} \cdot [\{b_n\}]_{=} = [\{a_nb_n\}]_{=}
    \]
\end{definition}

\begin{proposition}
    实数乘法的定义是良定义(Well-defined)的,即不依赖于代表元的选取,而且是封闭的。
\end{proposition}


\begin{lemma}
    设 $\{a_n\}$ 是柯西序列,则 $\{-a_n\}$ 也是柯西序列。
\end{lemma}


\begin{definition}[实数加法逆元]
    设 $[\{a_n\}]_{=}\in\mathbb{R}$,定义其加法逆元为
    \[
        -[\{a_n\}]_{=} = [\{-a_n\}]_{=}
    \]
    使得,$[\{a_n\}]_{=} + -[\{a_n\}]_{=} = [\{0\}]_{=}$。
\end{definition}

\begin{proposition}
    实数的加法逆元是良定义的,任何一个实数都有唯一的加法逆元。
\end{proposition}

\begin{lemma}
    设 $\{a_n\}$ 是柯西序列,且 $\{a_n\}\neq \{0\}$,则 $\{a_n^{-1}\}$ 也是柯西序列。
\end{lemma}

\begin{definition}[实数乘法逆元]
    设 $[\{a_n\}]_{=}\in\mathbb{R}$,且 $[\{a_n\}]_{=}\neq [\{0\}]_{=}$,定义其乘法逆元为
    \[
        [\{a_n\}]_{=}^{-1} = [\{a_n^{-1}\}]_{=}
    \]
    使得,$[\{a_n\}]_{=} \cdot [\{a_n\}]_{=}^{-1} = [\{1\}]_{=}$。
\end{definition}

\begin{proposition}
    乘法逆元的定义是良定义的,任意非零实数都有唯一的乘法逆元。
\end{proposition}

\begin{proof}
    
\end{proof}

\begin{definition}[实数的减法和除法]
    设 $[\{a_n\}]_{=},[\{b_n\}]_{=}\in\mathbb{R}$,定义减法为
    \[
        [\{a_n\}]_{=} - [\{b_n\}]_{=} = [\{a_n + (-b_n)\}]_{=}
    \]
    定义除法为,
    \[
        [\{a_n\}]_{=} / [\{b_n\}]_{=} = [\{a_n \cdot b_n^{-1}\}]_{=},\ [\{b_n\}]_{=}\neq [\{0\}]_{=}
    \]
\end{definition}
\vspace{1em}

\begin{definition}[实数的序 Order of Real Numbers]
    设 $[\{a_n\}]_{=},[\{b_n\}]_{=}\in\mathbb{R}$,定义序为
    \[
        [\{a_n\}]_{=} \le [\{b_n\}]_{=} \iff \exists N,\ \forall n>N,\ a_n \le b_n
    \]
\end{definition}

\begin{definition}[实数的绝对值 Absolute Value of Real Numbers]
    设 $[\{a_n\}]_{=}\in\mathbb{R}$,定义其绝对值为
    \[
        |[\{a_n\}]_{=}| = 
        \begin{cases}
            [\{a_n\}]_{=}, & [\{a_n\}]_{=} \ge [\{0\}]_{=} \\
            -[\{a_n\}]_{=}, & [\{a_n\}]_{=} < [\{0\}]_{=}
        \end{cases}
    \]
\end{definition}

\begin{definition}[实数的距离 Distance of Real Numbers]
    设 $[\{a_n\}]_{=},[\{b_n\}]_{=}\in\mathbb{R}$,定义它们之间的距离为
    \[
        d([\{a_n\}]_{=},[\{b_n\}]_{=}) = |[\{a_n\}]_{=} - [\{b_n\}]_{=}|
    \]
\end{definition}

\vspace{1em}
\subsection{实数集的完备性}
实数相比有理数最大的区别是“连续”,准确的说,实数集是完备的。实数的完备性使得极限运算在实数集上是良定义的。
\vspace{1em}

\begin{definition}[实数序列 sequence of real numbers]
    函数 $f:\mathbb{N}\to\mathbb{R}$ 为实数序列,记为 $\{x_n\}$,其中 $x_n=f(n),\ n\ge m,\cdots$。
\end{definition}

\begin{definition}[实数柯西序列 Cauchy sequence of real numbers]
    设 $\{x_n\}$ 是实数序列。$\{x_n\}$ 是柯西序列,当且仅当,对于任意给定的实数 $\epsilon>0$,都存在一个正整数 $N$,使得当 $i,j>N$ 时,有
    \[
        d(x_i,x_j) = |x_i - x_j| \le \epsilon
    \]
\end{definition}

\begin{definition}[有界序列 bounded sequence of real numbers]
    设 $\{x_n\}$ 是实数序列。如果存在实数 $M>0$,使得对任意的 $n$ 都有 $|x_n|\le M$,则称 $\{x_n\}$ 是有界序列。
\end{definition}

\begin{theorem}
    实数柯西序列都是有界序列;有界序列不一定是柯西序列。
\end{theorem}

\begin{note}
    实数序列的定义同有理数序列,不过序列是由实数构成的。类似地可以定义实数柯西序列和有界序列。
    一个实数序列是柯西序列蕴含该序列是有界序列。
\end{note}
\vspace{1em}

\begin{definition}[收敛序列与序列极限 Convergent Sequence and Limit of Sequence]
    设 $\{x_n\}$ 是实数序列。$\{x_n\}$ 收敛于实数 $x$,记为 $\lim_{n\to\infty} x_n = x$,当且仅当,对于任意给定的实数 $\epsilon>0$,都存在一个正整数 $N$,使得当 $n>N$ 时,有
    \[
        d(x_n,x) = |x_n - x| \le \epsilon
    \]
    此时,称 $x$ 为序列 $\{x_n\}$ 的极限。
\end{definition}

\begin{lemma}
    收敛序列的极限是唯一。
\end{lemma}
\begin{proof}
    
\end{proof}

\begin{theorem}[确界存在定理 Existence of Supremum and Infimum]
    设 $A\subseteq\mathbb{R}$,$A$ 非空且有上界,则 $A$ 有最小上界(即上确界);$A$ 非空且有下界,则 $A$ 有最大下界(即下确界)。
\end{theorem}

\begin{proof}
    
\end{proof}

\begin{note}
    确界存在性是实数集完备性的重要定理。有理数集则不具备这一性质。例 例 \ref{ex:sqrt2-approx} 中的两个柯西序列在有理数集上界,比如 3、2、1.5 等都是序列的上界,但没有最小上界,因为 $\sqrt{2}\notin\mathbb{Q}$。
\end{note}


\begin{theorem}[柯西收敛准则 Cauchy Convergence Criterion]
    实数序列 $\{x_n\}$ 收敛的充分必要条件是:对于任给的实数 $\epsilon>0$,都存在正整数 $N$,使得当 $i,j>N$ 时,有 $d(x_i,x_j) = |x_i - x_j| \le \epsilon$。
\end{theorem}
\begin{proof}
    
\end{proof}


\begin{theorem}[实数集的完备性 Completeness of Real Numbers]
    实数集具有完备性,当且仅当,所有的实数柯西序列都是收敛实数序列。也即,在实数集中,柯西序列与收敛序列是等价的,实数集对极限运算封闭。
\end{theorem}

\begin{note}
    确界存在定理、柯西收敛准则和实数完备性是三条条等价的定理,它们说明了实数集和有理数集的本质区别——有理数集是稠密的,而实数集是完备的。
    稠密的有理数集对极限运算不封闭,一个有理数序柯西序列的极限可能不是有理数,
    例 \ref{ex:sqrt2-approx} 中的两个柯西序列在有理数集上没有极限,但在实数集上都收敛于 $\sqrt{2}$。
    完备性保证了有理数柯西列一定能收敛到一个实数,而不会收敛到一个不属于实数集的元素。换句话说,完备性保证了实数集对极限运算封闭。
\end{note}

\vspace{1em}



\subsection{实数的幂运算}



\newpage