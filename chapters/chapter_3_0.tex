\chapter{数系的扩充}

在第一章中,利用无穷公理定义了自然数 \ref{def:natural_numbers},根据集合的包含关系,定义了自然数的序 \ref{def:natural_order},$(\mathbb{N},\le)$ 是一个全序集。
然后利用替代公理模式,定义了自然数的加法 \ref{def:natural_addition} 和乘法 \ref{def:natural_multiplication},$(\mathbb{N},+)$ 和 $(\mathbb{N},\times)$ 都是是一个交换幺半群;
乘法对加法分配,$(\mathbb{N},+,\times)$ 是一个含幺交换半环。
在本章中,将介绍如何从自然数出发,扩充出整数、有理数、实数等数系。数系的每一次扩充,都希望它能使更多的运算封闭,并且已有的运算性质能“向下兼容”。

\begin{definition}[等于关系 Equality Relation]
    $ R $ 为集合 $ X $ 上的一个等价关系。称 $ R $ 为 $ X $ 上的等于关系,当且仅当,$ R $ 满足:
    \begin{enumerate}
        \item 替代性:对于任意 $(x,y)\in R$,对于一切函数 $f:X\to X$,均有 $(f(x),f(y))\in R$
    \end{enumerate}
    记 $(x,y)\in R$ 为 $ x=y $。
\end{definition}

\begin{note}
    等于关系是更严格的等价关系。满足替代性说明,集合中等价的两元素,经过任意函数映射后的像也是等价的。在定义新的数系的时候,不仅需要验证新的运算是否对集合封闭,还需要给出一个等于关系。
\end{note}

\vspace{1em}