\section{拓扑结构}

\begin{definition}[拓扑空间 Topological Space]
    设$ X $是一个集合,$ \mathcal{T}(X) \subseteq \mathcal{P}(X) $ 是 $ X $ 的子集族,二元组 $ (X,\mathcal{T}(X)) $ 称为拓扑空间,当且仅当,$ \mathcal{T}(X) $ 满足:
    \begin{enumerate}
        \item 空集与全集:$ \varnothing \in \mathcal{T}(X) $ 且 $ X \in \mathcal{T}(X) $;
        \item 对任意并封闭:任意 $ \{U_i:i\in I\} \subseteq \mathcal{T}(X) $,有 $ \bigcup_{i\in I} U_i \in \mathcal{T}(X) $;
        \item 对有限交封闭:有限个 $ U_1,U_2,\ldots,U_n\in \mathcal{T}(X) $,有 $ U_1\cap U_2\cap \cdots \cap U_n \in \mathcal{T}(X) $。
    \end{enumerate}
    子集族 $ \mathcal{T}(X) $ 称为 $ X $ 上的\textbf{拓扑 Topology},$ \mathcal{T}(X) $ 中的元素称为\textbf{开集 Open Set}。
\end{definition}

\begin{note}
    根据拓扑空间的的定义,幂集 $ \mathcal{P}(X) $ 和 $ \{\varnothing,X\} $ 都是 $ X $ 上的拓扑,分别称为\textbf{离散拓扑 Discrete Topology}和\textbf{平凡拓扑 Trivial Topology}。
    拓扑空间的概念是对几何空间的抽象,拓扑空间中的开集是对几何空间中开区间、开球等概念的抽象。
    拓扑空间中的开集可以用来定义点的邻域、连续函数等概念。
\end{note}

\vspace{1em}

\begin{definition}[闭集 Closed Set]
    设 $ (X,\mathcal{T}(X)) $ 为拓扑空间,称 $ A\subseteq X $ 为闭集,当且仅当, $ \complement_X A \in \mathcal{T}(X) $。
\end{definition}

\begin{proposition}[闭集的性质]
    设 $ (X,\mathcal{T}(X)) $ 为拓扑空间,则有:
    \begin{enumerate}
        \item 空集与全集:$ \varnothing $ 与 $ X $ 是闭集;
        \item 对任意交封闭:$ \bigcap_{i\in I} A_i $ 是闭集,其中 $ \{A_i:i\in I\}$ 是闭集;
        \item 对有限并封闭:$ A_1\cup A_2\cup \cdots \cup A_n $ 是闭集,其中 $ A_1,A_2,\ldots,A_n $ 是闭集。
    \end{enumerate}
\end{proposition}

\begin{note}
    闭集是对几何空间中闭区间、闭球等概念的抽象。闭集与开集的关系是互补的,即一个集合是闭集,当且仅当它的补集是开集。
    拓扑空间中的闭集可以用来定义点的极限、闭包等概念。
\end{note}

\vspace{1em}

\subsection{连续函数}

\subsection{度量空间}


\newpage