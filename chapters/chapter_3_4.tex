\section{复数}
复数的引入源自解三次代数方程的负数平方根问题 。在实数集中,一元二次次方程 $x^2 + 1 = 0$ 无解。
如果,令 $i^2=-1$,称为虚数单位(Imaginary Unit),定义一个全新的数集——复数集,那么任何一个代数方程在复数集中都有解,并且解的个数恰好等于方程的次数,该结论称为代数基本定理。

\begin{theorem}[代数基本定理 Fundamental Theorem of Algebra]
    每一个次数为 $n$ 的非零单变量复系数多项式 $P(z)=a_nz^n + a_{n-1}z^{n-1} + \cdots + a_1z + a_0,\ a_n\neq 0$ 在复数域上恰有 $n$ 个根(重根按重数计算)。
\end{theorem}

\vspace{1em}

\subsection{复数的定义与表示}

\begin{definition}[复数 Complex Number] 有序对 $(a,b)\in\mathbb{R}^2$ 记为 $a+bi$,其中,
    \begin{enumerate}
        \item 约定:$i^2 = -1$;
        \item 等于关系:$a+bi = c+di \Leftrightarrow a=c \land b=d$;
        \item 等价类:$[a+bi]_{=} = \{c+di\in\mathbb{C} : a+bi = c+di\}$;
    \end{enumerate}
    称 $\mathbb{R}^2$ 关于 $=$ 关系的商集 $\mathbb{R}^2/= \{[a+bi]_{=} : a+bi\in\mathbb{R}^2\}$ 为复数集,记为 $\mathbb{C}$。复数集中的元素称为复数。
\end{definition}

\begin{note}
    复数同样是一个等价类,但集合中只有一个有序对 $(a,b)$,所以可以直接用 $a+bi$ 来表示复数。
    在复数中,实数 $a$ 称为复数的实部(Real Part),记为 $\Re(a+bi)=a$;实数 $b$ 称为复数的虚部(Imaginary Part),记为 $\Im(a+bi)=b$。
    全体实数可以嵌入到复数集中,即 $\forall a\in\mathbb{R}$,有 $a =a+0i \in \mathbb{C}$。
\end{note}

\vspace{1em}



\subsection{复数集的完备性}

\newpage