\section{有理数}
\begin{definition}[有理数 Rational Number]
    有序对 $(a,b)\in\mathbb{Z}^2$ 记为 $a/b$,其中,
    \begin{enumerate}
        \item 约定 $b\neq 0$。
        \item 等于关系:$a/b=c/d \iff ad=bc$
        \item 等价类:任意 $a/b\in\mathbb{Z}^2$ 的等价类为
        \[
            [a/b]_{=}=\{(c,d)\in\mathbb{Z}^2:ad=bc\}
        \]
    \end{enumerate}
    称 $\mathbb{Z}^2$ 关于 $=$ 关系的商集 $\mathbb{Z}^2/_{=}= \{[a/b]_{=} : a/b \in \mathbb{Z}^2\}$ 为有理数集,记为 $\mathbb{Q}$。有理数集中的元素称为有理数。
\end{definition}

\begin{note}
    在 $a/b$ 中,$/$ 不是除号,而是一个符号,此时还没有定义除法,仅表示这是一个有序对 $(a,b)$。
    有理数也不是有序对,而是有序对关于等于关系的等价类,比如,$1/2$ 和 $2/4$ 是不同的有序对,但它们属于同一个等价类 $[1/2]_{=}=[2/4]_{=}$,表示同一个有理数。
    全体整数可以嵌入到有理数集中,即 $\forall n\in\mathbb{Z}$,有 $n = [n/1]_{=}\in\mathbb{Q}$。
    例如,$0 = [0/1]_{=}$,$1 = [1/1]_{=}$,$2 = [2/1]_{=}$,$3 = [3/1]_{=}$,$\cdots$。$\complement_{\mathbb{Q}}\mathbb{Z}$ 中的元素称为真分数。
\end{note}

\begin{note}
    有序对中要求第二个元素不为零,即 $b\neq 0$,这是因为如果 $b=0$,则等于关系将变得不合理,比如,$1/0=2/0$ 将导致 $1\cdot0=2\cdot0$,即 $0=0$,
    这对任何 $a/0$ 和 $c/0$ 都成立,违背了等于关系的自反性,从而使得所有形如 $a/0$ 的有序对都属于同一个等价类,我们不希望这样的情况出现。
\end{note}
\vspace{1em}

\begin{definition}[有理数的加法 Addition of Rational Numbers]
    设 $[a/b]_{=},[c/d]_{=}\in\mathbb{Q}$,定义加法为
    \[
        [a/b]_{=} + [c/d]_{=} = [(ad+bc)/(bd)]_{=}
    \]
\end{definition}

\begin{definition}[有理数的乘法 Multiplication of Rational Numbers]
    设 $[a/b]_{=},[c/d]_{=}\in\mathbb{Q}$,定义乘法为
    \[
        [a/b]_{=} \cdot [c/d]_{=} = [(ac)/(bd)]_{=}
    \]
\end{definition}

\begin{definition}[有理数的加法逆元 Additive Inverse of Rational Numbers]
    设 $[a/b]_{=}\in\mathbb{Q}$,定义其加法逆元为
    \[
        -[a/b]_{=} = [-a/b]_{=}
    \]
    使得,$[a/b]_{=} + -[a/b]_{=} = [0/b^2]_{=}$。
\end{definition}

\begin{definition}[有理数的减法 Subtraction of Rational Numbers]
    设 $[a/b]_{=},[c/d]_{=}\in\mathbb{Q}$,定义减法为
    \[
        [a/b]_{=} - [c/d]_{=} = [a/b]_{=} + (-[c/d]_{=}) = [(ad-bc)/(bd)]_{=}
    \]
\end{definition}

\vspace{1em}

\begin{note}
    有理数的加法、减法和乘法都是良定义(Well-defined)的,即不依赖于代表元的选取,而且是封闭的。证明略。整数的加法和乘法在有理数集中保持不变。
    例如,$[2/1]_{=} + [3/1]_{=} = [(2\cdot1+3\cdot1)/(1\cdot1)]_{=} = [5/1]_{=}$,即 $2+3=5$。
\end{note}

\vspace{1em}

\begin{definition}[有理数的乘法逆元 Multiplicative Inverse of Rational Numbers]
    设 $[a/b]_{=}\in\mathbb{Q}$,且 $a\neq 0$,定义其乘法逆元为
    \[
        [a/b]_{=}^{-1} = [b/a]_{=}
    \]
    使得,$[a/b]_{=} \cdot [a/b]_{=}^{-1} = [ab/ab]_{=} = [1/1]_{=}$。
\end{definition}

\begin{definition}[有理数的除法 Division of Rational Numbers]
    设 $[a/b]_{=},[c/d]_{=}\in\mathbb{Q}$,且 $c\neq [0/1]_{=}$,定义除法为
    \[
        [a/b]_{=} / [c/d]_{=} = [a/b]_{=} \cdot [c/d]_{=}^{-1} = [(ad)/(bc)]_{=}
    \]
\end{definition}

\vspace{1em}

\begin{definition}[有理数的整数次幂 Integer Power of Rational Numbers]
    设 $[a/b]_{=}\in\mathbb{Q}$,$n\in\mathbb{Z}$,定义 $[a/b]_{=}$ 的整数次幂为
    \[
        ([a/b]_{=})^n = 
        \begin{cases}
            [1/1]_{=}, & n=0 \\
            [(\prod_{i=1}^{n} a)/(\prod_{i=1}^{n}b)]_{=}, & n>0 \\
            [(\prod_{i=1}^{-n} b)/(\prod_{i=1}^{-n}a)]_{=}, & n<0, a\neq [0/1]_{=}
        \end{cases}
    \]
\end{definition}

\begin{theorem}[有理数的整数次幂的性质 Properties of Integer Power of Rational Numbers]
    设 $x,y\in\mathbb{Q}$,$m,n\in\mathbb{Z}$,则
    \begin{enumerate}
        \item $x^m \cdot x^n = x^{m+n}$
        \item $(x^m)^n = x^{mn}$
        \item 若 $x\neq [0/1]_{=}$,则 $x^m / x^n = x^{m-n}$
        \item $(xy)^n = x^n y^n$
    \end{enumerate}
\end{theorem}

\vspace{1em}

\begin{note}
    可以证明,任何一个非零有理数都有唯一的乘法逆元。因此,有理数的除法也是良定义的,即不依赖于代表元的选取,而且是封闭的。
    整数集中,不存在乘法逆元,因此也不存在除法。相比整数集,有理数集扩充了乘法逆元,使得加法、减法、乘法和除法都能良定义。
\end{note}

\vspace{1em}

\begin{definition}[有理数的序关系 Order Relation of Rational Numbers]
    设 $[a/b]_{=},[c/d]_{=}\in\mathbb{Q}$,定义序关系为
    \[
        [a/b]_{=} \leq [c/d]_{=} \iff 
        \begin{cases}
            ad \leq bc, & bd > 0 \\
            ad \geq bc, & bd < 0
        \end{cases}
    \]
\end{definition}

\begin{definition}[有理数的绝对值 Absolute Value of Rational Numbers]
    设 $[a/b]_{=}\in\mathbb{Q}$,定义其绝对值为
    \[
        |[a/b]_{=}| = 
        \begin{cases}
            [a/b]_{=}, & a/b \geq [0/1]_{=} \\
            -[a/b]_{=}, & a/b < [0/1]_{=}
        \end{cases}
    \]
\end{definition}

\begin{definition}[有理数的距离 Distance of Rational Numbers]
    设 $[a/b]_{=},[c/d]_{=}\in\mathbb{Q}$,定义它们之间的距离为
    \[
        d([a/b]_{=},[c/d]_{=}) = |[a/b]_{=} - [c/d]_{=}|
    \]
\end{definition}

\vspace{1em}

\begin{theorem}[有理数绝对值的性质]
    设 $x,y\in\mathbb{Q}$,则
    \begin{enumerate}
        \item 非负性:$|x|\geq [0/1]_{=}$,且 $|x|=[0/1]_{=} \iff x=[0/1]_{=}$
        \item 乘积的绝对值:$|xy| = |x||y|$
        \item 幂的绝对值:$|x^n| = |x|^n,\ n\in\mathbb{Z}$
        \item 商的绝对值:$|x/y| = |x|/|y|,\ y\neq [0/1]_{=}$
        \item 三角不等式:$|x+y| \leq |x| + |y|$
        \item 反三角不等式:$||x| - |y|| \leq |x - y|$
        \item 距离的非负性:$d(x,y) \geq [0/1]_{=}$,且 $d(x,y)=[0/1]_{=} \iff x=y$
        \item 距离的三角不等式:$d(x,z) \leq d(x,y) + d(y,z)$
        \item 距离的对称性:$d(x,y) = d(y,x)$
    \end{enumerate}
\end{theorem}

\begin{note}
    三元组 $(\mathbb{Q},+,\cdot)$ 构成一个有理数域,二元组 $(\mathbb{Q},\leq)$ 是一个全序集,二元组 $(\mathbb{Q},d)$ 是一个度量空间。
\end{note}

\vspace{1em}

\begin{proposition}[有理数的稠密性 Density of Rational Numbers]
    设 $x,y\in\mathbb{Q}$,且 $x<y$,则存在 $z\in\mathbb{Q}$,使得 $x<z<y$。例如,取 $z=(x+y)/2$。
\end{proposition}

\begin{note}
    有理数的稠密性说明任意两个有理数之间,都有无数个有理数。尽管有理数是稠密的,但在有理数之间依然存在无限多的“空隙”,比如, 在 $1$ 和 $2$ 之间,不存在有理数 $x$ 使得 $x^2=2$,因此有理数集存在“空隙”,是“不连续的”。
\end{note}




\newpage