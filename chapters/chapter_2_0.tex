\chapter{关系与结构}

任意两事物之间存在某种关系(Relation),比如数之间的大小关系、相等关系,相似关系、正交关系、共线关系等。这些关系可以用关系谓词来叙述,
比如 a 大于 b、点 A 与点 B 共线、整数 x 整除整数 y 等。在没有定义元素之间的关系前,集合只是一堆元素的“堆砌”,
我们只能讨论集合与集合的包含关系,集合与元素的从属关系。当定义了元素之间的关系时,集合就会表现出某些性质,就称这个集合具备了某种结构(Structure)。
\vspace{1em}

受结构主义(Structuralism)思潮的影响,在二十世纪初,布尔巴基学派(Bourbaki)的数学家主张在集合论的基础上,通过元素之间的关系定义数学对象。集合限定了讨论对象的范围;关系是集合的笛卡尔积,通过分类公理模式筛选出来的子集;集合与关系一起构成了数学对象的结构。布尔巴基学派提出三种基本数学结构:
\begin{enumerate}
    \item 代数结构 Algebraic Structure:群、环、域、向量空间等,研究运算与代数性质,作为代数学的基础
    \item 序结构 Order Structure:偏序集、全序集等,研究元素之间的顺序关系,作为分析学的基础
    \item 拓扑结构 Topological Structure:拓扑空间、度量空间等,研究元素之间的邻近关系,作为几何学的基础
\end{enumerate}
\newpage