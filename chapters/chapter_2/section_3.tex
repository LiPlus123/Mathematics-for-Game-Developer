\section{等价关系}

\begin{definition}[等价关系 Equivalence Relation]
    设 $ X $ 为非空集合,$ R\subseteq X\times X $ 是 $X$ 上的二元关系。称 $ R $ 为 $ X $ 上的等价关系,当且仅当,对任意 $ x,y,z\in X $,$ R $ 满足:
    \begin{enumerate}
        \item 自反性 Reflexivity:$ (x,x)\in R $
        \item 对称性 Symmetry:若 $ (x,y)\in R $,则 $ (y,x)\in R $
        \item 传递性 Transitivity:若 $ (x,y)\in R $ 且 $ (y,z)\in R $,则 $ (x,z)\in R $
    \end{enumerate}
    对于任意 $ x,y\in X $,若 $ (x,y)\in R $,则称 $ x $ 与 $ y $ 关于关系 $ R $ 等价,记为 $ x\sim y $。可以根据等价关系,将集合划分为若干个互不相交的子集,每个子集称为一个等价类。
\end{definition}

\begin{note}
    等价关系是一种特殊的二元关系,需要满足自反性、对称性和传递性。
\end{note}
\vspace{1em}

\begin{example}
    定义在有限集 $X=\{1,2,\cdots,8\}$ 上的关系 $R$:
    \[
        R=\{( x,y ) :x,y\in X \wedge x\equiv y(\mathrm{mod}\ 3)\}
    \]
    $ x\equiv y(\mathrm{mod}\ 3) $ 称为 $ x,y $ 模 3 相等,不难验证该关系是自反的、对称的和传递的。其中 1、4、7 模 3 等于 1;2、5、8 模 3 等于 2;3、6 模 3 等于 0
    \label{ex:mod3_equivalence}
\end{example}
\vspace{1em}

\begin{definition}[等价类 Equivalence Class]
    设 $ R $ 为集合 $ X $ 上的等价关系。对于任意 $ x\in X $,定义 $ x $ 的等价类为
    \[
        [x]_R = \{y\in X : (x,y)\in R\}
    \]
    或简记为
    \[
        [x] = \{y\in X : y\sim x\}
    \]
    称 $ [x] $ 为包含元素 $ x $ 的等价类。
    
\end{definition}
\begin{note}
    $ x $ 的等价类是$ A $中所有与 $ x $ 等价的元素构成的集合。在例 \ref{ex:mod3_equivalence} 中,1 的等价类为 $ [1]=\{1,4,7\} $;2 的等价类为 $ [2]=\{2,5,8\} $;3 的等价类为 $ [3]=\{3,6\} $。这三个等价类,互相没有交集,用关系图表示为:
    \begin{center}
        \begin{tikzpicture}[
                >=Stealth,                 % 箭头样式
                node/.style = {draw, circle, minimum size=8mm}, % 节点样式
                every edge/.style={draw, -{Stealth}}            % 边为箭头
            ]

            % 等价类 {1,4,7}
            \begin{scope}[xshift=0cm]
                \node[node] (1) at (0,0) {1};
                \node[node] (4) at (1.2,2.0) {4};
                \node[node] (7) at (2.4,0) {7};
                % 自反性(自环)
                \draw[->] (1) to [loop above] (1);
                \draw[->] (4) to [loop above] (4);
                \draw[->] (7) to [loop above] (7);
                % 对称性(双向箭头)
                \draw[<->] (1) -- (4);
                \draw[<->] (4) -- (7);
                \draw[<->] (1) -- (7);
            \end{scope}

            % 等价类 {2,5,8}
            \begin{scope}[xshift=5.0cm]
                \node[node] (2) at (0,0) {2};
                \node[node] (5) at (1.2,2.0) {5};
                \node[node] (8) at (2.4,0) {8};
                % 自反性(自环)
                \draw[->] (2) to [loop above] (2);
                \draw[->] (5) to [loop above] (5);
                \draw[->] (8) to [loop above] (8);
                % 对称性(双向箭头)
                \draw[<->] (2) -- (5);
                \draw[<->] (5) -- (8);
                \draw[<->] (2) -- (8);
            \end{scope}

            % 等价类 {3,6}
            \begin{scope}[xshift=10.0cm]
                \node[node] (3) at (0,1.0) {3};
                \node[node] (6) at (2.4,1.0) {6};
                % 自反性(自环)
                \draw[->] (3) to [loop above] (3);
                \draw[->] (6) to [loop above] (6);
                % 对称性(双向箭头)
                \draw[<->] (3) -- (6);
            \end{scope}
            
        \end{tikzpicture}
    \end{center}
\end{note}
\vspace{1em}

\begin{definition}[商集 Quotient Set]
    设 $ R $ 为集合 $ X $ 上的等价关系。$ X $ 关于 $ R $ 的商集定义为
    \[
        X/R = \{[x] : x\in X\}
    \]
    称 $ X/R $ 为集合 $ X $ 关于等价关系 $ R $ 的商集。
    
\end{definition}
\begin{note}
    在例 \ref{ex:mod3_equivalence} 中,商集为
    \[
        X/R = \{[1], [2], [3]\} = \{\{1,4,7\}, \{2,5,8\}, \{3,6\}\}
    \]
\end{note}
\vspace{1em}

\subsection{划分}
\begin{definition}[划分 Partition]
    设 $ X $ 为非空集合,$ \mathcal{P} $ 是 $ X $ 的幂集。子集族 $\mathcal{S}\subseteq \mathcal{P}$ 称为 $ X $ 的一个划分,当且仅当,$ \mathcal{S} $ 满足:
    \begin{enumerate}
        \item 非空性 Non-emptiness:对任意 $ \varnothing \notin \mathcal{S} $
        \item 覆盖性 Coverage:$ \bigcup_{A\in \mathcal{S}} A = X $
        \item 互斥性 Mutual Exclusion:对任意 $ A,B\in \mathcal{S} $,若 $ A\neq B $,则 $ A\cap B = \varnothing $
    \end{enumerate}
\end{definition}

\begin{note}
    划分中,每一个元素是都是一个非空子集,这些子集是互不相交的,且它们的并集为整个集合。商集是根据等价关系得到的对集合的划分,不同的等价关系会得到不同的划分。
\end{note}

\newpage