\section{拓扑结构}

\begin{definition}[拓扑空间 Topological Space]
    设$ X $是一个集合,$ \mathcal{T}(X) \subseteq \mathcal{P}(X) $ 是 $ X $ 的子集族,二元组 $ (X,\mathcal{T}(X)) $ 称为拓扑空间,当且仅当,$ \mathcal{T}(X) $ 满足:
    \begin{enumerate}
        \item 空集与全集:$ \varnothing \in \mathcal{T}(X) $ 且 $ X \in \mathcal{T}(X) $;
        \item 对任意并封闭:任意 $ \{U_i:i\in I\} \subseteq \mathcal{T}(X) $,有 $ \bigcup_{i\in I} U_i \in \mathcal{T}(X) $;
        \item 对有限交封闭:有限个 $ U_1,U_2,\ldots,U_n\in \mathcal{T}(X) $,有 $ U_1\cap U_2\cap \cdots \cap U_n \in \mathcal{T}(X) $。
    \end{enumerate}
    子集族 $ \mathcal{T}(X) $ 称为 $ X $ 上的\textbf{拓扑 Topology},$ \mathcal{T}(X) $ 中的元素称为\textbf{开集 Open Set}。
\end{definition}

\begin{note}
    根据拓扑空间的的定义,幂集 $ \mathcal{P}(X) $ 和 $ \{\varnothing,X\} $ 都是 $ X $ 上的拓扑,分别称为\textbf{离散拓扑 Discrete Topology}和\textbf{平凡拓扑 Trivial Topology}。
    给定集合 $X$,$X$ 的任意子集族并非都是 $X$ 上的拓扑,$X$ 的拓扑也非唯一。比如,设 $ X=\{a,b,c\} $,那么下面的子集族都是 $ X $ 上的拓扑:
    \begin{align*}
        &\mathcal{T}_1(X)=\{\varnothing,X\},\\
        &\mathcal{T}_2(X)=\mathcal{P}(X),\\
        &\mathcal{T}_3(X)=\{\varnothing,X,\{a\}\},\\
        &\mathcal{T}_4(X)=\{\varnothing,X,\{a,b\}\},\\
        &\mathcal{T}_5(X)=\{\varnothing,X,\{a\},\{a,b\}\},\\
        &\mathcal{T}_6(X)=\{\varnothing,X,\{a,b\},\{b\},\{b,c\}\}.
    \end{align*}
    下面的子集族不是 $ X $ 上的拓扑:
    \begin{align*}
        &\mathcal{T}_7(X)=\{\varnothing,X,\{a\},\{b\}\},\\
        &\mathcal{T}_8(X)=\{\varnothing,X,\{a,b\},\{b,c\}\}.
    \end{align*}
\end{note}

\vspace{1em}

\begin{definition}[闭集 Closed Set]
    设 $ (X,\mathcal{T}(X)) $ 为拓扑空间,称 $ A\subseteq X $ 为闭集,当且仅当, $ \complement_X A \in \mathcal{T}(X) $。
\end{definition}
\vspace{1em}

\begin{proposition}[闭集的性质]
    设 $ (X,\mathcal{T}(X)) $ 为拓扑空间,则有:
    \begin{enumerate}
        \item 空集与全集:$ \varnothing $ 与 $ X $ 是闭集;
        \item 对任意交封闭:$ \bigcap_{i\in I} A_i $ 是闭集,其中 $ \{A_i:i\in I\}$ 是闭集;
        \item 对有限并封闭:$ A_1\cup A_2\cup \cdots \cup A_n $ 是闭集,其中 $ A_1,A_2,\ldots,A_n $ 是闭集。
    \end{enumerate}
\end{proposition}

\begin{note}
    闭集是对几何空间中闭区间、闭球等概念的抽象。闭集与开集的关系是互补的,即一个集合是闭集,当且仅当它的补集是开集。
    拓扑空间中的闭集可以用来定义点的极限、闭包等概念。
\end{note}

\vspace{1em}
\subsection{连续函数}
\begin{definition}[连续函数 Continuous Function]
    设 $ (X,\mathcal{T}(X)) $ 和 $ (Y,\mathcal{T}(Y)) $ 是拓扑空间,映射 $ f:X\to Y $ 称为连续函数,当且仅当,$ \forall V\in \mathcal{T}(Y) $,有 $ f^{-1}(V)\in \mathcal{T}(X) $。
\end{definition}
\begin{definition}[同胚 Homeomorphism]
    设 $ (X,\mathcal{T}(X)) $ 和 $ (Y,\mathcal{T}(Y)) $ 是拓扑空间,映射 $ f:X\to Y $ 称为同胚,当且仅当,$ f $ 是双射且 $ f $ 与 $ f^{-1} $ 都是连续函数,记为 $ X \cong Y $。
\end{definition}
\begin{note}
    连续函数是对几何空间中连续变化概念的抽象。连续函数保持了拓扑结构,即开集的原像仍然是开集。
    同胚是拓扑空间之间的一种等价关系,表示两个拓扑空间在拓扑意义下是相同的。拓扑空间的同胚类似于代数结构的同构。
\end{note}

% \begin{definition}[度量空间中连续函数的$\varepsilon-\delta$定义]
%     设 $ (X,d_X) $ 和 $ (Y,d_Y) $ 是度量空间,映射 $ f:X\to Y $ 称为连续函数,当且仅当,$ \forall x\in X $ 且 $ \forall \epsilon>0 $,存在 $ \delta>0 $,使得 $ \forall y\in X $,当 $ d_X(x,y)<\delta $ 时,有 $ d_Y(f(x),f(y))<\epsilon $。
% \end{definition}

% \begin{definition}[度量空间中连续函数的开球定义]
%     设 $ (X,d_X) $ 和 $ (Y,d_Y) $ 是度量空间,映射 $ f:X\to Y $ 称为连续函数,当且仅当,$ \forall x\in X $ 且 $ \forall \epsilon>0 $,存在 $ \delta>0 $,使得只要 $y\in B_X(x;\delta)$,就有 $ f(y)\in B_Y(f(x);\epsilon) $。
% \end{definition}

\vspace{1em}

\subsection{度量空间}
\begin{definition}[度量空间 Metric Space]
    设 $ X $ 是一个非空集合,映射 $ d:X\times X\to \mathbb{R} $ 称为度量,当且仅当,$ \forall x,y,z\in X $,有:
    \begin{enumerate}
        \item 非负性:$ d(x,y)\geq 0 $,且当且仅当 $ x=y $ 时取等号;
        \item 对称性:$ d(x,y)=d(y,x) $;
        \item 三角不等式:$ d(x,z)\leq d(x,y)+d(y,z) $。
    \end{enumerate}
    二元组 $ (X,d) $ 称为度量空间,函数 $ d $ 称为 $ X $ 上的度量。
\end{definition}
\vspace{1em}

\begin{definition}[开球 Open Ball]
    设 $ (X,d) $ 为度量空间,$ x\in X $ 且 $ r>0 $,集合 $ B(x,r)=\{y\in X:d(x,y)<r\} $ 称为以 $ x $ 为中心、$ r $ 为半径的开球。
\end{definition}
\vspace{1em}

\begin{proposition}[度量空间诱导的拓扑]
    设 $ (X,d) $ 为度量空间,定义 $ \mathcal{T}(X)=\{U\subseteq X:\forall x\in U,\exists r>0,B(x,r)\subseteq U\} $,则 $ (X,\mathcal{T}(X)) $ 为拓扑空间。
\end{proposition}
\vspace{1em}

\begin{example}
    实数集 $\mathbb{R}$ 和绝对值距离 $ d(x,y)=|x-y| $ 构成一个度量空间,其中,
    \begin{enumerate}
        \item 任意开区间 $ (a,b) $ 是开集;$(a,+\infty)$ 和 $(-\infty,b)$ 也是开集;
        \item 任意闭区间 $ [a,b] $ 是闭集,它是开区间 $(-\infty,b)\cap(a,+\infty)$ 的补集;有限集也是闭集;
        \item $\mathbb{R}$ 和 $\varnothing$ 既是开集也是闭集
        \item 任意半开半闭区间 $ [a,b) $ 和 $ (a,b] $ 既不是开集也不是闭集。
    \end{enumerate}
\end{example}

\vspace{1em}

\begin{note}
    度量空间是对几何空间中距离概念的抽象。度量函数 $ d $ 用来衡量集合中任意两个元素之间的距离。
    常见的度量包括欧氏距离、曼哈顿距离和切比雪夫距离等。
    度量空间可以诱导出拓扑结构,开球的集合构成了 $ X $ 上的一个拓扑,称为由度量 $ d $ 诱导的拓扑。
\end{note}

% \vspace{1em}
% \subsection{赋范线性空间}

% \begin{definition}[赋范线性空间 Normed Linear Space]
%     设 $ V $ 是域 $ \mathbb{F} $ 上的线性空间,映射 $ \|\cdot\|:V\to \mathbb{R} $ 称为范数,当且仅当,$ \forall x,y\in V $ 且 $ \forall \alpha\in \mathbb{F} $,有:
%     \begin{enumerate}
%         \item 非负性:$ \|x\|\geq 0 $,且当且仅当 $ x=0 $ 时取等号;
%         \item 齐次性:$ \|\alpha x\|=|\alpha|\|x\| $;
%         \item 三角不等式:$ \|x+y\|\leq \|x\|+\|y\| $。
%     \end{enumerate}
%     二元组 $ (V,\|\cdot\|) $ 称为赋范线性空间,函数 $ \|\cdot\| $ 称为 $ V $ 上的范数。
% \end{definition}

% \begin{proposition}[赋范线性空间诱导的度量]
%     设 $ (V,\|\cdot\|) $ 为赋范线性空间,定义 $ d:V\times V\to \mathbb{R} $,$ d(x,y)=\|x-y\| $,则 $ (V,d) $ 为度量空间。
% \end{proposition}

% \vspace{1em}

% \subsection{内积空间}

% \begin{definition}[内积空间 Inner Product Space]
%     设 $ V $ 是域 $ \mathbb{F} $ 上的线性空间,映射 $ \langle\cdot,\cdot\rangle:V\times V\to \mathbb{F} $ 称为内积,当且仅当,$ \forall x,y,z\in V $ 且 $ \forall \alpha\in \mathbb{F} $,有:
%     \begin{enumerate}
%         \item 共轭对称性:$ \langle x,y\rangle=\overline{\langle y,x\rangle} $;
%         \item 线性性:$ \langle \alpha x+y,z\rangle=\alpha\langle x,z\rangle+\langle y,z\rangle $;
%         \item 正定性:$ \langle x,x\rangle\geq 0 $,且当且仅当 $ x=0 $ 时取等号。
%     \end{enumerate}
%     二元组 $ (V,\langle\cdot,\cdot\rangle) $ 称为内积空间,函数 $ \langle\cdot,\cdot\rangle $ 称为 $ V $ 上的内积。
% \end{definition}
% \vspace{1em}

% \begin{proposition}[内积空间诱导的赋范线性空间]
%     设 $ (V,\langle\cdot,\cdot\rangle) $ 为内积空间,定义 $ \|\cdot\|:V\to \mathbb{R} $,$ \|x\|=\sqrt{\langle x,x\rangle} $,则 $ (V,\|\cdot\|) $ 为赋范线性空间。
% \end{proposition}
% \vspace{1em}

% \begin{note}
%     赋范线性空间既是线性空间又是度量空间。范数 $ \|\cdot\| $ 用来衡量向量的长度或大小。内积空间既是线性空间又是赋范线性空间。内积 $ \langle\cdot,\cdot\rangle $ 用来衡量向量之间的夹角和正交性。
% \end{note}

\newpage