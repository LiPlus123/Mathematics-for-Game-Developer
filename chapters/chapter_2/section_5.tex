\section{序结构}

\begin{definition}[偏序关系 Partial Order Relation]
    设 $ R $ 为非空集合 $X$ 上的二元关系。称 $ R $ 为 $ X $ 上的偏序关系,当且仅当,$ R $ 满足:对 $ x,y,z\in X $
    \begin{enumerate}
        \item 自反性(Reflexivity):$ (x,x)\in R $;
        \item 反对称性(Antisymmetry):若 $ (x,y)\in R $ 且 $ (y,x)\in R $,则 $ x=y $;
        \item 传递性(Transitivity):若 $ (x,y)\in R $ 且 $ (y,z)\in R $,则 $ (x,z)\in R $。
    \end{enumerate}
    偏序关系常用符号 $ \preceq $ 表示,$ (x,y)\in R $ 简记为 $ x\preceq y $。二元组 $ (X,\preceq) $ 称为偏序集。
\end{definition}
\vspace{1em}

\begin{definition}[严格偏序关系 Strict Partial Order Relation]
    设 $ R $ 为非空集合 $X$ 上的二元关系。称 $ R $ 为 $ X $ 上的严格偏序关系,当且仅当,$ R $ 满足:对 $ x,y,z\in X $
    \begin{enumerate}
        \item 反自反性(Irreflexivity):$ (x,x)\notin R $;
        \item 反对称性(Antisymmetry):若 $ (x,y)\in R $ 且 $ (y,x)\in R $,则 $ x=y $;
        \item 传递性(Transitivity):若 $ (x,y)\in R $ 且 $ (y,z)\in R $,则 $ (x,z)\in R $。
    \end{enumerate}
    严格偏序关系常用符号 $ \prec $ 表示,$ (x,y)\in R $ 简记为 $ x\prec y $。
\end{definition}
\vspace{1em}

\begin{definition}[全序关系 Total Order Relation]
    设 $ \preceq $ 为非空集合 $X$ 上的偏序关系。称 $ \preceq $ 为 $ X $ 上的全序关系,当且仅当,$ \preceq $ 满足:
    \begin{enumerate}
        \item 全序性(Totality):对任意 $ x,y\in X $,要么 $ x \preceq y $,要么 $ y\preceq x $。
    \end{enumerate}
    全序关系也称为线序关系,全序关系常用符号 $ \leq $ 表示,二元组 $ (X,\leq) $ 称为全序集或线序集。
\end{definition}

\begin{note}
    在集合论中,集合中的元素是没有顺序的概念的,当定义了集合中元素之间的某种顺序关系后,集合才具有了序结构。如果是全序关系,那么这个集合中任意两个元素都是可比的。
    序关系强调集合元素之间的可比性,如果集合中的元素是也是一些集合,那么集合之间的包含关系是一种偏序关系,集合之间的优势关系也是一种偏序关系。
\end{note}
\vspace{1em}

\subsection{上下界}

\begin{definition}[上界 Upper Bound]
    设 $ (X,\preceq) $ 为偏序集,$ Y\subseteq X $。若存在 $ b\in X $,使得对任意 $ y\in Y $,均有 $ y\preceq b $,则称 $ b $ 为子集 $ Y $ 的上界。
\end{definition}

\begin{definition}[下界 Lower Bound]
    设 $ (X,\preceq) $ 为偏序集,$ Y\subseteq X $。若存在 $ a\in X $,使得对任意 $ y\in Y $,均有 $ a\preceq y $,则称 $ a $ 为子集 $ Y $ 的下界。
\end{definition}

\begin{note}
    如果子集 $Y$ 存在上界(下界),但不唯一,且不一定属于子集 $ Y $;如果 $ Y $  没有上界(下界),则称 $ Y $ 无界。
\end{note}
\vspace{1em}

\begin{definition}[下确界 Infimum]
    设 $ (X,\preceq) $ 为偏序集,$ Y\subseteq X $。
    若 $ Y $ 存在下界,且存在 $ i\in X $,使得对任意 $ y\in Y $,均有 $ i\preceq y $,
    且对任意 $ a\in X $,若 $ a\preceq y $,则有 $ a\preceq i $,则称 $ i $ 为子集 $ Y $ 的下确界,记为 $ i=\inf Y $。
\end{definition}

\begin{definition}[上确界 Supremum]
    设 $ (X,\preceq) $ 为偏序集,$ Y\subseteq X $。
    若 $ Y $ 存在上界,且存在 $ s\in X $,使得对任意 $ y\in Y $,均有 $ y\preceq s $,
    且对任意 $ b\in X $,若 $ y\preceq b $,则有 $ s\preceq b $,则称 $ s $ 为子集 $ Y $ 的上确界,记为 $ s=\sup Y $。
\end{definition}

\begin{note}
    如果 $Y$ 存在上界,那么上确界是所有上界中最小的一个;如果 $Y$ 存在下界,那么下确界是所有下界中最大的一个。
\end{note}

\vspace{1em}

\begin{theorem}[确界唯一性]
    设 $ (X,\preceq) $ 为偏序集,$ Y\subseteq X $。若 $ Y $ 存在上确界(下确界),则上确界(下确界)唯一。
\end{theorem}

\begin{proof}
    假设 $ s_1,s_2\in X $ 均为子集 $ Y $ 的上确界,且 $ s_1\neq s_2 $。由上确界定义可知,对任意 $ y\in Y $,均有 $ y\preceq s_1 $ 且 $ y\preceq s_2 $,且对任意 $ b\in X $,若 $ y\preceq b $,则有 $ s_1\preceq b $ 且 $ s_2\preceq b $。取 $ b=s_2 $,则有 $ s_1\preceq s_2 $;取 $ b=s_1 $,则有 $ s_2\preceq s_1 $。由反对称性可知,$ s_1=s_2 $,与假设矛盾,因此上确界唯一。下确界唯一的证明类似。
\end{proof}

\vspace{1em}

\begin{definition}[最小元 Minimal Element]
    设 $ (X,\preceq) $ 为偏序集,$ Y\subseteq X $。若存在 $ m\in Y $,使得对任意 $ y\in Y $,均有 $ m\preceq y $,则称 $ m $ 为子集 $ Y $ 的最小元。
\end{definition}

\begin{definition}[最大元 Maximal Element]
    设 $ (X,\preceq) $ 为偏序集,$ Y\subseteq X $。若存在 $ M\in Y $,使得对任意 $ y\in Y $,均有 $ y\preceq M $,则称 $ M $ 为子集 $ Y $ 的最大元。
\end{definition}

% \begin{proposition}
%     设 $ (X,\preceq) $ 为偏序集,$ Y\subseteq X $。若 $ Y $ 存在最小元(最大元),则最小元(最大元)唯一。
% \end{proposition}


\vspace{1em}

\newpage