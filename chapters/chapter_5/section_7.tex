\section{指标记号}

下面介绍一种简化线性元素及其运算书写的\textbf{指标记号 Indicial notation}。

\subsection{自由指标}

\textbf{自由指标}:在表达式的一\textbf{项 Term}中,只出现一次的指标,可以是上标,也可以是下标。表示该表达式在该指标的取值范围内都成立。
例如:一个 $n$ 维向量 $\mathbf{v}\in F^n$ 可以写成有序数组的形式:
\[
    \mathbf{v} = (v_1, v_2, \ldots, v_n)
\]
写成自由指标的形式为:
\[
    v_i, \quad i = 1, 2, \ldots, n
\]
这里的 $i$ 就是一个自由指标,表示 $\mathbf{v}$ 的第 $i$ 个分量。再比如,一个 $m \times n$ 的矩阵 $\mathbf{A} \in F^{m \times n}$ 可以写成:
\[
    \mathbf{A} = \begin{bmatrix}
        a_{11} & a_{12} & \cdots & a_{1n} \\
        a_{21} & a_{22} & \cdots & a_{2n} \\
        \vdots & \vdots & \ddots & \vdots \\
        a_{m1} & a_{m2} & \cdots & a_{mn}
    \end{bmatrix}
\]
写成自由指标的形式为:
\[
    a_{ij} = a^{ij} = a_i^{\ j} = a_j^{\ i}, \quad i = 1, 2, \ldots, m, \quad j = 1, 2, \ldots, n
\]
这里的 $i$ 和 $j$ 就是自由指标,分别表示矩阵 $\mathbf{A}$ 的第 $i$ 行和第 $j$ 列的元素。
再比如,一个三指标的张量 $\mathbf{T} \in F^{m \times n \times p}$ 用自由指标可以写成:
\[
    t_{ijk} = t^{ijk} = t_i^{\ jk} = t_j^{\ ik} = t_k^{\ ij} = t_{ik}^{\ j} = t_{jk}^{\ i} = t_{ij}^{\ k}
\]
\[
    i = 1, 2, \ldots, m, \quad j = 1, 2, \ldots, n, \quad k = 1, 2, \ldots, p
\]
这里的 $i$、$j$ 和 $k$ 就是自由指标,分别表示张量 $\mathbf{T}$ 的第 $i$ 行、第 $j$ 列和第 $k$ 层的元素。
\vspace{1em}

\subsection{哑指标求和约定}
\textbf{哑指标}:在表达式的一项中,出现两次的指标,可以是上标,也可以是下标。表示该指标在该表达式中被求和。
哑标自动求和也称为\textbf{爱因斯坦求和约定 Einstein summation convention}。比如,一个 $n$ 维向量用基 $\mathbf{e}_i$ 线性表出:
\[
    \mathbf{v} = \sum^n_{i=1}v^i \mathbf{e}_i
\]
用哑指标求和约定可以写成:
\[
    \mathbf{v} = v^i \mathbf{e}_i
\]
这里的 $i$ 就是一个哑指标,表示对 $i$ 从 $1$ 到 $n$ 求和。省略了求和符号 $\sum$。
再比如,在 $F^n$ 内积空间中,两个向量 $\mathbf{u}, \mathbf{v} \in F^n$ 的内积可以写成:
\[
    \mathbf{u} \cdot \mathbf{v} = \sum^n_{i=1} u_i v^i
\]
用哑指标求和约定可以写成:
\[
    \mathbf{u} \cdot \mathbf{v} = u_i v^i
\]
再比如,一个 $m \times n$ 的矩阵 $\mathbf{A} \in F^{m \times n}$ 作用在一个 $n$ 维向量 $\mathbf{v} \in F^n$ 用矩阵乘法表示为:
\[
    \mathbf{A} \mathbf{v} = \begin{bmatrix}
        a_{11} & a_{12} & \cdots & a_{1n} \\
        a_{21} & a_{22} & \cdots & a_{2n} \\
        \vdots & \vdots & \ddots & \vdots \\
        a_{m1} & a_{m2} & \cdots & a_{mn}
    \end{bmatrix}\begin{matrix}
        v_1 \\ v_2 \\ \vdots \\ v_n
    \end{matrix}
\]
用哑指标求和约定可以写成:
\[
    (\mathbf{A} \mathbf{v})_i = a_{ij} v^j
\]
这里的 $j$ 就是一个哑指标,表示对 $j$ 从 $1$ 到 $n$ 求和,省略了求和符号 $\sum$。而 $i$ 是一个自由指标,表示结果向量 $\mathbf{A} \mathbf{v}$ 的第 $i$ 个分量。
再比如,矩阵的迹用哑指标求和约定可以写成:
\[
    \mathrm{tr}(\mathbf{A}) = a_i^{\ i} = a^i_{\ i} = a_{ii}
\]
表示对角线元素求和。

\vspace{1em}

\subsection{克罗内克尔符号}

\begin{definition}[克罗内克尔符号 Kronecker delta]:设一个两指标的数,可以记为 $\delta_{ij} = \delta^{ij} = \delta^{i}_{\ j}= \delta^{j}_{\ i}$:
\[
    \delta_{ij} = \begin{cases}
        1, & i = j \\
        0, & i \neq j
    \end{cases}
\]
\label{def:kronecker_delta}
\end{definition}

\begin{definition}[广义克罗内克尔符号 Generalized Kronecker delta]:设一个 $n$ 指标的数,可以记为 $\delta_{j_1 j_2 \ldots j_n}^{i_1 i_2 \ldots i_n}$,定义为:
\[
    \delta_{j_1 j_2 \ldots j_n}^{i_1 i_2 \ldots i_n} = \begin{cases}
        +1, & (j_1, j_2, \ldots, j_n) \text{ 是 } (i_1, i_2, \ldots, i_n) \text{ 的偶排列} \\
        -1, & (j_1, j_2, \ldots, j_n) \text{ 是 } (i_1, i_2, \ldots, i_n) \text{ 的奇排列} \\
        0, & (j_1, j_2, \ldots, j_n) \text{ 既非偶排列也不是奇排列}
    \end{cases}
\]
\label{def:generalized_kronecker_delta}
\end{definition}

\begin{example}
    通过广义克罗内克符号,行列式的展开式还可以表示为:
    \begin{align}
        \det(\mathbf{A}) &= \sum_{\sigma \in S_n} \left( \mathrm{sgn}(\sigma) \prod_{i=1}^{n} a_{i,\sigma(i)} \right) \\
        &= \sum_{(j_1, \cdots, j_n)\in S_n}^{n} \delta^{1 2 \cdots n}_{j_1 j_2 \cdots j_n}  a_{1}^{\ j_1}\cdots a_{n}^{\ j_n}
        \label{eq:generalized_kronecker_determinant}
    \end{align}
    其中,$(j_1,\cdots,j_n)\in S_n$ 是 $1,2,\cdots,n$ 的一个排列。
    \label{ex:generalized_kronecker_determinant}
\end{example}

\newpage