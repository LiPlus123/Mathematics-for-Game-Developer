\section{张量代数}

\subsection{多重线性泛函}

\begin{definition}[线性泛函 Linear Functional]
    设 $V$ 是一个 $F$-线性空间,$F$ 本身也可以看成一个 $F$-线性空间,那么 $T:V\to F$ 的线性映射称为\textbf{线性泛函}。
    \label{def:linear_functional}
\end{definition}

\begin{definition}[多重线性泛函 Multilinear Functional]
    设 $V_1,V_2,\ldots,V_n$ 是 $F$-线性空间,$F$ 本身也可以看成一个 $F$-线性空间,那么 $T:V_1\times V_2\times \cdots \times V_n \to F$ 的映射称为\textbf{多重线性泛函},当且仅当,对于任意的 $i = 1,2,\cdots,n$,$T$ 满足:
    \begin{enumerate}
        \item $\forall \mathbf{v}_i,\mathbf{u}_i \in V_i,\ T(\cdots,\mathbf{v}_i+\mathbf{u}_i,\cdots) = T(\cdots,\mathbf{v}_i,\cdots)+T(\cdots,\mathbf{u}_i,\cdots)$
        \item $\forall \mathbf{v}_i \in V_i,\ \forall r\in F,\ T(\cdots,r\mathbf{v}_i,\cdots) = rT(\cdots,\mathbf{v}_i,\cdots)$
    \end{enumerate}
    也即, $T$ 对每个变量都是线性的。全体多重线性泛函的集合记为 $\mathcal{L}(V_1,V_2,\ldots,V_n;F)$。
    \label{def:multilinear_functional}
\end{definition}

\subsection{对偶线性空间}

\begin{definition}[对偶线性空间 Dual Linear Space]
    设 $V$ 是一个 $F$-线性空间,则所有从 $V$ 到 $F$ 的线性泛函构成的集合 $\mathcal{L}(V,F)$ 也是个线性空间,称为 $V$ 的\textbf{对偶线性空间},记为 $V^*$。
    $V^*$ 中的元素称为 $V$ 的\textbf{对偶向量 Dual Vector} 或\textbf{余向量 Covector}。 
    \label{def:dual_linear_space}
\end{definition}

\begin{proposition}
    设 $V$ 是一个有限维 $F$-线性空间。$V^*$ 是 $V$ 的对偶空间。因为 $V^* = \mathcal{L}(V,F)$,
    所以 $\dim V^* = \dim \mathcal{L}(V,F) = \dim V \cdot \dim F = \dim V$,也即 $V$ 与其对偶空间 $V^*$ 是同构的。
\end{proposition}

\begin{note}
    线性空间与其对偶空间是同构的。
    线性空间 $V$ 也是 $V^*$ 的对偶空间,我们令 $V^{**} = (V^*)^*$。比如,在矩阵乘法的定义下,有:
    \[
        \begin{bmatrix}
            a_1 & a_2 & \cdots & a_n
        \end{bmatrix}\begin{bmatrix}
            b_1 \\ b_2 \\ \vdots \\ b_n
        \end{bmatrix} = \sum_{i=1}^{n} a_i b_i \in F
    \]
    因此,$F^{n\times 1}$ 的对偶空间是 $F^{1\times n}$,$\dim F^{1\times n} = \dim F^{n\times 1} = n$。
\end{note}

\newpage