\section{特殊矩阵}

\subsection{厄米矩阵与厄米算子}

\begin{definition}[对称矩阵 Symmetric Matrix]
    设 $\mathbf{A} \in F^{n \times n}$,如果 $\mathbf{A}^T = \mathbf{A}$,则称 $\mathbf{A}$ 为\textbf{对称矩阵}。
    \label{def:symmetric_matrix}
\end{definition}

\begin{definition}[共轭对称矩阵 Hermitian Matrix]
    设 $\mathbf{A} \in \mathbb{C}^{n \times n}$,如果 ,
    \[
        \mathbf{A}^H = \overline{\mathbf{A}^T} = \mathbf{A}
    \]
    则称 $\mathbf{A}$ 为\textbf{共轭对称矩阵} 或\textbf{厄米矩阵 Hermitian Matrix}。
    \label{def:hermitian_matrix}
\end{definition}

\begin{proposition}[共轭对称矩阵的性质]
    设 $\mathbf{A} \in \mathbb{C}^{n \times n}$ 是共轭对称矩阵,则有:
    \begin{enumerate}
        \item $(\mathbf{A}+\mathbf{B})^H = \mathbf{A}^H + \mathbf{B}^H = \mathbf{A}+\mathbf{B}$
        \item $(c\mathbf{A})^H = \bar{c} \mathbf{A}^H = \bar{c} \mathbf{A}$,其中 $c\in \mathbb{C}$
        \item $(\mathbf{A}\mathbf{B})^H = \mathbf{B}^H \mathbf{A}^H = \mathbf{B} \mathbf{A}$
    \end{enumerate}
\end{proposition}
\begin{note}
    实对称矩阵是实数域上的共轭对称矩阵。矩阵的共轭转置也常记为 $\mathbf{A}^{\dagger}$ 或 $\mathbf{A}^*$。
\end{note}
\vspace{1em}

\begin{definition}[伴随算子 Adjoint Operator]
    设 $V$ 是一个内积空间,$T:V\to V$ 是一个线性变换。如果存在一个线性变换 $T^*:V\to V$,使得 $\forall \mathbf{u},\mathbf{v}\in V$,都有
    \[
        \langle T(\mathbf{u}),\mathbf{v} \rangle = \langle \mathbf{u},T^*(\mathbf{v}) \rangle
    \]
    则称 $T^*$ 为 $T$ 的\textbf{伴随算子}。
    \label{def:adjoint_operator}
\end{definition}

\begin{definition}[自伴随算子 Self-Adjoint Operator]
    设 $V$ 是一个内积空间,线性变换 $T:V\to V$ 是\textbf{自伴随算子},当且仅当,$T = T^*$ ,也即
    \[
        \langle T(\mathbf{u}),\mathbf{v} \rangle = \langle \mathbf{u},T(\mathbf{v}) \rangle = \overline{\langle T(\mathbf{v}),\mathbf{u} \rangle}
    \]
    自伴随算子也称为\textbf{厄米算子 Hermitian Operator}。
    \label{def:self_adjoint_operator}
\end{definition}

\begin{note}
    在有限维的酉空间中,矩阵的共轭转置就是其伴随算子,共轭对称矩阵是自伴随算子。
    需要注意,在中文翻译中,伴随算子中的 adjoint 与伴随矩阵中的 adjugate 是完全不同的概念。
\end{note}

\begin{proposition}
    复内积空间中自伴随算子不同特征值对应的特征向量是正交的;自伴随算子的特征值都是实数。
    \label{prop:self_adjoint_operator_eigenvalue}
\end{proposition}
\begin{proof}
    设 $V$ 是一个复内积空间,$T\in \mathcal{L}(V;V)$ 是一个自伴随算子,$\lambda_1,\lambda_2$ 是 $T$ 的两个特征值,$\mathbf{v}_1,\mathbf{v}_2$ 是对应的特征向量。
    \begin{enumerate}
        \item 当 $\lambda_1 \neq \lambda_2$ 时,$\mathbf{v}_1 \neq \mathbf{v}_2 \neq \mathbf{0}$,则有
        \[
            \langle T(\mathbf{v}_1),\mathbf{v}_2 \rangle = \langle \lambda_1 \mathbf{v}_1,\mathbf{v}_2 \rangle = \lambda_1 \langle \mathbf{v}_1,\mathbf{v}_2 \rangle
        \]
        \[
            \langle \mathbf{v}_1,T(\mathbf{v}_2) \rangle = \langle \mathbf{v}_1,\lambda_2 \mathbf{v}_2 \rangle = \overline{\lambda_2} \langle \mathbf{v}_1,\mathbf{v}_2 \rangle
        \]
        其中, $\langle T(\mathbf{v}_1),\mathbf{v}_2 \rangle = \langle \mathbf{v}_1,T(\mathbf{v}_2) \rangle$,所以
        \[
            (\lambda_1 - \overline{\lambda_2}) \langle \mathbf{v}_1,\mathbf{v}_2 \rangle =  0
        \]
        因为 $\lambda_1 - \overline{\lambda_2} \neq 0$,所以 $\langle \mathbf{v}_1,\mathbf{v}_2 \rangle = 0$,也即 $\mathbf{v}_1 \perp \mathbf{v}_2$。
        \item 当 $\lambda_1 = \lambda_2 = \lambda$ 时,$\mathbf{v}_1 = \mathbf{v}_2 = \mathbf{v} \neq \mathbf{0}$,则有
        \[
            (\lambda - \overline{\lambda}) \langle \mathbf{v},\mathbf{v} \rangle =  0
        \]
        因为 $\langle \mathbf{v},\mathbf{v} \rangle > 0$,所以 $\lambda - \overline{\lambda} = 0$,也即 $\lambda = \overline{\lambda}$,$\lambda$ 是实数。
    \end{enumerate}
\end{proof}

\begin{note}
    命题 \ref{prop:self_adjoint_operator_eigenvalue} 是量子力学的基础。
    量子力学中,物理量对应于复内积空间(可能是有限维,也可能是无限维)的自伴随算子,
    对物理量测量前,系统处于叠加态(特征向量所张成的特征子空间中某个向量),
    当对物理量进行测量时,系统会坍缩到某个本征态(特征向量),
    测量值为特征向量对应的特征值,是一个实数。
\end{note}

\vspace{1em}
\subsection{正交矩阵与酉矩阵}

\begin{definition}[正交矩阵 Orthogonal Matrix]
    设 $\mathbf{A} \in \mathbb{R}^{n \times n}$,如果 $\mathbf{A}^T \mathbf{A} = \mathbf{A} \mathbf{A}^T = \mathbf{I}$,则称 $\mathbf{A}$ 为\textbf{正交矩阵}。
    \label{def:orthogonal_matrix}
\end{definition}

\begin{definition}[酉矩阵 Unitary Matrix]
    设 $\mathbf{A} \in \mathbb{C}^{n \times n}$,如果 $\mathbf{A}^H \mathbf{A} = \mathbf{A} \mathbf{A}^H = \mathbf{I}$,则称 $\mathbf{A}$ 为\textbf{酉矩阵}。
    \label{def:unitary_matrix}
\end{definition}

\begin{note}
    正交矩阵是实数域上的酉矩阵。根据定义,正交矩阵和酉矩阵都是可逆矩阵,并且有 $\mathbf{A}^{-1} = \mathbf{A}^T$ 或 $\mathbf{A}^{-1} = \mathbf{A}^H$。
\end{note}

\vspace{1em}

\begin{proposition}[正交(酉)矩阵的乘积依然是正交(酉)矩阵]
    设 $\mathbf{A},\mathbf{B} \in \mathbb{R}^{n \times n}$ 是两个正交矩阵,则 $\mathbf{A}\mathbf{B}$ 也是一个正交矩阵。
    \[
        (\mathbf{A}\mathbf{B})^T (\mathbf{A}\mathbf{B}) = \mathbf{B}^T \mathbf{A}^T \mathbf{A}\mathbf{B} = \mathbf{B}^T \mathbf{B} = \mathbf{I}
    \]    
    设 $\mathbf{A},\mathbf{B} \in \mathbb{C}^{n \times n}$ 是两个酉矩阵,则 $\mathbf{A}\mathbf{B}$ 也是一个酉矩阵。
    \[
        (\mathbf{A}\mathbf{B})^H (\mathbf{A}\mathbf{B}) = \mathbf{B}^H \mathbf{A}^H \mathbf{A}\mathbf{B} = \mathbf{B}^H \mathbf{B} = \mathbf{I}
    \]
\end{proposition}

\begin{proposition}[正交(酉)矩阵的行列式的绝对值为 1]
    设 $\mathbf{A} \in \mathbb{R}^{n \times n}$ 是一个正交矩阵,则有
    \[
        \det(\mathbf{A}^T \mathbf{A}) = \det(\mathbf{A})\det(\mathbf{A}) = \det(\mathbf{A})^2 = \det(\mathbf{I}) = 1
    \]
    所以 $\det(\mathbf{A}) = \pm 1$。
    \\
    \\
    设 $\mathbf{A} \in \mathbb{C}^{n \times n}$ 是一个酉矩阵,则有
    \[
        \det(\mathbf{A}^H \mathbf{A})  = \overline{\det(\mathbf{A})} \det(\mathbf{A})  = |\det(\mathbf{A})|^2 = \det(\mathbf{I}) = 1
    \]
    所以 $\det(\mathbf{A}) = e^{i\theta}$。
\end{proposition}

\begin{proposition}[正交(酉)矩阵的行(列)向量组构成标准正交基]
    设 $\mathbf{A} = [\mathbf{a}_1,\mathbf{a}_2,\ldots,\mathbf{a}_n] \in \mathbb{R}^{n \times n}$ 是一个正交矩阵,则 $\{\mathbf{a}_1,\mathbf{a}_2,\ldots,\mathbf{a}_n\}$ 是 $\mathbb{R}^n$ 的一个标准正交基。
    设 $\mathbf{A} = [\mathbf{a}_1,\mathbf{a}_2,\ldots,\mathbf{a}_n] \in \mathbb{C}^{n \times n}$ 是一个酉矩阵,则 $\{\mathbf{a}_1,\mathbf{a}_2,\ldots,\mathbf{a}_n\}$ 是 $\mathbb{C}^n$ 的一个标准正交基。
\end{proposition}

\vspace{1em}
\begin{proposition}[正交(酉)矩阵是保内积的线性变换]
    设 $\mathbf{A} \in \mathbb{R}^{n \times n}$ 是一个正交矩阵,则有:
    \[
        \langle \mathbf{A}\mathbf{x},\mathbf{A}\mathbf{y} \rangle = (\mathbf{A}\mathbf{x})^T \mathbf{A}\mathbf{y}  = \mathbf{x}^T \mathbf{A}^T \mathbf{A} \mathbf{y} = \mathbf{x}^T \mathbf{y} = \langle \mathbf{x},\mathbf{y} \rangle
    \]
    设 $\mathbf{A} \in \mathbb{C}^{n \times n}$ 是一个酉矩阵,则有:
    \[
        \langle \mathbf{A}\mathbf{x},\mathbf{A}\mathbf{y} \rangle = (\mathbf{A}\mathbf{x})^H \mathbf{A}\mathbf{y}  = \mathbf{x}^H \mathbf{A}^H \mathbf{A} \mathbf{y} = \mathbf{x}^H \mathbf{y} = \langle \mathbf{x},\mathbf{y} \rangle
    \]
\end{proposition}

\begin{corollary}[正交(酉)矩阵是保长度的线性变换]
    设 $\mathbf{A}$ 是一个正交(酉)矩阵,则有:
    \[
        \|\mathbf{A}\mathbf{x}\|_2 = \sqrt{\langle \mathbf{A}\mathbf{x},\mathbf{A}\mathbf{x} \rangle} = \sqrt{\langle \mathbf{x},\mathbf{x} \rangle} = \|\mathbf{x}\|_2
    \]
\end{corollary}

\begin{corollary}[正交(酉)矩阵的特征值的绝对值为 1]
    设 $\mathbf{A}$ 是一个正交(酉)矩阵,$\lambda$ 是 $\mathbf{A}$ 的一个特征值,则有
    \[
        |\lambda| = 1
    \]
\end{corollary}
\begin{proof}
    设 $\mathbf{v} \neq \mathbf{0}$ 是 $\mathbf{A}$ 对应于特征值 $\lambda$ 的特征向量,则有
    \[
        \mathbf{A}\mathbf{v} = \lambda \mathbf{v}
    \]
    因为 $\mathbf{A}$ 是正交(酉)矩阵,所以 
    \[
        \|\mathbf{A}\mathbf{x}\|_2 = \|\lambda \mathbf{x}\|_2 = |\lambda| \|\mathbf{x}\|_2 = \|\mathbf{x}\|_2
    \]
    所以 $|\lambda| = 1$。
\end{proof}

\begin{note}
    正交矩阵和酉矩阵都是保内积的线性变换,其几何意义是保持向量的长度和向量之间的夹角不变,比如旋转变化就是一种正交变换。
    在无限维空间中,保内积的线性变换称为酉算子。
\end{note}
\vspace{0.5em}

\begin{proposition}[正交(酉)矩阵不同特征值对应的特征向量是正交的]
    设 $\mathbf{A}$ 是一个正交(酉)矩阵,$\lambda_1,\lambda_2$ 是 $\mathbf{A}$ 的两个不同特征值,$\mathbf{v}_1,\mathbf{v}_2$ 是对应的特征向量,则有
    \[
        \langle \mathbf{v}_1,\mathbf{v}_2 \rangle = 0
    \]
\end{proposition}
\begin{proof}
    设 $\mathbf{A}$ 是一个正交(酉)矩阵,$\lambda_1,\lambda_2$ 是 $\mathbf{A}$ 的两个不同特征值,$\mathbf{v}_1,\mathbf{v}_2$ 是对应的特征向量,则有
    \[
        \mathbf{A}\mathbf{v}_1 = \lambda_1 \mathbf{v}_1, \quad \mathbf{A}\mathbf{v}_2 = \lambda_2 \mathbf{v}_2
    \]
    因为 $\mathbf{A}$ 是正交(酉)矩阵,所以 
    \[
        \langle \mathbf{A}\mathbf{v}_1,\mathbf{A}\mathbf{v}_2 \rangle = \langle \lambda_1 \mathbf{v}_1,\lambda_2 \mathbf{v}_2 \rangle = \lambda_1 \overline{\lambda_2} \langle \mathbf{v}_1,\mathbf{v}_2 \rangle = \langle \mathbf{v}_1,\mathbf{v}_2 \rangle
    \]
    所以 
    \[
        (\lambda_1 \overline{\lambda_2} - 1) \langle \mathbf{v}_1,\mathbf{v}_2 \rangle = 0
    \]
    因为 $\lambda_1 \neq \lambda_2$ 并且 $\lambda_1\bar{\lambda}_1 = \lambda_2\bar{\lambda}_2 = 1$,所以 $\lambda_1 \overline{\lambda_2}  \neq 1$,所以 $\langle \mathbf{v}_1,\mathbf{v}_2 \rangle = 0$。
\end{proof}

\vspace{1em}
\subsection{三角矩阵与可三角化矩阵}
\begin{definition}[上三角矩阵 Upper Triangular Matrix]
    设 $\mathbf{A} = [a_{ij}] \in F^{m \times n}$,如果
    \[
        a_{ij} = 0, \quad \forall i > j
    \]
    则称 $\mathbf{A}$ 为\textbf{上三角矩阵}。
    \label{def:upper_triangular_matrix}
\end{definition}

\begin{definition}[下三角矩阵 Lower Triangular Matrix]
    设 $\mathbf{A} = [a_{ij}] \in F^{m \times n}$,如果
    \[
        a_{ij} = 0, \quad \forall i < j
    \]
    则称 $\mathbf{A}$ 为\textbf{下三角矩阵}。
    \label{def:lower_triangular_matrix}
\end{definition}

\begin{proposition}[三角矩阵的主对角元素即为它的特征值]
    设 $\mathbf{A} \in F^{n \times n}$ 是一个上三角矩阵,那么
    \[
        \mathbf{A} = \begin{bmatrix}
            a_{11} & a_{12} & \cdots & a_{1n} \\
            0 & a_{22} & \cdots & a_{2n} \\
            \vdots & \vdots & \ddots & \vdots \\
            0 & 0 & \cdots & a_{nn}
        \end{bmatrix}
    \]
    特征多项式为:
    \[
        p(\lambda) = \det(\mathbf{A} - \lambda \mathbf{I}) = \begin{vmatrix}
            a_{11} - \lambda & a_{12} & \cdots & a_{1n} \\
            0 & a_{22} - \lambda & \cdots & a_{2n} \\
            \vdots & \vdots & \ddots & \vdots \\
            0 & 0 & \cdots & a_{nn} - \lambda
        \end{vmatrix} = (a_{11} - \lambda)(a_{22} - \lambda) \cdots (a_{nn} - \lambda)
    \]
    则 $\mathbf{A}$ 的特征值为主对角元素,即
    \[
        \lambda_1 = a_{11}, \lambda_2 = a_{22}, \ldots, \lambda_n = a_{nn}
    \]
    同理,下三角矩阵的特征值也为其主对角元素。
    \label{prop:triangular_matrix_eigenvalue}
\end{proposition}

\begin{note}
    三角矩阵的有非常好的性质,它的特征值可以直接从对角元素中读出。
    根据命题 \ref{prop:diagonal_matrix_eigenvalue} 和命题 \ref{prop:relationship_between_eigenvalues} 可知,三角矩阵的行列式等于其对角元素的乘积;
    三角矩阵可逆,当且仅当,其对角元素全不为零。
\end{note}

\begin{definition}[可三角化矩阵 Triangularizable Matrix]
    设 $\mathbf{A} \in F^{n \times n}$,如果存在一个可逆矩阵 $\mathbf{P} \in F^{n \times n}$,使得
    \[
        \mathbf{P}^{-1}\mathbf{A}\mathbf{P} = \mathbf{T}
    \]
    其中,$\mathbf{T}$ 是一个上(下)三角矩阵,则称 $\mathbf{A}$ 是\textbf{可三角化矩阵}。
    \label{def:triangularizable_matrix}
\end{definition}

\begin{proposition}[可三角化矩阵的充要条件]
    设 $\mathbf{A} \in F^{n \times n}$,则 $\mathbf{A}$ 可三角化的充分必要条件是 $\mathbf{A}$ 在域 $F$ 上有 $n$ 个特征值(重根算作多个特征值)。
    \label{prop:triangularizable_matrix_necessary_sufficient_condition}
\end{proposition}

\begin{note}
    可三角化矩阵是指可以通过相似变换将其化为三角矩阵的矩阵,也即在一组适当的基底下,线性变换可以表示为三角矩阵。
    根据代数基本定理\ref{thm:FundamentalTheoremOfAlgebra},在复数域上,任意 $n$ 阶方阵都有 $n$ 个特征值(重根算作多个特征值),所以任意复数域上的方阵都是可三角化的。
    但是在实数域上,并非所有的方阵都是可三角化的,比如
    \[
        \mathbf{A} = \begin{bmatrix}
            0 & -1 \\
            1 & 0
        \end{bmatrix}
    \]
    的特征多项式为 $p(\lambda) = \lambda^2 + 1$,在实数域上没有特征值,所以 $\mathbf{A}$ 在实数域上是不可三角化矩阵。
\end{note}

\vspace{1em}
\subsection{对角矩阵与可对角化矩阵}

\begin{definition}[对角矩阵 Diagonal Matrix]
    设 $\mathbf{A} = [a_{ij}] \in F^{m \times n}$,如果
    \[
        a_{ij} = 0, \quad \forall i \neq j
    \]
    则称 $\mathbf{A}$ 为\textbf{对角矩阵}。
    \label{def:diagonal_matrix}
\end{definition}

\begin{proposition}[对角矩阵的主对角元素即为它的特征值]
    设 $\mathbf{A} = [a_{ij}] \in F^{n \times n}$ 是一个对角矩阵,那么 $\mathbf{A}$ 的特征值为其对角元素 $a_{11}, a_{22}, \ldots, a_{nn}$。
    \label{prop:diagonal_matrix_eigenvalue}
\end{proposition}

\begin{note}
    对角矩阵是特殊的三角矩阵,既可以看成是上三角矩阵,也可以看成是下三角矩阵。三角矩阵的关于特征值的性质同样适用于对角矩阵。
\end{note}
\vspace{1em}

\begin{definition}[可对角化矩阵 Diagonalizable Matrix]
    设 $\mathbf{A} \in F^{n \times n}$,如果存在一个可逆矩阵 $\mathbf{P} \in F^{n \times n}$,使得
    \[
        \mathbf{P}^{-1}\mathbf{A}\mathbf{P} = \mathbf{D}
    \]
    其中,$\mathbf{D}$ 是一个对角矩阵,则称 $\mathbf{A}$ 是\textbf{可对角化矩阵}。
    \label{def:diagonalizable_matrix}
\end{definition}

\begin{proposition}[可对角化矩阵的充要条件]
    设 $\mathbf{A} \in F^{n \times n}$,则 $\mathbf{A}$ 可对角化的充分必要条件是 $\mathbf{A}$ 有 $n$ 个线性无关的特征向量。
    \label{prop:diagonalizable_matrix_necessary_sufficient_condition}
\end{proposition}

\begin{proof}
    设 $\mathbf{A} \in F^{n \times n}$ 可对角化,则存在一个可逆矩阵 $\mathbf{P} \in F^{n \times n}$,使得
    \[
        \mathbf{P}^{-1}\mathbf{A}\mathbf{P} = \mathbf{D} \iff \mathbf{A}\mathbf{P} = \mathbf{D}\mathbf{P}
    \]
    其中,$\mathbf{D}$ 是一个对角矩阵。设 $\mathbf{D}$ 的主对角元素为 $d_1,d_2,\ldots,d_n$,则根据命题 \ref{prop:diagonal_matrix_eigenvalue} 可知,$d_1,d_2,\ldots,d_n$ 是 $\mathbf{D}$ 的特征值。
    设 $\mathbf{e}_1,\mathbf{e}_2,\ldots,\mathbf{e}_n$ 是 $F^n$ 的标准正交基,则有
    \[
        \mathbf{D}\mathbf{e}_i = d_i \mathbf{e}_i, \quad i = 1,2,\ldots,n
    \]
    令 $\mathbf{v}_i = \mathbf{P}\mathbf{e}_i$,则有
    \[
        \mathbf{A}\mathbf{v}_i = \mathbf{A}\mathbf{P}\mathbf{e}_i = \mathbf{P}\mathbf{D}\mathbf{e}_i = d_i \mathbf{P}\mathbf{e}_i = d_i \mathbf{v}_i, \quad i = 1,2,\ldots,n
    \]
    所以,$\mathbf{v}_1,\mathbf{v}_2,\ldots,\mathbf{v}_n$ 是 $\mathbf{A}$ 的 $n$ 个线性无关的特征向量。
    \\
    \\
    反之,设 $\mathbf{A} \in F^{n \times n}$ 有 $n$ 个线性无关的特征向量 $\mathbf{v}_1,\mathbf{v}_2,\ldots,\mathbf{v}_n$,对应的特征值为 $\lambda_1,\lambda_2,\ldots,\lambda_n$。
    令 $\mathbf{P} = [\mathbf{v}_1,\mathbf{v}_2,\ldots,\mathbf{v}_n]$,则 $\mathbf{P}$ 是一个可逆矩阵。
    令 $\mathbf{D}$ 为对角矩阵,其主对角元素为 $\lambda_1,\lambda_2,\ldots,\lambda_n$,则有
    \[
        \mathbf{A}\mathbf{P} = \mathbf{A}[\mathbf{v}_1,\mathbf{v}_2,\ldots,\mathbf{v}_n] = [\mathbf{A}\mathbf{v}_1,\mathbf{A}\mathbf{v}_2,\ldots,\mathbf{A}\mathbf{v}_n] = [\lambda_1 \mathbf{v}_1,\lambda_2 \mathbf{v}_2,\ldots,\lambda_n \mathbf{v}_n] = \mathbf{P}\mathbf{D}
    \]
    所以,$\mathbf{P}^{-1}\mathbf{A}\mathbf{P} = \mathbf{D}$,即 $\mathbf{A}$ 可对角化。
\end{proof}

\begin{corollary}[矩阵可对角化的充要条件]
    矩阵 $\mathbf{A}\in F^{n\times n}$ 可对角化,有如下等价命题:
    \begin{enumerate}
        \item 存在一个可逆矩阵 $\mathbf{P} \in F^{n \times n}$,使得 $\mathbf{P}^{-1}\mathbf{A}\mathbf{P} = \mathbf{D}$,其中,$\mathbf{D}$ 是一个对角矩阵
        \item $\mathbf{A}$ 有 $n$ 个线性无关的特征向量
        \item $\mathbf{A}$ 的每个特征值 $\lambda_i$ 的几何重数等于其代数重数
    \end{enumerate}
\end{corollary}

\begin{note}
    可对角化矩阵是一种特殊的可三角化矩阵,是对矩阵的进一步简化。
    可对角化的矩阵一定是可三角化的矩阵,但反之不然。
    例如,上三角矩阵
    \[\begin{bmatrix}
        1 & 1 \\
        0 & 1
    \end{bmatrix}\] 
    只有一个特征值 1,对应的特征向量只有一个线性无关的向量 $(1,0)$,所以它不可对角化。
    可对角化矩阵相比可三角化矩阵,有更强的约束条件。
    可对角化矩阵要求有 $n$ 个线性无关的特征向量,但并不意味着矩阵有 $n$ 个不同的特征值。
\end{note}
\vspace{1em}

\begin{proposition}[实对称矩阵是可对角化的矩阵]
    设 $\mathbf{A} \in \mathbb{R}^{n \times n}$ 是一个实对称矩阵,则 $\mathbf{A}$ 可对角化。且存在一个正交矩阵 $\mathbf{Q} \in \mathbb{R}^{n \times n}$,使得
    \[
        \mathbf{Q}^T \mathbf{A} \mathbf{Q} = \mathbf{D}
    \]
    其中,$\mathbf{D}$ 是一个实对角矩阵。
    \label{prop:real_symmetric_matrix_diagonalizable}
\end{proposition}

\begin{proposition}[复厄米矩阵是可对角化的矩阵]
    设 $\mathbf{A} \in \mathbb{C}^{n \times n}$ 是一个复厄米矩阵,则 $\mathbf{A}$ 可对角化。且存在一个酉矩阵 $\mathbf{U} \in \mathbb{C}^{n \times n}$,使得
    \[
        \mathbf{U}^H \mathbf{A} \mathbf{U} = \mathbf{D}
    \]
    其中,$\mathbf{D}$ 是一个实对角矩阵。
    \label{prop:hermitian_matrix_diagonalizable}
\end{proposition}

\begin{note}
    实对称矩阵和复厄米矩阵都是可对角化的矩阵的充分条件,但不是必要条件。    
\end{note}
\vspace{1em}

\subsection{二次型与对称半定矩阵}

\begin{definition}[二次型 Quadratic Form]
    设 $\mathbf{A} \in F^{n \times n}$,$\mathbf{x} \in F^n$,则函数
    \[
        f(\mathbf{x}) = \mathbf{x}^T \mathbf{A} \mathbf{x}
    \]
    称为 $\mathbf{A}$ 在变量 $\mathbf{x}$ 上的\textbf{二次型}。
    \label{def:quadratic_form}
\end{definition}
\vspace{1em}

\begin{example}[二次型与圆锥曲线]
    二维空间中的圆锥曲线的一般形式为:
    \begin{equation}
        Ax^2 + Bxy + Cy^2 + Dx + Ey + F = 0
        \label{eq:conic_section_general_form}
    \end{equation}
    其中,$A,B,C,D,E,F$ 为常数,用二次型表示为:
    \begin{equation}
        \begin{bmatrix}
            x & y & 1
        \end{bmatrix}
        \begin{bmatrix}
            A & B/2 & D/2 \\
            B/2 & C & E/2 \\
            D/2 & E/2 & F
        \end{bmatrix}
        \begin{bmatrix}
            x \\ y \\ 1
        \end{bmatrix} = 0
        \label{eq:conic_section_quadratic_form}
    \end{equation}
    其中二次项部分:
    \begin{equation}
        \mathbf{A}=\begin{bmatrix}
            x & y
        \end{bmatrix}
        \begin{bmatrix}
            A & B/2 \\
            B/2 & C
        \end{bmatrix}
        \begin{bmatrix}
            x \\ y
        \end{bmatrix} = Ax^2 + Bxy + Cy^2
        \label{eq:conic_section_quadratic_form_quadratic_part}
    \end{equation}

    决定了圆锥曲线的类型:
    \begin{enumerate}
        \item 当 $\det{\mathbf{A}} = B^2 - 4AC < 0$ 时,表示椭圆(包括圆)
        \item 当 $\det{\mathbf{A}} = B^2 - 4AC = 0$ 时,表示抛物线
        \item 当 $\det{\mathbf{A}} = B^2 - 4AC > 0$ 时,表示双曲线
    \end{enumerate}
    \label{ex:conic_section_quadratic_form}
\end{example}
\vspace{1em}

\begin{definition}
    设 $\mathbf{A}\in \mathbb{R}^{n\times n}$ 是一个实对称矩阵 $\mathbf{A}=\mathbf{A}^T$。
    \begin{enumerate}
        \item $\mathbf{A}$ 是\textbf{对称半正定 Symmetric Positive Semi-Definite} 的,当且仅当,
        $\forall \mathbf{x} \neq \mathbf{0} \in \mathbb{R}^n$,有 $\mathbf{x}^T\mathbf{A}\mathbf{x} \geq 0$;
        \item $\mathbf{A}$ 是\textbf{对称正定 Symmetric Positive Definite} 的,当且仅当,
        $\forall \mathbf{x} \neq \mathbf{0} \in \mathbb{R}^n$,有 $\mathbf{x}^T\mathbf{A}\mathbf{x} > 0$;
        \item $\mathbf{A}$ 是\textbf{对称半负定 Symmetric Negative Semi-Definite} 的,当且仅当,
        $\forall \mathbf{x} \neq \mathbf{0} \in \mathbb{R}^n$,有 $\mathbf{x}^T\mathbf{A}\mathbf{x} \leq 0$;
        \item $\mathbf{A}$ 是\textbf{对称负定 Symmetric Negative Definite} 的,当且仅当,
        $\forall \mathbf{x} \neq \mathbf{0} \in \mathbb{R}^n$,有 $\mathbf{x}^T\mathbf{A}\mathbf{x} < 0$。
    \end{enumerate}
    \label{def:symmetric_positive_definite_matrix}
\end{definition}

\begin{note}
    $\mathbf{A}$ 是一个实对称矩阵,根据命题 \ref{prop:real_symmetric_matrix_diagonalizable} 可知,$\mathbf{A}$ 可对角化,
    且存在正交矩阵 $\mathbf{Q}$,使得 $\mathbf{Q}^T \mathbf{A} \mathbf{Q} = \mathbf{D}$,其中,$\mathbf{D}$ 是一个实对角矩阵。
    设 $\mathbf{D}$ 的主对角元素为 $\lambda_1,\lambda_2,\ldots,\lambda_n$,是 $\mathbf{A}$ 的 $n$ 个特征值(重根算作多个特征值)。
    由此可以得到实对称矩阵正定性的特征值判定法。
\end{note}
\vspace{1em}

\begin{proposition}[实对称矩阵的正定性的特征值判定法]
    设 $\mathbf{A}\in \mathbb{R}^{n\times n}$ 是一个实对称矩阵,$\lambda_1,\ldots,\lambda_n$ 是 $\mathbf{A}$ 的 $n$ 个特征值(重根算作多个特征值),则有:
    \begin{enumerate}
        \item $\mathbf{A}$ 是对称半正定的,当且仅当,$\forall i=1,2,\ldots,n$,有 $\lambda_i \geq 0$;
        \item $\mathbf{A}$ 是对称正定的,当且仅当,$\forall i=1,2,\ldots,n$,有 $\lambda_i > 0$;
        \item $\mathbf{A}$ 是对称半负定的,当且仅当,$\forall i=1,2,\ldots,n$,有 $\lambda_i \leq 0$;
        \item $\mathbf{A}$ 是对称负定的,当且仅当,$\forall i=1,2,\ldots,n$,有 $\lambda_i < 0$。
    \end{enumerate}
    \label{prop:real_symmetric_matrix_positive_definite}
\end{proposition}

\begin{corollary}
    一个对称正定矩阵和对称负定矩阵的特征值均不为 0,因此一定是可逆矩阵;并且如果 $\mathbf{A}$ 是对称正定矩阵,那么 $-\mathbf{A}$ 是对称负定矩阵。
\end{corollary}

\begin{note}
    实对称矩阵的正定性与其特征值的关系,是判断实对称矩阵正定性的一个重要方法。
    此外还可以通过实对称矩阵的主子式判定法来判断实对称矩阵的正定性。
\end{note}
\vspace{1em}

\begin{definition}[顺序主子矩阵 Leading Principal Submatrix]
    设 $\mathbf{A} = [a_{ij}] \in F^{n \times n}$,则 $\mathbf{A}$ 的 $k$ 阶\textbf{顺序主子矩阵}为
    \[
        \mathbf{A}_k = [a_{ij}]_{i,j=1}^k = \begin{bmatrix}
            a_{11} & a_{12} & \cdots & a_{1k} \\
            a_{21} & a_{22} & \cdots & a_{2k} \\
            \vdots & \vdots & \ddots & \vdots \\
            a_{k1} & a_{k2} & \cdots & a_{kk}
        \end{bmatrix}
        \in F^{k \times k}
    \]
    其中,$k = 1,2,\ldots,n$。$\mathbf{A}_k$ 的行列式 $\det(\mathbf{A}_k)$ 称为 $\mathbf{A}$ 的第 $k$ 阶\textbf{顺序主子式}。
    \label{def:leading_principal_submatrix}
\end{definition}

\begin{proposition}[实对称矩阵的正定性的顺序主子式判定法]
    设 $\mathbf{A}\in \mathbb{R}^{n\times n}$ 是一个实对称矩阵,$\mathbf{A}_k$ 是 $\mathbf{A}$ 的 $k$ 阶顺序主子矩阵,则有:
    \begin{enumerate}
        \item $\mathbf{A}$ 是对称正定的,当且仅当,$\forall k=1,2,\ldots,n$,有 $\det(\mathbf{A}_k) > 0$;
        \item $\mathbf{A}$ 是对称半正定的,当且仅当,$\forall k=1,2,\ldots,n$,有 $\det(\mathbf{A}_k) \geq 0$;
        \item $\mathbf{A}$ 是对称负定的,当且仅当,$\forall k=1,2,\ldots,n$,有 $(-1)^k \det(\mathbf{A}_k) > 0$;
        \item $\mathbf{A}$ 是对称半负定的,当且仅当,$\forall k=1,2,\ldots,n$,有 $(-1)^k \det(\mathbf{A}_k) \geq 0$。
    \end{enumerate}
    \label{prop:real_symmetric_matrix_positive_definite_leading_principal_minor}
\end{proposition}
\begin{proof}
    
\end{proof}
\vspace{2em}

\begin{proposition}[实对称矩阵的正定性的判定法 3]
    设 $\mathbf{A}\in \mathbb{R}^{n\times n}$ 是一个实对称矩阵,那么:
    \begin{enumerate}
        \item $\mathbf{A}$ 是对称正定的,当且仅当,存在一个可逆矩阵 $\mathbf{P} \in \mathbb{R}^{n \times n}$,使得 $\mathbf{A} = \mathbf{P}^T \mathbf{P}$;
        \item $\mathbf{A}$ 是对称半正定的,当且仅当,存在一个矩阵 $\mathbf{P} \in \mathbb{R}^{m \times n}$,使得 $\mathbf{A} = \mathbf{P}^T \mathbf{P}$;
        \item $\mathbf{A}$ 是对称负定的,当且仅当,存在一个可逆矩阵 $\mathbf{P} \in \mathbb{R}^{n \times n}$,使得 $\mathbf{A} = -\mathbf{P}^T \mathbf{P}$;
        \item $\mathbf{A}$ 是对称半负定的,当且仅当,存在一个矩阵 $\mathbf{P} \in \mathbb{R}^{m \times n}$,使得 $\mathbf{A} = -\mathbf{P}^T \mathbf{P}$。
    \end{enumerate}
    \label{prop:real_symmetric_matrix_positive_definite_3}
\end{proposition}

\begin{proof}
    只证明命题 \ref{prop:real_symmetric_matrix_positive_definite_3} 中第 1 条,第 2、3、4 的证明类似。
    \begin{align*}
        \mathbf{x}^T\mathbf{A}\mathbf{x} & = \mathbf{x}^T \mathbf{P}^T \mathbf{P} \mathbf{x} \\
        & = (\mathbf{P}\mathbf{x})^T (\mathbf{P}\mathbf{x}) \\
        & = \|\mathbf{P}\mathbf{x}\|_2^2
    \end{align*}
    因为 $\mathbf{P}$ 是可逆矩阵,并且 $\mathbf{x}\neq \mathbf{0}$,所以 $\mathbf{P}\mathbf{x} \neq \mathbf{0}$,并且 $\|\mathbf{P}\mathbf{x}\|_2^2 > 0$。
    根据定义 \ref{def:symmetric_positive_definite_matrix},$\mathbf{A}$ 是对称正定的。
\end{proof}

% \begin{example}[二次型与二阶线性偏微分方程]
%     二阶线性偏微分方程的一般形式为:
%     \[
%         \sum^n_{i=1}\sum^n_{j=1}a_{ij}(\mathbf{x})u_{x_i,x_j}(\mathbf{x})+\sum^n_{i=1}b_i(\mathbf{x})u_{x_i}(\mathbf{x})+c(\mathbf{x})u(\mathbf{x})=f(\mathbf{x})
%     \]
%     其中,$a_{ij},b_i,c:F^n\to F$ 是已知函数,$u:F^n\to F$ 是未知函数。二阶偏导部分:
%     \[
%         \sum^n_{i=1}\sum^n_{j=1}a_{ij}(\mathbf{x})u_{x_i,x_j}(\mathbf{x}) = \mathcal{L}u(\mathbf{x})
%     \]
%     同样决定了二阶线性偏微分方程的类型。将线性算子 $\mathcal{L}$ 用用二次型表示为:
%     \[
%         \mathcal{L}=\begin{bmatrix}
%             \frac{\partial}{\partial x_1} & \frac{\partial}{\partial x_2} & \cdots & \frac{\partial}{\partial x_n}
%             \end{bmatrix}\begin{bmatrix}
%             a_{11} &  a_{12} & \cdots &  a_{1n}\\
%             a_{21} &  a_{22} & \cdots &  a_{2n}\\
%             \vdots &  \vdots & \ddots &  \vdots\\
%             a_{n1} &  a_{n2} & \cdots &  a_{nn}\\
%             \end{bmatrix}\begin{bmatrix}
%             \frac{\partial}{\partial x_1} \\
%             \frac{\partial}{\partial x_2} \\
%             \vdots \\
%             \frac{\partial}{\partial x_n} \\
%         \end{bmatrix}
%     \]
%     其中,
% \end{example}

\begin{note}
    在优化问题中,二次规划问题中的目标函数通常是一个二次型,
    该二次型对应的矩阵是对称正定矩阵,这样可以保证目标函数是凸函数,从而保证优化问题有唯一解。
\end{note}

\newpage