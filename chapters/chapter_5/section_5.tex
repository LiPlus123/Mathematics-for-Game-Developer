\section{特征值与特性向量}

\begin{definition}[特征值与特征向量 Eigenvalue and Eigenvector]
    设 $V$ 是一个 $F$-线性空间,线性变换 $T\in \mathcal{L}(V,V)$,如果存在标量 $\lambda\in F$ 和非零向量 $\mathbf{v}\in V$,使得
    \[
        T(\mathbf{v}) = \lambda \mathbf{v}
    \]
    则称 $\lambda$ 为 $T$ 的\textbf{特征值},$\mathbf{v}$ 为 $T$ 对应于特征值 $\lambda$ 的\textbf{特征向量}。
    \label{def:eigenvalue_and_eigenvector}
\end{definition}

\begin{note}
    线性变换的的特征向量是一种在变换下仅改变大小而不改变方向的向量;特征值表示缩放的程度。
    不管是有限维还是无限维线性空间,特征值与特征向量的定义都是一样的。
    不过在无限维线性空间中,线性变换将不能用矩阵表示,特征值与特征向量的研究会更加困难,下面的内容主要针对有限维线性空间。
\end{note}

\vspace{1em}

\subsection{特征多项式}



\newpage