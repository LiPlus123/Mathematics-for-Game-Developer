\chapter{线性代数}

\section{线性空间的定义}

\begin{definition}[线性空间 Linear Space]
    设 $ (F,+,\cdot) $ 为域,$ (V,+) $ 为阿贝尔群。如果存在一个标量乘法 $ \cdot: F\times V \to V $,满足以下条件:
    \begin{enumerate}
        \item 对任意 $ r,s\in F $,$ \mathbf{v}\in V $,有 $ (r+s)\cdot \mathbf{v} = r\cdot \mathbf{v} + s\cdot \mathbf{v} $;
        \item 对任意 $ r\in F $,$ \mathbf{u},\mathbf{v}\in V $,有 $ r\cdot (\mathbf{u}+\mathbf{v}) = r\cdot \mathbf{u} + r\cdot \mathbf{v} $;
        \item 对任意 $ r,s\in F $,$ \mathbf{v}\in V $,有 $ (r\cdot s)\cdot \mathbf{v} = r\cdot (s\cdot \mathbf{v}) $;
        \item 对任意 $ \mathbf{v}\in V $,有 $ 1\cdot \mathbf{v} = \mathbf{v} $,$1\in F$ 为 $ F $ 的乘法单位元。
    \end{enumerate}
    则称 $ (V,+,\cdot) $ 为 $ F $-\textbf{线性空间},或向量空间 Vector Space。$V$ 中的元素称为向量 Vector,$F$ 中的元素称为标量 Scalar。
    \label{def:linear_space}
\end{definition}

\begin{note}
    线性空间是在交换群上定义了关于域的标量乘法,在提到线性空间时,一定要明确它的数域。$\mathbf{0}\in \mathbf{V}$ 是加法单位元,也称为零向量 Zero Vector。
\end{note}

\begin{definition}[线性子空间 Linear Subspace]
    设 $ (V,+,\cdot) $ 为 $ F $-线性空间,$ W \subseteq V $。如果 $ (W,+,\cdot) $ 也是 $ F $-线性空间,则称 $ W $ 为 $ V $ 的\textbf{线性子空间}
    \label{def:linear_subspace}
\end{definition}

\begin{theorem}[线性子空间判定定理]
    设 $ (V,+,\cdot) $ 为 $ F $-线性空间,$ W \subseteq V $。则 $ W $ 为 $ V $ 的线性子空间的充分必要条件为:
    \begin{enumerate}
        \item $ \mathbf{0}\in W $;
        \item 对任意 $ \mathbf{u},\mathbf{v}\in W $,有 $ \mathbf{u}-\mathbf{v}\in W $;
        \item 对任意 $ r,s\in F $,$ \mathbf{u},\mathbf{v}\in W $,有 $ r \mathbf{u}+s \mathbf{v}\in W $。
    \end{enumerate}
    \label{thm:linear_subspace_judge}
\end{theorem}

\begin{note}
    在线性空间的判定定理中,条件 1,2 也是子群的判定定理,因此线性子空间也是原空间的子群。
    线性子空间会继承原线性空间的结构,比如,二维平面 $ \mathbb{R}^2 $ 是三维空间 $ \mathbb{R}^3 $ 的线性子空间。
\end{note}

\vspace{1em}

\begin{definition}[$F^n$ 空间]
    设 $F$ 为一个域,$F^n$ 中的元素为 $n$ 元数组 $\mathbf{v}=(v_1,v_2,\cdots,v_n)$。$F^n$ 上定义向量加法和标量乘法:
    \begin{enumerate}
        \item 向量加法:对于任意 $\mathbf{u}=(u_1,\cdots,u_n),\mathbf{v}=(v_1,\cdots,v_n)\in F^n$,有
        \[
            \mathbf{u}+\mathbf{v}=(u_1+v_1,u_2+v_2,\cdots,u_n+v_n).
        \]
        \item 标量乘法:对于任意 $r\in F$,$\mathbf{v}=(v_1,v_2,\cdots,v_n)\in F^n$,有
        \[
            r\cdot \mathbf{v} = (r\cdot v_1,r\cdot v_2,\cdots,r\cdot v_n).
        \]
    \end{enumerate}
    则 $F^n$ 是一个 $F$-线性空间。
    \label{def:Fn_space}
\end{definition}

\begin{note}
    $F^n$ 空间是最基础的线性空间,$F^n$ 空间的维数为 $n$,所有 $n$ 维 $F$-线性空间都与 $F^n$ 空间同构。
    如果 $F=\mathbb{R}$,在 $\mathbb{R}^n$ 的基础上在赋予内积结构,得到\textbf{欧几里得空间 Euclidean Space};
    如果 $F=\mathbb{C}$,在 $\mathbb{C}^n$ 的基础上赋予内积结构,得到\textbf{酉空间 Unitary Space}。
\end{note}

\newpage