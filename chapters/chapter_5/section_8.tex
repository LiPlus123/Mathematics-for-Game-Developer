\section{张量代数}

\subsection{多重线性泛函}

\begin{definition}[线性泛函 Linear Functional]
    设 $V$ 是一个 $F$-线性空间,$F$ 本身也可以看成一个 $F$-线性空间,那么 $T:V\to F$ 的线性映射称为\textbf{线性泛函}。
    \label{def:linear_functional}
\end{definition}

\begin{definition}[多重线性泛函 Multilinear Functional]
    设 $V_1,V_2,\ldots,V_n$ 是 $F$-线性空间,$F$ 本身也可以看成一个 $F$-线性空间,那么 $T:V_1\times V_2\times \cdots \times V_n \to F$ 的映射称为\textbf{多重线性泛函},当且仅当,对于任意的 $i = 1,2,\cdots,n$,$T$ 满足:
    \begin{enumerate}
        \item $\forall \mathbf{v}_i,\mathbf{u}_i \in V_i,\ T(\cdots,\mathbf{v}_i+\mathbf{u}_i,\cdots) = T(\cdots,\mathbf{v}_i,\cdots)+T(\cdots,\mathbf{u}_i,\cdots)$
        \item $\forall \mathbf{v}_i \in V_i,\ \forall r\in F,\ T(\cdots,r\mathbf{v}_i,\cdots) = rT(\cdots,\mathbf{v}_i,\cdots)$
    \end{enumerate}
    全体多重线性泛函的集合记为 $\mathcal{L}(V_1,V_2,\ldots,V_n;F)$。
    \label{def:multilinear_functional}
\end{definition}

\begin{definition}[多重线性泛函的加法]
    $\forall T_1,T_2\in \mathcal{L}(V_1,V_2,\ldots,V_n;F)$,定义其和为
    \[
        (T_1+T_2)(\mathbf{v}_1,\mathbf{v}_2,\ldots,\mathbf{v}_n) = T_1(\mathbf{v}_1,\mathbf{v}_2,\ldots,\mathbf{v}_n) + T_2(\mathbf{v}_1,\mathbf{v}_2,\ldots,\mathbf{v}_n)
    \]
    \label{def:multilinear_functional_addition}
\end{definition}

\begin{definition}[多重线性泛函的数乘]
    $\forall T\in \mathcal{L}(V_1,V_2,\ldots,V_n;F)$,$\forall r\in F$,定义其数乘为
    \[
        (rT)(\mathbf{v}_1,\mathbf{v}_2,\ldots,\mathbf{v}_n) = rT(\mathbf{v}_1,\mathbf{v}_2,\ldots,\mathbf{v}_n)
    \]
    \label{def:multilinear_functional_scalar_multiplication}
\end{definition}

\begin{proposition}[全体多重线性泛函构成一个线性空间]
    $\mathcal{L}(V_1,V_2,\ldots,V_n;F)$ 在加法 \ref{def:multilinear_functional_addition} 和数乘 \ref{def:multilinear_functional_scalar_multiplication} 下是一个 $F$-线性空间。
    其维数为
    \[
        \dim \mathcal{L}(V_1,V_2,\ldots,V_n;F) = \dim V_1 \cdot \dim V_2 \cdots \dim V_n
    \]
\end{proposition}

\begin{note}
    多重线性泛函是对线性泛函的推广,它对每个变量都是线性的。全体多重线性泛函构成一个线性空间,其维数是各个线性空间维数的乘积。
\end{note}


\vspace{1em}
\subsection{对偶线性空间}

\begin{definition}[对偶线性空间 Dual Linear Space]
    设 $V$ 是一个 $F$-线性空间,则所有从 $V$ 到 $F$ 的线性泛函构成的集合 $\mathcal{L}(V,F)$ 也是个线性空间,称为 $V$ 的\textbf{对偶线性空间},记为 $V^*$。
    $V^*$ 中的元素称为 $V$ 的\textbf{对偶向量 Dual Vector} 或\textbf{余向量 Covector}。 
    \label{def:dual_linear_space}
\end{definition}

\begin{proposition}[线性空间与其对偶空间是自然同构的]
    设 $V$ 是一个有限维 $F$-线性空间。$V^*$ 是 $V$ 的对偶空间。因为 $V^* = \mathcal{L}(V,F)$,
    所以 $\dim V^* = \dim \mathcal{L}(V,F) = \dim V \cdot \dim F = \dim V$,也即 $V$ 与其对偶空间 $V^*$ 是同构的。
\end{proposition}

\begin{note}
    线性空间与其对偶空间是自然同构的。
    线性空间 $V$ 也是 $V^*$ 的对偶空间,我们令 $V^{**} = (V^*)^*$。比如,在矩阵乘法的定义下,有:
    \[
        \begin{bmatrix}
            a_1 & a_2 & \cdots & a_n
        \end{bmatrix}\begin{bmatrix}
            b_1 \\ b_2 \\ \vdots \\ b_n
        \end{bmatrix} = \sum_{i=1}^{n} a_i b_i \in F
    \]
    因此,$F^{n\times 1}$ 的对偶空间是 $F^{1\times n}$,$\dim F^{1\times n} = \dim F^{n\times 1} = n$。
\end{note}
\vspace{1em}

\subsection{张量的定义}

\begin{definition}[张量 Tensor]
    设 $V$ 是一个 $F$-线性空间,$V^*$ 是 $V$ 的对偶空间。则 $V$ 的 $(r,s)$ 型张量是一个多重线性泛函 $T \in \mathcal{L}(\underbrace{V^*,V^*,\ldots,V^*}_{r\text{ 个}},\underbrace{V,V,\ldots,V}_{s\text{ 个}};F)$:
    \[
        T(\underbrace{\bm{\omega}^1,\bm{\omega}^2,\ldots,\bm{\omega}^r}_{r\text{ 个}},\underbrace{\mathbf{v}_1,\mathbf{v}_2,\ldots,\mathbf{v}_s}_{s\text{ 个}})
    \]
    全体 $(r,s)$ 型张量的集合记为 $V^k_s$。
\end{definition}

\begin{note}
    张量是一种特殊的多重线性泛函,它的前 $r$ 个变量是对偶空间 $V^*$ 中的向量,后 $s$ 个变量是线性空间 $V$ 中的向量。
    $V^k_s$ 是 $\mathcal{L}(\underbrace{V^*,V^*,\ldots,V^*}_{r\text{ 个}},\underbrace{V,V,\ldots,V}_{s\text{ 个}};F)$ 的线性子空间。
    张量空间的维数为 $\dim V^k_s = (\dim V)^{r+s}$。如果 $\dim V = n$,则 $\dim V^k_s = n^{r+s}$。
\end{note}

\vspace{1em}

\begin{definition}[张量积 Tensor Product]
    设 $V$ 是一个 $F$-线性空间,$V^*$ 是 $V$ 的对偶空间。$T \in V^r_s$,$S \in V^p_q$,则定义 $T$ 和 $S$ 的张量积为 $T \otimes S \in V^{r+p}_{s+q}$,那么:
    \[
        (T \otimes S)(\underbrace{\bm{\omega}^1,\ldots,\bm{\omega}^{r+p}}_{r+p\text{ 个}},\underbrace{\mathbf{v}_1,\ldots,\mathbf{v}_{s+q}}_{s+q\text{ 个}}) = T(\bm{\omega}^1,\ldots,\bm{\omega}^r,\mathbf{v}_1,\ldots,\mathbf{v}_s) \cdot S(\bm{\omega}^{r+1},\ldots,\bm{\omega}^{r+p},\mathbf{v}_{s+1},\ldots,\mathbf{v}_{s+q})
    \]
    \label{def:tensor_product}
\end{definition}

\begin{proposition}[张量积的性质]
    设 $V$ 是一个 $F$-线性空间,$V^*$ 是 $V$ 的对偶空间。$T,T_1,T_2 \in V^r_s$,$S,S_1,S_2 \in V^p_q$,$a,b\in F$,则有:
    \begin{enumerate}
        \item 结合律:$(T \otimes S) \otimes R = T \otimes (S \otimes R)$
        \item 分配律:$(T_1 + T_2) \otimes S = T_1 \otimes S + T_2 \otimes S$
        \item 分配律:$T \otimes (S_1 + S_2) = T \otimes S_1 + T \otimes S_2$
        \item 数乘结合律:$(aT) \otimes S = a(T \otimes S) = T \otimes (aS)$
    \end{enumerate}
\end{proposition}

\begin{definition}[简单张量 Simple Tensor]
    设 $V$ 是一个 $F$-线性空间,$V^*$ 是 $V$ 的对偶空间。
    如果存在 $\mathbf{u}_1,\cdots,\mathbf{u}_r\in V,\ \mathbf{u}_1^*,\cdots,\mathbf{u}_s^*\in V^*$,得到 $(r,s)$ 型张量:
    \begin{align*}
        \mathbf{T}(\mathbf{v}_1^*,\cdots,\mathbf{v}_r^*,\mathbf{v}_1,\cdots,\mathbf{v}_s) &:= \mathbf{u}_1\otimes \cdots \otimes\mathbf{u}_r \otimes \mathbf{u}_1^*\otimes \cdots \otimes\mathbf{u}_s^* (\mathbf{v}_1^*,\cdots,\mathbf{v}_r^*,\mathbf{v}_1,\cdots,\mathbf{v}_s)\\
        &=\mathbf{u}_1(\mathbf{v}_1^*)\cdots\mathbf{u}_r(\mathbf{v}_r^*)\mathbf{u}_1^*(\mathbf{v}_1)\cdots\mathbf{u}_s^*(\mathbf{v}_s)
    \end{align*}
    则称 $T$ 为一个\textbf{简单张量}。
\end{definition}

\begin{note}
    张量积是构造张量的一种重要方法。
    下面介绍通过张量积构造的简单张量 —— 并矢,
    它是一种简单 $(1,1)$ 型张量。
\end{note}
\vspace{1em}

\begin{example}[并矢 Dyadics]
    设 $\mathbb{R}^{1\times n}$ 与其对偶空间 $\mathbb{R}^{n\times 1}$,任意行向量:
    \[
        \mathbf{a} = \begin{bmatrix}
            a_1 & a_2 & \cdots & a_n
        \end{bmatrix} \in \mathbb{R}^{1\times n}
    \]
    与任意列向量:
    \[
        \mathbf{b}^* = \begin{bmatrix}
            b^1 \\ b^2 \\ \vdots \\ b^n
        \end{bmatrix} \in \mathbb{R}^{n\times 1}
    \]
    $\mathbf{a}, \mathbf{b}^*$ 的张量积记为 $\mathbf{a}\otimes \mathbf{b}^*$ 称为\textbf{并矢}:
    \begin{align*}
        \mathbf{a}\otimes \mathbf{b}^*(\mathbf{v}^*,\mathbf{u}) &= \mathbf{a}(\mathbf{v}^*)\mathbf{b}^*(\mathbf{u})\\
        &= \begin{bmatrix}
            v_1 & v_2 & \cdots & v_n
        \end{bmatrix} \begin{bmatrix}
            b^1 \\ b^2 \\ \vdots \\ b^n
        \end{bmatrix} \cdot \begin{bmatrix}
            a_1 & a_2 & \cdots & a_n
        \end{bmatrix} \begin{bmatrix}
            u^1 \\ u^2 \\ \vdots \\ u^n
        \end{bmatrix} \\
        &= \begin{bmatrix}
            v_1 & v_2 & \cdots & v_n
        \end{bmatrix} \begin{bmatrix}
            b^1a_1 & b^1a_2 & \cdots & b^1a_n \\
            b^2a_1 & b^2a_2 & \cdots & b^2a_n \\
            \vdots & \vdots & \ddots & \vdots \\
            b^na_1 & b^na_2 & \cdots & b^na_n
        \end{bmatrix}\begin{bmatrix}
            u^1 \\ u^2 \\ \vdots \\ u^n
        \end{bmatrix}
    \end{align*}
    因此一个 $(1,1)$ 型张量 $T \in \mathbb{R}^1_1$ 可以表示为一个 $n\times n$ 的矩阵。
    \label{ex:dyadics}
\end{example}

\vspace{1em}

\subsection{协变基与逆变基}

\vspace{1em}
\subsection{张量的缩并}

\vspace{1em}
\subsection{度量张量}

\vspace{1em}
\subsection{外形式}



\newpage

