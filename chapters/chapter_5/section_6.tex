\section{矩阵分解}

% \subsection{和分解}



\subsection{LU 分解}

\begin{definition}[LU 分解]
设 $\mathbf{A} \in \mathbb{R}^{n \times n}$,如果存在一个下三角矩阵 $\mathbf{L} \in \mathbb{R}^{n \times n}$ 和一个上三角矩阵 $\mathbf{U} \in \mathbb{R}^{n \times n}$,使得
\[
\mathbf{A} = \mathbf{L} \mathbf{U}
\]
则称 $\mathbf{A}$ 有 LU 分解,$\mathbf{L}$ 和 $\mathbf{U}$ 分别称为 $\mathbf{A}$ 的下三角矩阵和上三角矩阵。
\end{definition}

\begin{proposition}
    
\end{proposition}

\subsection{QR 分解}

% \begin{definition}[QR 分解]
% 设 $\mathbf{A} \in \mathbb{R}^{m \times n}$,如果存在一个正交矩阵 $\mathbf{Q} \in \mathbb{R}^{m \times m}$ 和一个上三角矩阵 $\mathbf{R} \in \mathbb{R}^{m \times n}$,使得
% \[
% \mathbf{A} = \mathbf{Q} \mathbf{R}
% \]
% 则称 $\mathbf{A}$ 有 QR 分解,$\mathbf{Q}$ 和 $\mathbf{R}$ 分别称为 $\mathbf{A}$ 的正交矩阵和上三角矩阵。

\subsection{特征值分解}


\subsection{奇异值分解}

\subsection{极分级}

\newpage