\section{范数与内积}

\subsection{赋范线性空间}

\begin{definition}[赋范线性空间 Normed Linear Space]
    设 $ V $ 是 $F$-线性空间,映射 $ \|\cdot\|:V\to \mathbb{R} $ 称为范数,当且仅当,$ \forall \mathbf{x},\mathbf{y}\in V $ 且 $ \forall \alpha\in \mathbb{F} $,有:
    \begin{enumerate}
        \item 非负性:$ \|\mathbf{x}\|\geq 0 $,且当且仅当 $ \mathbf{x}=0 $ 时取等号;
        \item 齐次性:$ \|\alpha \mathbf{x}\|=|\alpha|\|\mathbf{x}\| $;
        \item 三角不等式:$ \|\mathbf{x}+\mathbf{y}\| \leq \|\mathbf{x}\|+\|\mathbf{y}\| $。
    \end{enumerate}
    二元组 $ (V,\|\cdot\|) $ 称为赋范线性空间,函数 $ \|\cdot\| $ 称为 $ V $ 上的范数。
    \label{def:normed_linear_space}
\end{definition}

\begin{proposition}[赋范线性空间诱导的度量]
    设 $ (V,\|\cdot\|) $ 为赋范线性空间,定义 $ d:V\times V\to \mathbb{R} $:
    \[
        \forall \mathbf{x},\mathbf{y}\in V, d(\mathbf{x},\mathbf{y})=\|\mathbf{x}-\mathbf{y}\| 
    \]
    则 $ (V,d) $ 为度量空间。
\end{proposition}

\begin{note}
    赋范线性空间是对向量空间中长度概念的抽象。范数函数 $ \|\cdot\| $ 用来衡量向量的大小或长度。
    赋范线性空间可以诱导出度量结构,因此赋范线性空间也是度量空间。
\end{note}

\vspace{1em}
\subsection{常见范数的定义}

\begin{definition}[$p$-范数 $p$-Norm]
    设 $ \mathbf{v}=(v_1,v_2,\ldots,v_n)\in F^n $,其中 $F$ 是 $\mathbb{R}$ 或 $\mathbb{C}$。定义 $ \mathbf{v} $ 的\textbf{$p$-范数}为
    \[
        \|\mathbf{v}\|_p = \left( \sum_{i=1}^{n} |v_i|^p \right)^{\frac{1}{p}}
    \]
    特别地,当 $ p=2 $ 时,称为\textbf{欧几里得范数},记为 $\|\mathbf{v}\|_2$;当 $ p=1 $ 时,称为\textbf{曼哈顿范数},记为 $\|\mathbf{v}\|_1$;当 $ p=\infty $ 时,称为\textbf{切比雪夫范数},记为 $\|\mathbf{v}\|_\infty$,其定义为
    \[
        \|\mathbf{v}\|_\infty = \max(|v_1|,|v_2|,\ldots,|v_n|)
    \]
    \label{def:p-norm}
\end{definition}

% \subsection{矩阵范数}

\vspace{1em}
\subsection{内积空间}

\begin{definition}[内积空间 Inner Product Space]
    设 $ V $ 是 $F$-线性空间,其中 $F$ 通常是 $\mathbb{R}$ 或 $\mathbb{C}$。映射 $ \langle\cdot,\cdot\rangle:V\times V\to F $ 称为内积,当且仅当,$ \forall \mathbf{x},\mathbf{y},\mathbf{z}\in V $ 且 $ \forall \alpha\in F $,有:
    \begin{enumerate}
        \item 共轭对称性:$ \langle \mathbf{x},\mathbf{y}\rangle=\overline{\langle \mathbf{y},\mathbf{x}\rangle} $;
        \item 线性性:$ \langle \alpha \mathbf{x}+\mathbf{y},\mathbf{z}\rangle=\alpha\langle \mathbf{x},\mathbf{z}\rangle+\langle \mathbf{y},\mathbf{z}\rangle $;
        \item 正定性:$ \langle \mathbf{x},\mathbf{x}\rangle \in \mathbb{R} \geq 0 $,且当且仅当 $ \mathbf{x}=0 $ 时取等号。
    \end{enumerate}
    二元组 $ (V,\langle\cdot,\cdot\rangle) $ 称为内积空间,函数 $ \langle\cdot,\cdot\rangle $ 称为 $ V $ 上的内积。
    \label{def:inner_product_space}
\end{definition}

\begin{proposition}[内积空间诱导的赋范线性空间]
    设 $ (V,\langle\cdot,\cdot\rangle) $ 为 $F$-内积空间,其中 $F$ 是 $\mathbb{R}$ 或 $\mathbb{C}$。定义 $ \|\cdot\|:V\to \mathbb{R} $:
    \[
        \forall \mathbf{x}\in V, \|\mathbf{x}\|=\sqrt{\langle \mathbf{x},\mathbf{x}\rangle}
    \]
    则 $ (V,\|\cdot\|) $ 为赋范线性空间。
\end{proposition}

\begin{note}
    内积空间是对向量空间中角度和长度概念的抽象。内积函数 $ \langle\cdot,\cdot\rangle $ 用来衡量两个向量之间的夹角和长度。
    内积空间可以诱导出赋范结构,因此内积空间也是赋范线性空间。
\end{note}

\begin{definition}[正交 Orthogonality]
    设 $ (V,\langle\cdot,\cdot\rangle) $ 为 $F$-内积空间,$ \mathbf{x},\mathbf{y}\in V $。如果 $ \langle \mathbf{x},\mathbf{y}\rangle=0 $,则称 $ \mathbf{x} $ 与 $ \mathbf{y} $ 正交,记作 $ \mathbf{x}\perp \mathbf{y} $。
    \label{def:orthogonality}
\end{definition}

\newpage