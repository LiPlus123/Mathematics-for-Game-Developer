\section{矩阵分解}

\subsection{LU 分解}

\begin{definition}[LU 分解]
设 $\mathbf{A} \in \mathbb{R}^{n \times n}$,如果存在一个下三角矩阵 $\mathbf{L} \in \mathbb{R}^{n \times n}$ 和一个上三角矩阵 $\mathbf{U} \in \mathbb{R}^{n \times n}$,使得
\[
\mathbf{A} = \mathbf{L} \mathbf{U}
\]
则称 $\mathbf{A}$ 有 LU 分解。
\end{definition}

\begin{example}[使用 LU 分解求解线性方程组]
    设 $\mathbf{A} = \mathbf{L} \mathbf{U}$,则可以将线性方程组 $\mathbf{A}\mathbf{x} = \mathbf{b}$ 转化为两个阶段:
    \begin{align*}
        \mathbf{L}\mathbf{y} &= \mathbf{b}\\
        \mathbf{U}\mathbf{x} &= \mathbf{y}
    \end{align*}
    先利用前向替换法求解 $\mathbf{y}$:
    \begin{align*}
        y_1 &= \frac{b_1}{l_{11}} \\
        y_i &= \frac{1}{l_{ii}}\left[b_i - \sum_{j=1}^{i-1} l_{ij} y_j\right], \quad i=2,3,\cdots,n
    \end{align*}
    再利用后向替换法求解 $\mathbf{x}$:
    \begin{align*}
        x_n &= \frac{y_n}{u_{nn}} \\
        x_i &= \frac{1}{u_{ii}}\left[y_i - \sum_{j=i+1}^{n} u_{ij} x_j\right], \quad i=n-1,n-2,\cdots,1
    \end{align*}
    这样就可以高效地求解线性方程组 $\mathbf{A}\mathbf{x} = \mathbf{b}$。
    \label{ex:lu_decomposition_solve_linear_system}
\end{example}

\begin{example}[使用 LU 分解计算行列式]
    设 $\mathbf{A} = \mathbf{L} \mathbf{U}$,则行列式可以表示为:
    \[
        \det(\mathbf{A}) = \det(\mathbf{L}) \cdot \det(\mathbf{U})
    \]
    由于 $\mathbf{L}$ 和 $\mathbf{U}$ 都是三角矩阵,所以它们的行列式等于对角线元素的乘积:
    \[
        \det(\mathbf{L}) = \prod_{i=1}^{n} l_{ii}, \quad \det(\mathbf{U}) = \prod_{i=1}^{n} u_{ii}
    \]
    因此,行列式可以简化为:
    \[
        \det(\mathbf{A}) = \left( \prod_{i=1}^{n} l_{ii} \right) \left( \prod_{i=1}^{n} u_{ii} \right)
    \]
    这样就可以高效地计算行列式。
    \label{ex:lu_decomposition_determinant}
\end{example}


\begin{proposition}[LU 分解的存在性]
    设 $\mathbf{A} \in \mathbb{R}^{n \times n}$,
    \begin{enumerate}
        \item 矩阵能 LU 分解的充分条件:如果主子式 $\det{\mathbf{A}_k}$ 均非零,那么 $\mathbf{A}$ 能做 LU 分解;
        \item 矩阵能 LU 分解的必要条件:如果矩阵 $\mathbf{A}$ 能做 LU 分解,那么 $\mathbf{A}$ 可逆。
    \end{enumerate}
\end{proposition}

\begin{note}
    矩阵的 LU 分解并不总是存在,但对于大多数实际应用中的矩阵,LU 分解是可行且高效的。
    通常使用\textbf{高斯消元法}、\textbf{Doolittle 方法}或 \textbf{Crout 方法}来计算 LU 分解。
\end{note}
\vspace{1em}

\subsection{Cholesky 分解}

\begin{proposition}[Cholesky 分解]
    设 $\mathbf{A} \in \mathbb{R}^{n \times n}$ 是对称正定矩阵,则存在唯一的下三角矩阵 $\mathbf{L} \in \mathbb{R}^{n \times n}$,使得
    \[
        \mathbf{A} = \mathbf{L} \mathbf{L}^T
    \]
    该分解称为 \textbf{Cholesky 分解}。
    \label{proposition:cholesky_decomposition}
\end{proposition}

\begin{note}
    Cholesky 分解是一种特殊的 LU 分解,适用于对称正定矩阵。
    Cholesky 分解的计算效率高于一般的 LU 分解,因为它只需要计算一半的矩阵元素。
    Cholesky 分解分为两步:
    \begin{enumerate}
        \item 计算下三角矩阵 $\mathbf{L}$ 的对角线元素:
        \[
            l_{ii} = \sqrt{a_{ii} - \sum_{k=1}^{i-1} l_{ik}^2}, \quad i=1,2,\ldots,n
        \]
        \item 计算下三角矩阵 $\mathbf{L}$ 的非对角线元素:
        \[
            l_{ij} = \frac{1}{l_{jj}}\left(a_{ij} - \sum_{k=1}^{j-1} l_{ik} l_{jk}\right), \quad i=j+1,j+2,\ldots,n; j=1,2,\ldots,n-1
        \]
    \end{enumerate}
    \label{note:cholesky_decomposition}
\end{note}
\vspace{1em}

\subsection{QR 分解}
\begin{proposition}[QR 分解]
    任何矩阵 $\mathbf{A} \in \mathbb{R}^{m \times n}$ 都可以分解为一个正交矩阵 $\mathbf{Q} \in \mathbb{R}^{m \times m}$ 和一个上三角矩阵 $\mathbf{R} \in \mathbb{R}^{m \times n}$ 的乘积,即
    \[
        \mathbf{A} = \mathbf{Q} \mathbf{R}
    \]
    该分解称为 \textbf{QR 分解}。
\end{proposition}

\begin{proof}
    设 $\mathbf{A} = [\mathbf{a}_1, \mathbf{a}_2, \ldots, \mathbf{a}_n]$,使用 Gram-Schmidt 正交化 \ref{prop:gram_schmidt} 过程构造正交矩阵 $\mathbf{Q}$:
    \begin{align*}
        \mathbf{u}_1 &= \mathbf{a}_1 \\
        \mathbf{e}_1 &= \frac{\mathbf{u}_1}{\|\mathbf{u}_1\|} \\
        \mathbf{u}_2 &= \mathbf{a}_2 - (\mathbf{a}_2 \cdot \mathbf{e}_1) \mathbf{e}_1 \\
        \mathbf{e}_2 &= \frac{\mathbf{u}_2}{\|\mathbf{u}_2\|} \\
        &\vdots \\
        \mathbf{u}_n &= \mathbf{a}_n - \sum_{j=1}^{n-1} (\mathbf{a}_n \cdot \mathbf{e}_j) \mathbf{e}_j \\
        \mathbf{e}_n &= \frac{\mathbf{u}_n}{\|\mathbf{u}_n\|}
    \end{align*}
    其中,$\{\mathbf{e}_1, \mathbf{e}_2, \ldots, \mathbf{e}_n\}$ 是正交归一化后的向量组,构成正交矩阵 $\mathbf{Q} = [\mathbf{e}_1, \mathbf{e}_2, \ldots, \mathbf{e}_m]$。
    
    接下来,构造上三角矩阵 $\mathbf{R}$:
    \[
        r_{ij} = 
        \begin{cases}
            \mathbf{a}_j \cdot \mathbf{e}_i, & i \leq j \\ 
            0, & i > j 
        \end{cases}
    \]
    这样,$\mathbf{R}$ 是一个上三角矩阵。
    
    最后,可以验证 $\mathbf{A} = \mathbf{Q} \mathbf{R}$:
    \[
        \mathbf{A} = [\mathbf{a}_1, \ldots, \mathbf{a}_n] = [\sum_{i=1}^{m} r_{i1} \mathbf{e}_i, \ldots, \sum_{i=1}^{m} r_{in} \mathbf{e}_i] = \mathbf{Q} \mathbf{R}
    \]
\end{proof}

\vspace{1em}

\subsection{特征值分解}
\begin{definition}[特征值分解 Eigenvalue Decomposition]
    设 $\mathbf{A} \in \mathbb{R}^{n \times n}$,如果存在一个可逆矩阵 $\mathbf{P} \in \mathbb{R}^{n \times n}$ 和一个对角矩阵 $\mathbf{D} \in \mathbb{R}^{n \times n}$,使得
    \[
        \mathbf{A} = \mathbf{P} \mathbf{D} \mathbf{P}^{-1}
    \]
    则称 $\mathbf{A}$ 有\textbf{特征值分解}。
\end{definition}

\begin{note}
    矩阵的特征值分解与可对角化矩阵的定义 \ref{def:diagonalizable_matrix} 等价。
    能特征值分解的矩阵与可对角化矩阵的的条件相同,即矩阵 $\mathbf{A}$ 有 $n$ 个线性无关的特征向量。
\end{note}
\vspace{1em}

\subsection{奇异值分解}
\begin{proposition}[奇异值分解 Singular Value Decomposition]
    任意矩阵 $\mathbf{A} \in \mathbb{R}^{m \times n}$ 都可以分解为:
    \[
        \mathbf{A} = \mathbf{U} \boldsymbol{\Sigma} \mathbf{V}^T
    \]
    其中,
    \begin{enumerate}
        \item $\mathbf{U} \in \mathbb{R}^{m \times m}$ 是正交矩阵,其中的列向量称为\textbf{左奇异向量};
        \item $\mathbf{V} \in \mathbb{R}^{n \times n}$ 是正交矩阵,其中的列向量称为\textbf{右奇异向量};
        \item $\boldsymbol{\Sigma} \in \mathbb{R}^{m \times n}$ 是对角矩阵,对角线上的元素称为\textbf{奇异值},所有奇异值均为非负实数。
    \end{enumerate}
\end{proposition}

\begin{proof}
    $\mathbf{A}^T \mathbf{A} \in \mathbb{R}^{n\times n}$ 是一个对称矩阵,根据命题 \ref{prop:real_symmetric_matrix_diagonalizable},
    $\mathbf{A}^T \mathbf{A}$ 是可对角化矩阵,且存在一个正交矩阵 $\mathbf{V} \in \mathbb{R}^{n \times n}$,使得
    \[
        \mathbf{A}^T \mathbf{A} = \mathbf{V} \boldsymbol{\Lambda} \mathbf{V}^T
    \]
    其中,$\boldsymbol{\Lambda}$ 是对角矩阵,对角线上的元素为 $\mathbf{A}^T \mathbf{A}$ 的特征值。根据命题 \ref{prop:real_symmetric_matrix_positive_definite_3},
    $\mathbf{A}^T \mathbf{A}$ 是半正定矩阵,所以其特征值非负。设 $\boldsymbol{\Sigma} = [\sigma_{ij}] \in \mathbb{R}^{m \times n}$ 是一个对角矩阵,
    对角线元素为 $\sigma_{ii} = \sqrt{\lambda_i}$,其中 $\lambda_i$ 是 $\mathbf{A}^T \mathbf{A}$ 的第 $i$ 个特征值,那么令矩阵 $\mathbf{U}$ 满足:
    \begin{equation}
        \mathbf{U}\boldsymbol{\Sigma} = \mathbf{A} \mathbf{V} \in \mathbb{R}^{m \times n}
        \label{eq:u_sigma_av}
    \end{equation}
    那么:
    \begin{equation}
        \boldsymbol{\Sigma}^T\mathbf{U}^T = \mathbf{V}^T \mathbf{A}^T \in \mathbb{R}^{n \times m}
        \label{eq:sigma_t_u_t_atv_t}
    \end{equation}
    将等式 \ref{eq:u_sigma_av} 和 \ref{eq:sigma_t_u_t_atv_t} 等式左右两边分别相乘得:
    \begin{align*}
        \mathbf{U}\boldsymbol{\Sigma}\boldsymbol{\Sigma}^T\mathbf{U}^T & = \mathbf{A} \mathbf{V}\mathbf{V}^T \mathbf{A}^T  \\
        \mathbf{U}\boldsymbol{\Lambda}^{\prime}\mathbf{U}^T & = \mathbf{A} \mathbf{A}^T
    \end{align*}
    其中,$\boldsymbol{\Lambda}^{\prime} \in \mathbb{R}^{m\times m}$ 是对角矩阵,对角线元素为 $\sigma_{ii}^2 = \lambda_i$;
    $\mathbf{A} \mathbf{A}^T$ 也是对称半正定矩阵,根据命题 \ref{prop:real_symmetric_matrix_diagonalizable},$\mathbf{U}$ 是正交矩阵。
    因此,$\mathbf{A} = \mathbf{U} \boldsymbol{\Sigma} \mathbf{V}^T$,其中 $\mathbf{U}$ 和 $\mathbf{V}$ 是正交矩阵,$\boldsymbol{\Sigma}$ 是对角矩阵,
    并且奇异值 $\sigma_{ii} = \sqrt{\lambda_i} \geq 0$。
\end{proof}

\begin{corollary}[奇异值分解与特征值分解的关系]
    设 $\mathbf{A} = \mathbf{U} \boldsymbol{\Sigma} \mathbf{V}^T$ 那么:
    \begin{align*}
        \mathbf{A}^T\mathbf{A} &= [\mathbf{U}\mathbf{\Sigma}\mathbf{V}^T]^T[\mathbf{U}\mathbf{\Sigma}\mathbf{V}^T] \\
        &= \mathbf{V}\mathbf{\Sigma}^T\mathbf{U}^T\mathbf{U}\mathbf{\Sigma}\mathbf{V}^T \\
        &= \mathbf{V}\mathbf{\Sigma}^T\mathbf{\Sigma}\mathbf{V}^T \\
        &= \mathbf{V}\mathbf{\Sigma}^2\mathbf{V}^T
    \end{align*}
    同理可得:
    \begin{align*}
        \mathbf{A}\mathbf{A}^T &= [\mathbf{U}\mathbf{\Sigma}\mathbf{V}^T][\mathbf{U}\mathbf{\Sigma}\mathbf{V}^T]^T \\
        &= \mathbf{U}\mathbf{\Sigma}\mathbf{V}^T\mathbf{V}\mathbf{\Sigma}^T\mathbf{U}^T \\
        &= \mathbf{U}\mathbf{\Sigma}\mathbf{\Sigma}^T\mathbf{U}^T \\
        &= \mathbf{U}\mathbf{\Sigma}^2\mathbf{U}^T
    \end{align*}
\end{corollary}

\begin{note}
    奇异值分解相比特征值分解更为通用,因为它适用于任意矩阵,而不仅限于方阵。
    奇异值分解中,$\mathbf{U}$ 和 $\mathbf{V}$ 是正交矩阵,表示刚性变换,不改变向量的长度和夹角,
    $\boldsymbol{\Sigma}$ 是对角矩阵,表示拉伸或压缩变化。
\end{note}

\vspace{1em}
\subsection{极分解}

\begin{proposition}[极分解 Polar Decomposition]
    任意 $\mathbf{A} \in \mathbb{R}^{n \times n}$ 都可以分解为:
    \[
        \mathbf{A} = \mathbf{O} \mathbf{R}
    \]
    或
    \[
        \mathbf{A} = \mathbf{L} \mathbf{O}
    \]
    其中,$\mathbf{O} \in \mathbb{R}^{n \times n}$ 是一个正交矩阵,$\mathbf{L}, \mathbf{R} \in \mathbb{R}^{n \times n}$ 是对称半正定矩阵。
    称为矩阵的\textbf{极分解}。
    \label{proposition:polar_decomposition}
\end{proposition}

\begin{proof}
    设矩阵 $\mathbf{A}$ 的奇异值 $\mathbf{A} = \mathbf{U} \boldsymbol{\Sigma} \mathbf{V}^T$,那么:
    \[
        \mathbf{A} = (\mathbf{U}\mathbf{V}^T) (\mathbf{V} \boldsymbol{\Sigma} \mathbf{V}^T)
    \]
    令,$\mathbf{O} = \mathbf{U}\mathbf{V}^T$,满足:
    \[
        \mathbf{O}\mathbf{O}^T = \mathbf{U}\mathbf{V}^T\mathbf{V}\mathbf{U}^T = \mathbf{I}
    \]
    是正交矩阵。令
    \[
        \mathbf{R} = \mathbf{V} \boldsymbol{\Sigma} \mathbf{V}^T
    \]
    则 $\mathbf{R}$ 是对称半正定矩阵。类似地,可以得到 $\mathbf{A} = \mathbf{L} \mathbf{O}$ 的分解形式。
    其中,$\mathbf{L} = \mathbf{U} \boldsymbol{\Sigma} \mathbf{U}^T$ 也是对称半正定矩阵。
\end{proof}

\begin{proposition}
    矩阵  $\mathbf{A}$ 的极分解中,$\mathbf{L}, \mathbf{R}$ 满足如下关系:
    \begin{align*}
        \mathbf{L} & = \mathbf{A}\mathbf{O}^{-1} = \mathbf{O} \mathbf{R} \mathbf{O}^T \\
        \mathbf{R} & = \mathbf{O}^{-1}\mathbf{A} = \mathbf{O}^T \mathbf{L} \mathbf{O}
    \end{align*}
    因此,$\mathbf{L}$ 和 $\mathbf{R}$ 是相似矩阵。
\end{proposition}

\begin{note}
    极分解可以看成是奇异值分解的一种特殊情况。
\end{note}

\newpage